\section{Coordenadas}

\subsection{}

{\nologo
\begin{frame}\frametitle{Coordenadas}

\vspace{-2mm}
\begin{prop}{\textbf{Propiedad 1}}
	Sea $\mathcal{B}=\{\mathbf{v}_1, \mathbf{v}_2, \hdots , \mathbf{v}_n \}$ una base de un espacio vectorial $V$. Entonces
	para cada vector $\mathbf{v}\in V$, existen escalares \textit{únicos} $c_1, c_2,\hdots, c_n$	
	tales que
	\[
	\mathbf{v} = c_1 \mathbf{v}_1 + c_2\mathbf{v}_2 + \cdots + c_n\mathbf{v}_n
	\]
\end{prop}	

\begin{defi}{\textbf{Definición 1}}
	Sea $\mathcal{B}=\{\mathbf{v}_1, \mathbf{v}_2, \hdots , \mathbf{v}_n \}$ una base para un espacio \hspace{-1mm} vectorial $V$ y 
	$\mathbf{v}\in V$ \hspace{-1mm} tal que
	
	\vspace{-5mm}
	\[
		\mathbf{v} = c_1 \mathbf{v}_1 + c_2\mathbf{v}_2 + \cdots + c_n\mathbf{v}_n.
	\]
	A los escalares $c_1, c_2,\hdots,c_n$
	se les llama las \textbf{\textit{coordenadas de $\mathbf{v}$ respecto a}} $\mathcal{B}$ y al vector columna
	\[
		\left[ \mathbf{v} \right]_{\mathcal{B}} = 
		\left[
			\begin{array}{c}
				c_1\\
				c_2\\
				\vdots\\
				c_n
			\end{array}
		\right]
	\]
	se le denomina el \textbf{\textit{vector coordenado de $\mathbf{v}$ respecto a}} $\mathcal{B}$.
\end{defi}	

\end{frame}
}
	
%%------------------------------------------------------------------------------------------------------

\subsection{}

{\nologo
\begin{frame}%\frametitle{Coordenadas}

\begin{defi}{\textbf{Definición 1}}
	Sea $\mathcal{B}=\{\mathbf{v}_1, \mathbf{v}_2, \hdots , \mathbf{v}_n \}$ una base para un espacio \hspace{-1mm} vectorial $V$ y 
	$\mathbf{v}\in V$ \hspace{-1mm} tal que
	
	\vspace{-5mm}
	\[
	\mathbf{v} = c_1 \mathbf{v}_1 + c_2\mathbf{v}_2 + \cdots + c_n\mathbf{v}_n.
	\]
	A los escalares $c_1, c_2,\hdots,c_n$
	se les llama las \textbf{\textit{coordenadas de $\mathbf{v}$ respecto a}} $\mathcal{B}$ y al vector columna
	\[
	\left[ \mathbf{v} \right]_{\mathcal{B}} = 
	\left[
	\begin{array}{c}
	c_1\\
	c_2\\
	\vdots\\
	c_n
	\end{array}
	\right]
	\]
	se le denomina el \textbf{\textit{vector coordenado de $\mathbf{v}$ respecto a}} $\mathcal{B}$.
\end{defi}	

\vspace{0mm}

\begin{ej}{\textbf{Ejemplo 1}}\justifying 
	Encuentre el vector coordenado $\left[ x \right]_{\mathcal{B}}$ de $\mathbf{x}=(-3,2,1)$ respecto a
	la base canónica $\mathcal{B} = \left\{ (1,0,0), (0,1,0), (0,0,1) \right\}$ de $\r^3$.
\end{ej}
\textit{Solución.}

\end{frame}
}

%%------------------------------------------------------------------------------------------------------

\subsection{}

{\nologo
\begin{frame}%\frametitle{Coordenadas}
	
	\begin{defi}{\textbf{Definición 1}}
		Sea $\mathcal{B}=\{\mathbf{v}_1, \mathbf{v}_2, \hdots , \mathbf{v}_n \}$ una base para un espacio \hspace{-1mm} vectorial $V$ y 
		$\mathbf{v}\in V$ \hspace{-1mm} tal que
		
		\vspace{-5mm}
		\[
		\mathbf{v} = c_1 \mathbf{v}_1 + c_2\mathbf{v}_2 + \cdots + c_n\mathbf{v}_n.
		\]
		A los escalares $c_1, c_2,\hdots,c_n$
		se les llama las \textbf{\textit{coordenadas de $\mathbf{v}$ respecto a}} $\mathcal{B}$ y al vector columna
		\[
		\left[ \mathbf{v} \right]_{\mathcal{B}} = 
		\left[
		\begin{array}{c}
		c_1\\
		c_2\\
		\vdots\\
		c_n
		\end{array}
		\right]
		\]
		se le denomina el \textbf{\textit{vector coordenado de $\mathbf{v}$ respecto a}} $\mathcal{B}$.
	\end{defi}	
	
	\vspace{0mm}
	
	\begin{ej}{\textbf{Ejemplo 2}}
		Encuentre el vector coordenado $\left[ p(x) \right]_{\mathcal{B}}$ de $p(x)=2-3x+5x^2$ respecto a
		la base canónica $\mathcal{B} = \left\{1,\,x,\,x^2 \right\}$ de $P_2$.
	\end{ej}
	\textit{Solución.}
	
\end{frame}
}

%%------------------------------------------------------------------------------------------------------

\subsection{}

\begin{frame}\frametitle{Coordenadas}

%\begin{defi}{\textbf{Definición 1}}
%	Sea $\mathcal{B}=\{\mathbf{v}_1, \mathbf{v}_2, \hdots , \mathbf{v}_n \}$ una base para un espacio \hspace{-1mm} vectorial $V$ y 
%	$\mathbf{v}\in V$ \hspace{-1mm} tal que
%	
%	\vspace{-5mm}
%	\[
%	\mathbf{v} = c_1 \mathbf{v}_1 + c_2\mathbf{v}_2 + \cdots + c_n\mathbf{v}_n.
%	\]
%	A los escalares $c_1, c_2,\hdots,c_n$
%	se les llama las \textbf{\textit{coordenadas de $\mathbf{v}$ respecto a}} $\mathcal{B}$ y al vector columna
%	\[
%	\left[ \mathbf{v} \right]_{\mathcal{B}} = 
%	\left[
%	\begin{array}{c}
%	c_1\\
%	c_2\\
%	\vdots\\
%	c_n
%	\end{array}
%	\right]
%	\]
%	se le denomina el \textbf{\textit{vector coordenado de $\mathbf{v}$ respecto a}} $\mathcal{B}$.
%\end{defi}	

\vspace{0mm}

\begin{ej}{\textbf{Ejemplo 3}}
	Encuentre el vector coordenado $\left[ A \right]_{\mathcal{B}}$ de 
	\[
	A = 
	\left( 
	\begin{array}{@{\hspace{1\tabcolsep}}rr}	
	2 & -1 \\[2mm] 
	4 & 3
	\end{array} 
	\right)
	\]
	respecto a la base canónica $\mathcal{B} = \left\{ E_{11},\ E_{12},\ E_{21},\ E_{22} \right\}$ de $M_{22}$.
\end{ej}
\textit{Solución.}

\end{frame}

%%------------------------------------------------------------------------------------------------------

\subsection{}

\begin{frame}\frametitle{Coordenadas}
	
%	\begin{defi}{\textbf{Definición 1}}
%		Sea $\mathcal{B}=\{\mathbf{v}_1, \mathbf{v}_2, \hdots , \mathbf{v}_n \}$ una base para un espacio \hspace{-1mm} vectorial $V$ y 
%		$\mathbf{v}\in V$ \hspace{-1mm} tal que
%		
%		\vspace{-5mm}
%		\[
%		\mathbf{v} = c_1 \mathbf{v}_1 + c_2\mathbf{v}_2 + \cdots + c_n\mathbf{v}_n.
%		\]
%		A los escalares $c_1, c_2,\hdots,c_n$
%		se les llama las \textbf{\textit{coordenadas de $\mathbf{v}$ respecto a}} $\mathcal{B}$ y al vector columna
%		\[
%		\left[ \mathbf{v} \right]_{\mathcal{B}} = 
%		\left[
%		\begin{array}{c}
%		c_1\\
%		c_2\\
%		\vdots\\
%		c_n
%		\end{array}
%		\right]
%		\]
%		se le denomina el \textbf{\textit{vector coordenado de $\mathbf{v}$ respecto a}} $\mathcal{B}$.
%	\end{defi}	
	
	\vspace{0mm}
	
	\begin{ej}{\textbf{Ejemplo 4}}\justifying 
		El vector coordenado de $\mathbf{x}$ en $\r^2$ respecto a la base (no estándar) $\mathcal{B} = \left\{ (1,0), (1,2) \right\}$ 
		es
		
		\vspace{-3mm}
		\[
		\left[ \mathbf{x} \right]_{\mathcal{B}} = 
		\left[
		\begin{array}{c}
			3\\
			2
		\end{array}
		\right].
		\]
		
		%\vspace{-2mm}
		Determine el vector de coordenadas de $\mathbf{x}$ en $\r^2$ respecto a la base canónica.
	\end{ej}
	\textit{Solución.}
	
\end{frame}

%%------------------------------------------------------------------------------------------------------

\subsection{}

\begin{frame}\frametitle{Coordenadas}
	
%	\begin{defi}{\textbf{Definición 1}}
%		Sea $\mathcal{B}=\{\mathbf{v}_1, \mathbf{v}_2, \hdots , \mathbf{v}_n \}$ una base para un espacio \hspace{-1mm} vectorial $V$ y 
%		$\mathbf{v}\in V$ \hspace{-1mm} tal que
%		
%		\vspace{-5mm}
%		\[
%		\mathbf{v} = c_1 \mathbf{v}_1 + c_2\mathbf{v}_2 + \cdots + c_n\mathbf{v}_n.
%		\]
%		A los escalares $c_1, c_2,\hdots,c_n$
%		se les llama las \textbf{\textit{coordenadas de $\mathbf{v}$ respecto a}} $\mathcal{B}$ y al vector columna
%		\[
%		\left[ \mathbf{v} \right]_{\mathcal{B}} = 
%		\left[
%		\begin{array}{c}
%		c_1\\
%		c_2\\
%		\vdots\\
%		c_n
%		\end{array}
%		\right]
%		\]
%		se le denomina el \textbf{\textit{vector coordenado de $\mathbf{v}$ respecto a}} $\mathcal{B}$.
%	\end{defi}	
	
	\vspace{0mm}
	
	\begin{ej}{\textbf{Ejemplo 5}}\justifying 
		Encuentre el vector de coordenadas de $\left[ x \right]_{\mathcal{B}}$ de $\mathbf{x}=(1,2,-1)$ 
		en $\r^3$ relativo a la base (no estándar) $\mathcal{B} = \left\{ (1,0,1), (0,-1,2), (2,3,-5) \right\}$ de $\r^3$.
	\end{ej}
	\textit{Solución.}
	
\end{frame}

%%------------------------------------------------------------------------------------------------------

\subsection{}

\begin{frame}\frametitle{Coordenadas}

%\begin{defi}{\textbf{Definición 1}}
%	Sea $\mathcal{B}=\{\mathbf{v}_1, \mathbf{v}_2, \hdots , \mathbf{v}_n \}$ una base para un espacio \hspace{-1mm} vectorial $V$ y 
%	$\mathbf{v}\in V$ \hspace{-1mm} tal que
%	
%	\vspace{-5mm}
%	\[
%	\mathbf{v} = c_1 \mathbf{v}_1 + c_2\mathbf{v}_2 + \cdots + c_n\mathbf{v}_n.
%	\]
%	A los escalares $c_1, c_2,\hdots,c_n$
%	se les llama las \textbf{\textit{coordenadas de $\mathbf{v}$ respecto a}} $\mathcal{B}$ y al vector columna
%	\[
%	\left[ \mathbf{v} \right]_{\mathcal{B}} = 
%	\left[
%	\begin{array}{c}
%	c_1\\
%	c_2\\
%	\vdots\\
%	c_n
%	\end{array}
%	\right]
%	\]
%	se le denomina el \textbf{\textit{vector coordenado de $\mathbf{v}$ respecto a}} $\mathcal{B}$.
%\end{defi}	

\vspace{0mm}

\begin{ej}{\textbf{Ejemplo 6}}\justifying
	Encuentre el vector coordenado $\left[ p(x) \right]_{\mathcal{C}}$ de $p(x)=1+2x-x^2$ respecto a
	la base $\mathcal{C} = \left\{1+x,\, x+x^2,\, 1+x^2 \right\}$ de $P_2$.
\end{ej}
\textit{Solución.}

\end{frame}

%%------------------------------------------------------------------------------------------------------

\subsection{}

\begin{frame}\frametitle{Coordenadas}

\begin{prop}{\textbf{Propiedad 2}}
	Sea $\mathcal{B}=\{\mathbf{v}_1, \mathbf{v}_2, \hdots , \mathbf{v}_n \}$ una base para un espacio vectorial $V$. 
	Sean $\mathbf{u}$ y $\mathbf{v}$ vectores en $V$ y $c$ un escalar. Entonces:
	\begin{enumerate}
		\item[\labelname{$a$}] $\left[ \mathbf{u} +\mathbf{v} \right]_{\mathcal{B}} = \left[ \mathbf{u} \right]_{\mathcal{B}} + \left[ \mathbf{v} \right]_{\mathcal{B}}$
		\item[\labelname{$a$}] $\left[ c \mathbf{u} \right]_{\mathcal{B}} = c\left[ \mathbf{u} \right]_{\mathcal{B}}$
	\end{enumerate}
\end{prop}	

\end{frame}

%%------------------------------------------------------------------------------------------------------

\subsection{}

\begin{frame}\frametitle{Coordenadas}
	
	\begin{prop}{\textbf{Propiedad 3}} \justifying
		Sea $\mathcal{B}=\{\mathbf{v}_1, \mathbf{v}_2, \hdots , \mathbf{v}_n \}$ una base para un espacio vectorial $V$
		y sean $\mathbf{u}_1, \mathbf{u}_2, \hdots , \mathbf{u}_k$ vectores en $V$. Entonces el conjunto de vectores
		$\{\mathbf{u}_1, \mathbf{u}_2, \hdots , \mathbf{u}_k \}$ es linealmente independiente (LI) en $V$ si y sólo si
		\[
			\{{[\mathbf{u}_1]}_{\mathcal{B}}, {[\mathbf{u}_2]}_{\mathcal{B}}, \hdots , {[\mathbf{u}_k]}_{\mathcal{B}} \}
		\]
		es linealmente independiente (LI) en $\r^n$.
	\end{prop}	
	
\end{frame}


