\section{Isometrías}

\subsection{}

\begin{frame}\frametitle{Matrices ortogonales}

	\begin{block}{\textbf{Definición 1  }}
		\justifying
		Una matriz cuadrada $Q$ se dice que es  \textbf{\textit{ortogonal}} si es invertible y si 
		\[
			Q^{-1} = Q^T.
		\]
	\end{block}
	
	\begin{alertblock}{\textbf{Observación 1}}
		Si $Q$ es una matriz cuadrada $n\times n$ invertible y
		\[
			 Q^TQ = I_n,
		\]
		entonces $Q$ es ortogonal.
	\end{alertblock}
	
\end{frame}

% ---------------------------------------------------------------------------------------------------

\subsection{}

\begin{frame}%\frametitle{Matrices ortogonales}
	
	\begin{block}{\textbf{Definición 1  }}
		\justifying
		Una matriz cuadrada $Q$ se dice que es  \textbf{\textit{ortogonal}} si es invertible y si 
		\[
		Q^{-1} = Q^T.
		\]
	\end{block}
	
	\begin{ej}{\textbf{Ejemplo 1}}
		\justifying
		Determine cuáles de las siguientes matrices son ortogonales.
		\[
			A = 
			\left(
			\begin{array}{@{\hspace{0\tabcolsep}}r@{\hspace{\tabcolsep}}r@{\hspace{0.3\tabcolsep}}}
			0 & 1  \\[1mm]
			-1 & 0 
			\end{array}
			\right),\
			%\quad 
			B = 
			\left(
			\begin{array}{@{\hspace{0.3\tabcolsep}}r@{\hspace{\tabcolsep}}r@{\hspace{0.3\tabcolsep}}}
			3 & -2  \\[1mm]
			2 & -1 
			\end{array}
			\right),\
			%\quad 
			C = 
			\left(
			\begin{array}{@{\hspace{0.3\tabcolsep}}r@{\hspace{\tabcolsep}}r@{\hspace{0.5\tabcolsep}}}
			\cos\theta & -\sen\theta  \\[1mm]
			\sen\theta & \cos\theta 
			\end{array}
			\right),\
			D = %\hspace{-1mm}
			\left(
			\begin{array}{@{\hspace{0.3\tabcolsep}}r@{\hspace{1.5\tabcolsep}}r@{\hspace{1.5\tabcolsep}}r@{\hspace{0.3\tabcolsep}}}
			0 & 1 & 0 \\[1mm]
			1 & 0 & 0\\[1mm]
			0 & 0 & 1
			\end{array}
			\right)
		\]

	\end{ej}
	\textit{Solución.}
	
\end{frame}


% ---------------------------------------------------------------------------------------------------

\subsection{}

\begin{frame}\frametitle{Propiedades de las matrices ortogonales}

%\begin{defi}{}\justifying
%	Recordemos que el \textbf{\textit{producto punto}} o \textbf{\textit{producto escalar}} de dos vectores
%	\[
%	\mathbf{u}=(u_1,\hdots,u_n) \qquad \text{y} \qquad \mathbf{v}=(v_1,\hdots,v_n)
%	\]
%	de $\r^n$ se define como
%	\[
%	\mathbf{u} \cdot \mathbf{v} = u_1 v_1 + u_2 v_2 +  \cdots + u_n v_n.
%	\]
%\end{defi}	

\begin{prop}{\textbf{Propiedad 1}}\justifying
	Una matriz cuadrada $Q$ es ortogonal si y sólo si sus vectores columnas forman un conjunto \textbf{ortonormal}.
\end{prop}	

\begin{ej}{\textbf{Ejemplo 2}} \justifying
	Determine si el conjunto de vectores dado a continuación es ortonormal.
	\[	
	S = 
	\left\{ \begin{array}{c@{\hspace{-\tabcolsep}}} \phantom{.}\\ \phantom{.}\\ \phantom{.} \\ \phantom{.} \end{array} \right.
	\underbrace{\left(\begin{array}{r} \frac{2}{3}\\[1mm] \frac{2}{3}\\[1mm] \frac{1}{3} \\  \end{array} \right)}_{\color{blue}\mathbf{v}_1},\
	\underbrace{\left(\begin{array}{r} -\frac{2}{3}\\[1mm] \frac{1}{3}\\[1mm] \frac{2}{3} \\  \end{array} \right)}_{\color{blue}\mathbf{v}_1},\
	\underbrace{\left(\begin{array}{r} \frac{1}{3}\\[1mm] -\frac{2}{3}\\[1mm] \frac{2}{3} \\  \end{array} \right)}_{\color{blue}\mathbf{v}_1}\
	\left. \begin{array}{c@{\hspace{-\tabcolsep}}} \phantom{.}\\ \phantom{.}\\ \phantom{.} \\ \phantom{.} \end{array} \right\}
	\]
\end{ej}	
\textit{Solución.}

\end{frame}

% ---------------------------------------------------------------------------------------------------

\subsection{}

{\nologo 
\begin{frame}\frametitle{Propiedades del producto punto}
	
	\begin{alertblock}{\textbf{Observación 2}}\justifying
		Si los vectores $\mathbf{u}=(u_1,\hdots,u_n) $ y $\mathbf{v}=(v_1,\hdots,v_n)$ de $\r^n$ 
		los representamos como vectores columna, entonces el \textbf{\textit{producto punto}} o \textbf{\textit{producto escalar}} de ellos
		se puede expresar como el producto de matrices
		\[
		\mathbf{u}\cdot \mathbf{v} =
		\mathbf{u}^T \mathbf{v} =
		\left(
		\begin{array}{cccc}
		u_1 & u_2 & \cdots & u_n \\
		\end{array}
		\right) 
		\left(
		\begin{array}{c}
		v_1\\
		v_2\\
		\vdots \\[1mm]
		v_n
		\end{array}
		\right)
		= u_1 v_1 +  \cdots + u_nv_n. 
		\]
	\end{alertblock}
	
	\begin{prop}{\textbf{Propiedad 2}}
		\justifying
		Sea $A$ una matriz $m\times n$ con entradas reales. Entonces para todo vector $\mathbf{x}$ en $\r^n$ y 
		todo vector $\mathbf{y}$ en $\r^m$,
		\[
			\left(A\mathbf{x}\right)\cdot \mathbf{y} = \mathbf{x}\cdot \left(A^T\mathbf{y}\right)
		\]
	\end{prop}	
	
\end{frame}
}

% ---------------------------------------------------------------------------------------------------

\subsection{}

\begin{frame}%\frametitle{Isometrías}

	\begin{prop}{\textbf{Propiedad 2}}
	\justifying
	Sea $A$ una matriz $m\times n$ con entradas reales. Entonces para todo vector $\mathbf{x}$ en $\r^n$ y 
	todo vector $\mathbf{y}$ en $\r^m$,
	
	\vspace{-2mm}
	\[
	\left(A\mathbf{x}\right)\cdot \mathbf{y} = \mathbf{x}\cdot \left(A^T\mathbf{y}\right)
	\]
\end{prop}	

\begin{prop}{\textbf{Propiedad 3}}
	\justifying
	Sea $Q$ una matriz $n\times n$. Las siguientes afirmaciones son equivalentes (si una es verdadera, todas
	las otras también son verdaderas). 
	\begin{enumerate}[$a$]\justifying 
		\item $Q$ es ortogonal.
		\item $\Vert Q\mathbf{x}\Vert = \Vert \mathbf{x}\Vert $ para todo $\mathbf{x} $ en $\r^n$.
		\item $ Q\mathbf{x}\cdot Q\mathbf{y} = \mathbf{x}\cdot \mathbf{y} $ para todo 
		$\mathbf{x}$ y $\mathbf{y}$ en $\r^n$.
	\end{enumerate}
\end{prop}	

%\begin{block}{\textbf{Definición 2 (Isometría) }}
%	\justifying
%	Una transformación lineal $T:\r^n\to \r^n$ se dice que es una \textbf{\textit{isometría}} si
%	para todo vector en $\mathbf{x}$ en $\r^n$,
%	
%	\vspace{-4mm}
%	\[
%		\Vert T(\mathbf{x})\Vert = \Vert \mathbf{x}\Vert.
%	\]
%\end{block}

\end{frame}

% ---------------------------------------------------------------------------------------------------

\subsection{}

\begin{frame}\frametitle{Isometrías}
			
	\begin{block}{\textbf{Definición 2 (Isometría) }}
		\justifying
		Una transformación lineal $T:\r^n\to \r^n$ se dice que es una \textbf{\textit{isometría}} si para
		todo vector $\mathbf{x}$ en $\r^n$,
		
		\vspace{-4mm}
		\[
		\Vert T(\mathbf{x})\Vert = \Vert \mathbf{x}\Vert.
		\]
	\end{block}
	
	\begin{alertblock}{\textbf{Observación 3}}
		Si  $T:\r^n\to \r^n$ es una isometría, entonces para todo vector 
		$\mathbf{x}$ y $\mathbf{y}$ en $\r^n$,
		
		\vspace{-0mm}
		\[
		\Vert T(\mathbf{x})-T(\mathbf{y})\Vert = \Vert \mathbf{x}-\mathbf{y}\Vert.
		\]
	\end{alertblock}

\end{frame}

% ---------------------------------------------------------------------------------------------------

\subsection{}

\begin{frame}\frametitle{Isometrías}
	
	\begin{block}{\textbf{Definición 2 (Isometría) }}
		\justifying
		Una transformación lineal $T:\r^n\to \r^n$ se dice que es una \textbf{\textit{isometría}} si para
		todo vector $\mathbf{x}$ en $\r^n$,
		
		\vspace{-4mm}
		\[
		\Vert T(\mathbf{x})\Vert = \Vert \mathbf{x}\Vert.
		\]
	\end{block}	
	
	\begin{prop}{\textbf{Propiedad 4}}			
		\justifying
		Si  $T:\r^n\to \r^n$ es una isometría, entonces para todo vector 
		$\mathbf{x}$ y $\mathbf{y}$ en $\r^n$,
		
		\vspace{-0mm}
		\[
			T(\mathbf{x})\cdot T(\mathbf{y}) = \mathbf{x}\cdot \mathbf{y}.
		\]
	\end{prop}	
	
\end{frame}

% ---------------------------------------------------------------------------------------------------

\subsection{}

\begin{frame}%\frametitle{Isometrías}
	
%	\begin{prop}{\textbf{Propiedad 3}}
%		\justifying
%		Sea $Q$ una matriz $n\times n$. Entonces las siguientes afirmaciones son equivalentes (si una es verdadera, todas
%		las otras también son verdaderas). 
%		\begin{enumerate}[$a$]\justifying 
%			\item $Q$ es ortogonal.
%			\item $\Vert Q\mathbf{x}\Vert = \Vert \mathbf{x}\Vert $ para todo $\mathbf{x} $ en $\r^n$.
%			\item $\Vert Q\mathbf{x}\cdot Q\mathbf{y}\Vert = \Vert \mathbf{x}\Vert $ para todo 
%			$\mathbf{x}$ y $\mathbf{y}$ en $\r^n$.
%		\end{enumerate}
%	\end{prop}	
	
	\begin{block}{\textbf{Definición 2 (Isometría) }}
		\justifying
		Una transformación lineal $T:\r^n\to \r^n$ se dice que es una \textbf{\textit{isometría}} si para 
		todo vector $\mathbf{x}$ en $\r^n$,
		
		\vspace{-4mm}
		\[
		\Vert T(\mathbf{x})\Vert = \Vert \mathbf{x}\Vert.
		\]
	\end{block}
	
	\begin{ej}{\textbf{Ejemplo 3}}
		\justifying
		Determine si función $T:\r^2\to \r^2$ definida a continuación es isometría.
		\[
		T
		\left(
		\begin{array}{@{\hspace{0.3\tabcolsep}}c@{\hspace{0.3\tabcolsep}}}
		x \\[1mm]
		y
		\end{array}
		\right)
		= 
		\left(
		\begin{array}{@{\hspace{0.3\tabcolsep}}r@{\hspace{1.2\tabcolsep}}r@{\hspace{0.3\tabcolsep}}}
		\frac{\sqrt{2}}{2} & -\frac{\sqrt{2}}{2} \\[1mm]
		\frac{\sqrt{2}}{2} & \frac{\sqrt{2}}{2} 
		\end{array}
		\right)
		\left(
		\begin{array}{@{\hspace{0.3\tabcolsep}}c@{\hspace{0.3\tabcolsep}}}
		x \\[2mm]
		y
		\end{array}
		\right).
		\]		
	\end{ej}
	\textit{Solución.}
	
\end{frame}

% ---------------------------------------------------------------------------------------------------

\subsection{}

\begin{frame}\frametitle{Caracterización de las isometrías}

\begin{prop}{\textbf{Propiedad 5}}
	\justifying
	Una transformación lineal $T:\r^n\to \r^n$ es una isometría si y sólo si su
	matriz de representación estándar $A_T$ es ortogonal.	
\end{prop}	

\begin{alertblock}{\textbf{Observación 4}}
	Toda isometría es un isomorfismo.
\end{alertblock}

\end{frame}

% ---------------------------------------------------------------------------------------------------

\subsection{}

\begin{frame}\frametitle{Construcción de isometrías}
	
	\begin{prop}{\textbf{Propiedad 6}}
		\justifying
		Sean $\{ \mathbf{u}_1,\hdots, \mathbf{u}_n \}$ y $\{ \mathbf{w}_1,\hdots, \mathbf{w}_n \}$ bases
		ortogonormales de $\r^n$ y $T:\r^n\to \r^n$ la transformación lineal definida por
		\[
			T(\mathbf{u}_i) = \mathbf{w}_i,
		\]
		para $i=1,\hdots,n$. Entonces $T$ es una isometría.
	\end{prop}	
	
\end{frame}

% ---------------------------------------------------------------------------------------------------

\subsection{}

\begin{frame}\frametitle{Propiedades de las isometrías}
		
	\begin{prop}{\textbf{Propiedad 7}}
		\justifying
		Sea $T:\r^n\to \r^n$ una isometría. Si $\{ \mathbf{u}_1,\hdots, \mathbf{u}_n \}$ es una base ortonormal 
		de $\r^n$, entonces $\{ T(\mathbf{u}_1),\hdots, T(\mathbf{u}_n) \}$ es una base ortonormal de $\r^n$.
	\end{prop}	
	
\end{frame}

% ---------------------------------------------------------------------------------------------------

\subsection{}

\begin{frame}\frametitle{Isometrías en $\r^2$}
	
%	\begin{prop}{\textbf{Propiedad 5}}
%		\justifying
%		Una transformación lineal $T:\r^n\to \r^n$ es una isometría si y sólo si su
%		matriz de representación estándar $A_T$ es ortogonal.	
%	\end{prop}	
	
	\begin{prop}{\textbf{Propiedad 8}}
		\justifying
		Si $T:\r^2\to \r^2$ es una isometría, entonces $T$ es:
		\begin{enumerate}[$a$]
			\item una transformación de rotación, 
			\[
			T
			\left(
			\begin{array}{@{\hspace{0.3\tabcolsep}}c@{\hspace{0.3\tabcolsep}}}
			x \\[1mm]
			y
			\end{array}
			\right)
			= 
			\left(
			\begin{array}{@{\hspace{0.3\tabcolsep}}r@{\hspace{1.2\tabcolsep}}r@{\hspace{0.3\tabcolsep}}}
			\cos\theta & -\sen\theta \\[1mm]
			\sen\theta & \cos\theta
			\end{array}
			\right)
			\left(
			\begin{array}{@{\hspace{0.3\tabcolsep}}c@{\hspace{0.3\tabcolsep}}}
			x \\[2mm]
			y
			\end{array}
			\right),
			\]		
			\item o bien una reflexión respecto al eje $x$, 
			\[		
			\left(
			\begin{array}{@{\hspace{0.3\tabcolsep}}r@{\hspace{1.2\tabcolsep}}r@{\hspace{0.3\tabcolsep}}}
			1 & 0 \\[1mm]
			0 & -1
			\end{array}
			\right)
			\left(
			\begin{array}{@{\hspace{0.3\tabcolsep}}c@{\hspace{0.3\tabcolsep}}}
			x \\[2mm]
			y
			\end{array}
			\right)
			=
			\left(
			\begin{array}{@{\hspace{0.3\tabcolsep}}r@{\hspace{0.3\tabcolsep}}}
			x \\[1mm]
			-y
			\end{array}
			\right),
			\]				
			seguida de una transformación de rotación.
		\end{enumerate}
	\end{prop}	
	
\end{frame}
