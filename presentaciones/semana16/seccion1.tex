\section{Jordan}

\subsection{}

{\nologo
\begin{frame}%\frametitle{Introducción}

\begin{block}{\textbf{Definición 1 (Matriz de bloques de Jordan)}}
	\justifying
	Por $N_k$ denotamos a la matriz $k\times k$ que tiene unos arriba de la diagonal principal y ceros en las demás
	posiciones:
	\[
	N_k = 
	\left(
	\begin{array}{@{\hspace{0.1\tabcolsep}}c@{\hspace{1.5\tabcolsep}}c@{\hspace{1.5\tabcolsep}}c@{\hspace{1.5\tabcolsep}}c@{\hspace{1.5\tabcolsep}}c@{\hspace{0.1\tabcolsep}}}
	     0 & {\color{red}1} & 0 & \cdots & 0 \\[2mm]
	     0 & 0              & {\color{red}1} & \cdots & 0 \\[0mm]
	\vdots & \vdots         &  \ddots         & \ddots & \vdots \\[1mm]
  		 0 & 0              & \cdots  & 0 & {\color{red}1} \\[2mm]
  		 0 & 0              & \cdots  & 0 & 0 \\[0mm]
	\end{array}
	\right) 	
	\]
	
	Con la matriz $N_k$ definimos la \textit{matriz de bloques de Jordan}
	\[
	B(\lambda) = \lambda I + N_k = 
	\left(
	\begin{array}{@{\hspace{0.1\tabcolsep}}c@{\hspace{1.5\tabcolsep}}c@{\hspace{1.5\tabcolsep}}c@{\hspace{1.5\tabcolsep}}c@{\hspace{1.5\tabcolsep}}c@{\hspace{0.1\tabcolsep}}}
	\lambda & {\color{red}1} & 0 & \cdots & 0 \\[2mm]
	0 & \lambda              & {\color{red}1} & \cdots & 0 \\[0mm]
	\vdots & \vdots         &  \ddots         & \ddots & \vdots \\[1mm]
	0 & 0              & \cdots  & \lambda & {\color{red}1} \\[2mm]
	0 & 0              & \cdots  & 0 & \lambda \\[0mm]
	\end{array}
	\right) 	
	\]
\end{block}

\end{frame}
}

%------------------------------------------------------------------------------------------------------

\subsection{}

\begin{frame}%\frametitle{Introducción}
	
	\begin{block}{\textbf{Definición 2 (Matriz de Jordan)}}
		\justifying	
		Definimos la \textit{matriz de Jordan}
		\[
		J = 
		\left(
		\begin{array}{@{\hspace{0.1\tabcolsep}}c@{\hspace{1.5\tabcolsep}}c@{\hspace{1.5\tabcolsep}}c@{\hspace{1.5\tabcolsep}}c@{\hspace{0.1\tabcolsep}}}
		B_1(\lambda_1) & 0  & \cdots  & 0 \\[2mm]
		           0 & B_2(\lambda_2) & \cdots & 0 \\[0mm]
		      \vdots &     \vdots   & \ddots & \vdots \\[1mm]
		           0 &          0   & \cdots       & B_r(\lambda_r)  \\[2mm]
		\end{array}
		\right),
		\]
		donde cada $B_j(\lambda_j)$ es una matriz de bloques de Jordan.
	\end{block}
	
	
\end{frame}

%%------------------------------------------------------------------------------------------------------

\subsection{}

\begin{frame}\frametitle{Ejemplo de matrices de Jordan}
	
	\[
	J = 
	\left(
	\begin{array}{@{\hspace{0.1\tabcolsep}}c@{\hspace{\tabcolsep}}c@{\hspace{\tabcolsep}}c@{\hspace{\tabcolsep}}c@{\hspace{\tabcolsep}}c@{\hspace{0.1\tabcolsep}}}
	3 & {\color{red}1} & \vdots & 0 \\[0mm]
	0 & 3 & \vdots & 0 \\[-0mm]
	\cdots & \cdots  & \cdot & \cdots \\[-2mm]
	0 & 0 & \vdots & 4 \\
	\end{array}
	\right) 	
	\]
	
\end{frame}

%%------------------------------------------------------------------------------------------------------

\subsection{}

\begin{frame}\frametitle{Ejemplo de matrices de Jordan}
	
	\[
	J = 
	\left(
	\begin{array}{@{\hspace{0.1\tabcolsep}}r@{\hspace{\tabcolsep}}r@{\hspace{0.5\tabcolsep}}r@{\hspace{\tabcolsep}}r@{\hspace{\tabcolsep}}r@{\hspace{\tabcolsep}}c@{\hspace{\tabcolsep}}l@{\hspace{\tabcolsep}}l@{\hspace{0.1\tabcolsep}}}
	-2 & \vdots & 0 & 0 & 0 & 0 & 0\\[0mm]
	\cdots & \cdot & \cdots  & \cdots & \cdots & \cdot & & \\[-2mm]
	0 & \vdots & -2 & {\color{red}1}   & 0 & \vdots  & 0\\[-0mm]
	0 & \vdots & 0  & -2               & {\color{red}1} & \vdots  & 0\\[-0mm]
	0 & \vdots & 0  & 0                & -2 & \vdots  & 0\\[-0mm]
	& \cdot & \cdots  & \cdots & \cdots & \cdots  & \cdots & \\[-2mm]
	-2 &  & 0 & 0 & 0 & \vdots & 7\\[0mm]
	\end{array}
	\right) 	
	\]
	
\end{frame}

%%------------------------------------------------------------------------------------------------------

\subsection{}

\begin{frame}\frametitle{Ejemplo de matrices de Jordan}
	
	\[
	J = 
	\left(
	\begin{array}{@{\hspace{0.1\tabcolsep}}r@{\hspace{\tabcolsep}}r@{\hspace{\tabcolsep}}r@{\hspace{0.8\tabcolsep}}c@{\hspace{\tabcolsep}}c@{\hspace{\tabcolsep}}c@{\hspace{\tabcolsep}}l@{\hspace{\tabcolsep}}r@{\hspace{1.5\tabcolsep}}r@{\hspace{0.1\tabcolsep}}}
	3 & {\color{red} 1} & \vdots & 0 & 0 & 0 &  & 0 & 0 \\[0mm]
	0 & 3 & \vdots & 0 & 0 & 0 &  & 0 & 0 \\[0mm]
	\cdots & \cdots & \cdot & \cdots & \cdots & \cdots & \cdot &  &  \\[-2mm]
	0 & 0 & \vdots & 4 & {\color{blue} 1} & 0 & \vdots & 0 & 0 \\[0mm]
	0 & 0 & \vdots & 0 & 4 & {\color{blue} 1} & \vdots & 0 & 0 \\[0mm]
	0 & 0 & \vdots & 0 & 0 & 4 & \vdots & 0 & 0 \\[0mm]
	 & & \cdot & \cdots & \cdots & \cdots & \cdot & \cdots & \cdots \\[-2mm]	
	0 & 0 &  & 0 & 0 & 0 & \vdots & 2 & {\color{green} 1} \\[0mm]
	0 & 0 &  & 0 & 0 & 0 & \vdots & 0 & 2 \\[0mm]
	\end{array}
	\right) 	
	\]
	
\end{frame}

%%------------------------------------------------------------------------------------------------------

\subsection{}

{\nologo
\begin{frame}%\frametitle{Matrices de Jordan}

\vspace{-3mm}
\begin{prop}{\textbf{Propiedad 1 (Matrices de Jordan $2\times2$)}}
	Las únicas matrices de Jordan de $2\times 2$ son de la forma
	\[
	J =
	\left(
		\begin{array}{cc}
			\lambda_1 & 0\\[1mm]
			0 & \lambda_2
		\end{array}
	\right)
	\quad \text{ó} \qquad
	J =
	\left(
	\begin{array}{cc}
	\lambda & {\color{red}1}\\[1mm]
	0 & \lambda
	\end{array}
	\right)
	\]
\end{prop}

\vspace{-1mm}
\begin{prop}{\textbf{Propiedad 2 (Matrices de Jordan $3\times 3$)}}
	Las únicas matrices de Jordan de $3\times 3$ son de la forma
	\[
	J =
	\left(
	\begin{array}{ccc}
	\lambda_1 & 0 & 0\\[1mm]
	0 & \lambda_2 & 0\\[1mm]
	0 & 0 & \lambda_3
	\end{array}
	\right)
	\quad \text{ó} \qquad
	J =
	\left(
	\begin{array}{@{\hspace{0.1\tabcolsep}}c@{\hspace{0.8\tabcolsep}}c@{\hspace{\tabcolsep}}c@{\hspace{\tabcolsep}}c@{\hspace{0.1\tabcolsep}}}
	\lambda_1 & \vdots & 0 & 0 \\[0mm]
	\cdots & \cdot  & \cdots & \cdots \\[-2mm]
	0 & \vdots & \lambda_2 & {\color{red}1} \\[-0.5mm]		
	0 & \vdots & 0 & \lambda_2 \\
	\end{array}
	\right)
	\]
	
	\vspace{-2mm}
	ó
	\[
	J =
	\left(
	\begin{array}{@{\hspace{0.1\tabcolsep}}c@{\hspace{\tabcolsep}}c@{\hspace{\tabcolsep}}c@{\hspace{\tabcolsep}}c@{\hspace{\tabcolsep}}c@{\hspace{0.1\tabcolsep}}}
	\lambda_1 & {\color{red}1} & \vdots & 0 \\[0mm]
	0 & \lambda_1 & \vdots & 0 \\[-0mm]
	\cdots & \cdots  & \cdot & \cdots \\[-2mm]
	0 & 0 & \vdots & \lambda_2 \\
	\end{array}
	\right) 	
	\quad \text{ó} \qquad
	J =
	\left(
	\begin{array}{ccc}
	\lambda_1 & {\color{red}1} & 0\\[1mm]
	0 & \lambda_1 & {\color{red}1}\\[1mm]
	0 & 0 & \lambda_1
	\end{array}
	\right)
	\]
\end{prop}

\end{frame}
}

%%------------------------------------------------------------------------------------------------------

\subsection{} 

{\nologo
\begin{frame}%\frametitle{Forma canónica de Jordan}

	\vspace{-2mm}
	\begin{prop}{\textbf{Propiedad 3} (Forma canónica de Jordan)}\justifying
	Sea $A$ una matriz $n\times n$ con entradas reales o complejas. Entonces existe una matriz invertible $C$ de 
	$n\times n$ con entradas complejas tal que
	\[
		C^{-1}AC = J,
	\]
	donde $J$ es una matriz de Jordan cuyos elementos en la diagonal son los valores propios de $A$. 
	Más aún, la matriz de Jordan $J$ es única, excepto por el orden en el que aparecen los bloques de Jordan.
	\end{prop}
	
	\vspace{-1mm}
	\begin{alertblock}{\textbf{Observación 1}}
	
		\begin{enumerate}[$a$]
			\item Las entradas de $C$ en la propiedad 3 pueden ser reales.
			\item La matriz $C$ en la propiedad 3 no necesariamente es única.
			
			\item En la propiedad 3, si por ejemplo $A$ es semejante a
			\[
			J = 
			\left(
			\begin{array}{@{\hspace{0.1\tabcolsep}}c@{\hspace{\tabcolsep}}c@{\hspace{\tabcolsep}}c@{\hspace{\tabcolsep}}c@{\hspace{\tabcolsep}}c@{\hspace{0.1\tabcolsep}}}
			3 & {\color{red}1} & \vdots & 0 \\[0mm]
			0 & 3 & \vdots & 0 \\[-0mm]
			\cdots & \cdots  & \cdot & \cdots \\[-2mm]
			0 & 0 & \vdots & 4 \\
			\end{array}
			\right), 	
			\quad \text{también lo es } \quad 
			J = 
			\left(
			\begin{array}{@{\hspace{0.1\tabcolsep}}c@{\hspace{0.8\tabcolsep}}c@{\hspace{0.8\tabcolsep}}c@{\hspace{\tabcolsep}}c@{\hspace{0.1\tabcolsep}}}
			4 & \vdots & 0 & 0 \\[0mm]
			\cdots & \cdot  & \cdots & \cdots \\[-2mm]
			0 & \vdots & 3 & {\color{red}1} \\[-0mm]		
			0 & \vdots & 0 & 3 \\
			\end{array}
			\right) 	
			\]
		\end{enumerate}
	\end{alertblock}	

\end{frame}
}

%%------------------------------------------------------------------------------------------------------

\subsection{} 

{\nologo 
\begin{frame}\frametitle{Forma canónica de Jordan}
	
	\begin{prop}{\textbf{Propiedad 3}}\justifying
		Sea $A$ una matriz $n\times n$ con entradas reales o complejas. Entonces existe una matriz invertible $C$ de 
		$n\times n$ con entradas complejas tal que
		\[
		C^{-1}AC = J,
		\]
		donde $J$ es una matriz de Jordan cuyos elementos en la diagonal son los valores propios de $A$. 
		Más aún, la matriz de Jordan $J$ es única, excepto por el orden en el que aparecen los bloques de Jordan.
	\end{prop}
	
	\vspace{5mm}
	\begin{block}{\textbf{Definición 2 (Forma canónica de Jordan)}}
	La matriz $J$ en la propiedad 3 se denomina la \textbf{\textit{forma canónica de Jordan}} de $A$.
	\end{block}
		
\end{frame}
}

%%------------------------------------------------------------------------------------------------------

\subsection{} 

{\nologo 
\begin{frame}\frametitle{Vector propio generalizado}
	
	\begin{prop}{\textbf{Propiedad 4}}\justifying
		Suponga que $A$ es una matriz de $2\times 2$ que tiene un valor propio $\lambda$ de multiplicidad 
		algebraica 2 y de multiplicidad geométrica 1. Si $\mathbf{v}_1$ un vector propio correspondiente a
		$\lambda$, entonces existe un vector $\mathbf{v}_2$ que satisface la ecuación
		\[
			(A-\lambda I) \mathbf{v}_2 = \mathbf{v}_1
		\]
	\end{prop}
	
	\vspace{0mm}
	\begin{block}{\textbf{Definición 3 (Vector propio generalizado)}}
		Sea $A$ es una matriz de $2\times 2$ con un solo valor propio $\lambda$ de multiplicidad 
		geométrica 1. Un vector $\mathbf{v}_2$ que satisfaga la ecuación
		\[
		(A-\lambda I) \mathbf{v}_2 = \mathbf{v}_1
		\]
		se denomina \textbf{\textit{vector propio generalizado}} de $A$.
	\end{block}
	
\end{frame}
}

%%------------------------------------------------------------------------------------------------------

\subsection{}

\begin{frame}\frametitle{Ejemplo de vector propio generalizado}

\begin{ej}{\textbf{Ejemplo 1}} \justifying
	Halle un vector propio generalizado de
	\[
	A = 
	\left(
	\begin{array}{cc}
	3 & -2 \\[1mm]
	8 & -5
	\end{array}
	\right)
	\]
\end{ej}

\textit{Solución.}

\vspace{2mm}
\begin{itemize}
	\item Polinomio característico de $A$:
	
	\vspace{2mm}	
	\[
		p(\lambda) = |P-\lambda I| =
		\left|	
		\begin{array}{@{\hspace{0.2\tabcolsep}}c@{\hspace{1.2\tabcolsep}}c@{\hspace{0.2\tabcolsep}}}
		3-\lambda & -2 \\[1mm]
		8 & -5-\lambda
		\end{array}
		\right| 
		=
		\lambda^2 +2\lambda + 1
	\]
	
	\vspace{5mm}	
	\item Valores propios de $A$:
	
	\vspace{0mm}
	\[
	p(\lambda)  = \lambda^2 +2\lambda + 1 = (\lambda+1)^2=0
	\quad \Longrightarrow \quad 
	\lambda_1=\lambda_2=-1.
	\]
	
\end{itemize}

\end{frame}

%%------------------------------------------------------------------------------------------------------

\subsection{}

\begin{frame}\frametitle{Ejemplo de vector propio generalizado}
	
	\begin{itemize}
		\item Espacio propio $E_{\lambda_1}=E_{-1}=N_{A+I}$:
		
		\[				
		(A+I \ | \ \mathbf{0})
		=
		\left(
		\begin{array}{@{\hspace{0.2\tabcolsep}}r@{\hspace{\tabcolsep}}r@{\hspace{\tabcolsep}}|r@{\hspace{0.2\tabcolsep}}}
		4 & -2 & 0  \\[1mm]
		8 & -4 & 0
		\end{array}
		\right) 
		\xrightarrow[]{\phantom{xx} }		
		\left(
		\begin{array}{@{\hspace{0.2\tabcolsep}}r@{\hspace{\tabcolsep}}r@{\hspace{\tabcolsep}}|r@{\hspace{0.2\tabcolsep}}}
		1 & -\frac{1}{2} & 0  \\[1mm]
		0 &    0 & 0
		\end{array}
		\right) 
		\quad \Rightarrow \quad 
		\mathbf{v}_1 = 
		\left(
		\begin{array}{@{\hspace{0.3\tabcolsep}}c@{\hspace{0.5\tabcolsep}}}
		1   \\[1mm]
		2 
		\end{array}
		\right) 
		\]
		
		\vspace{8mm}
		\item  Para el vector propio generalizado $\mathbf{v}_2$, resolvemos $(A-\lambda I)\mathbf{v}_2= \mathbf{v}_1$:
		
		\[				
		(A+I \ | \ \mathbf{v}_1)
		=
		\left(
		\begin{array}{@{\hspace{0.2\tabcolsep}}r@{\hspace{1.2\tabcolsep}}r@{\hspace{\tabcolsep}}|r@{\hspace{0.2\tabcolsep}}}
		4 & -2 & 1  \\[1mm]
		8 & -4 & 2
		\end{array}
		\right) 
		\xrightarrow[]{\phantom{xx} }		
		\left(
		\begin{array}{@{\hspace{0.2\tabcolsep}}r@{\hspace{1.2\tabcolsep}}r@{\hspace{\tabcolsep}}|r@{\hspace{0.2\tabcolsep}}}
		1 & -\frac{1}{2} & \frac{1}{4}\\[1mm]
		0 &    0 & 0
		\end{array}
		\right) 
		\quad \Rightarrow \quad 
		\begin{array}{@{\hspace{0.3\tabcolsep}}r@{\hspace{0.5\tabcolsep}}c@{\hspace{0.5\tabcolsep}}l@{\hspace{0.5\tabcolsep}}}
		x & = & \frac{1}{4} + \frac{1}{2}y   \\[1mm]
		y & = & y
		\end{array}		
		\]
		
		\vspace{3mm}
		\[
			\mathbf{v}_2 = 
			\left(
			\begin{array}{@{\hspace{0.3\tabcolsep}}c@{\hspace{0.5\tabcolsep}}}
			x   \\[1mm]
			y 
			\end{array}
			\right) 
			=
			\left(
			\begin{array}{@{\hspace{0.3\tabcolsep}}c@{\hspace{0.5\tabcolsep}}}
			\frac{1}{4}   \\[1mm]
			0 
			\end{array}
			\right) 
		\]
		
	\end{itemize}
	
\end{frame}

%------------------------------------------------------------------------------------------------ 

\subsection{} 

\begin{frame}\frametitle{Forma canónica de Jordan para matrices $2\times 2$}
	
	\begin{prop}{\textbf{Propiedad 5}}\justifying
		Suponga que $A$ es una matriz de $2\times 2$ que tiene un valor propio $\lambda$ de multiplicidad 
		algebraica 2 y de multiplicidad geométrica 1. Si $\mathbf{v}_1$ un vector propio correspondiente a
		$\lambda$ y $\mathbf{v}_2$ es un \textit{vector propio generalizado} de $A$, entonces
		\[
		C^{-1} A C = J,
		\qquad \text{donde} \qquad
		J =
		\left(
		\begin{array}{@{\hspace{0.2\tabcolsep}}c@{\hspace{1.2\tabcolsep}}c@{\hspace{0.2\tabcolsep}}}
		\lambda & 1 \\[1mm]
		0 & \lambda
		\end{array}
		\right)
		\]
		es la forma canónica de Jordan de $A$ y $C$ es la matriz cuyas columnas son los vectores $\mathbf{v}_1$
		y  $\mathbf{v}_2$.
	\end{prop}
		
\end{frame}

%%------------------------------------------------------------------------------------------------------

\subsection{}

\begin{frame}\frametitle{Forma canónica de Jordan para matrices $2\times 2$}
	
	\begin{ej}{\textbf{Ejemplo 2}} \justifying
		Halle la forma canónica de Jordan de
		\[
		A = 
		\left(
		\begin{array}{cc}
		3 & -2 \\[1mm]
		8 & -5
		\end{array}
		\right)
		\]
	\end{ej}
	
	\textit{Solución.}
	
	\vspace{2mm}
	\begin{itemize}
		\item Valores y vectores propios de $A$ obtenidos:
		
		\vspace{1mm}	
		\[
		\lambda = -1,
		\quad \
		\mathbf{v}_1 = 
		\left(
		\begin{array}{c}
		1 \\[1mm]
		2
		\end{array}
		\right),
		\quad \
		\mathbf{v}_2 = 
		\left(
		\begin{array}{c}
		\frac{1}{4} \\[1mm]
		0
		\end{array}
		\right),
		\quad \
		C = 
		\left(
		\begin{array}{cc}
		1 & \frac{1}{4} \\[1mm]
		2 & 0
		\end{array}
		\right).
		\]
								
		\vspace{3mm}	
		\item Forma canónica de Jordan de $A$:
		
		\vspace{0mm}	
		\[
		C^{-1}AC
		= 
		\left(
		\begin{array}{rr}
		-1 & 1 \\[1mm]
		 0 & -1
		\end{array}
		\right)
		= J
		\]
		
	\end{itemize}
	
\end{frame}

%------------------------------------------------------------------------------------------------ 

\subsection{} 

{\nologo 
\begin{frame}\frametitle{Forma canónica de Jordan para matrices $3\times 3$}
	
	\vspace{-2mm}	
	\begin{prop}{\textbf{Propiedad 6}}\justifying
		Suponga que $A$ es una matriz $3\times 3$ que tiene un valor propio $\lambda$ de multiplicidad 
		algebraica 3 y de multiplicidad geométrica 1. Si $\mathbf{v}_1$ un vector propio correspondiente a
		$\lambda$, entonces:
		\begin{enumerate}[$a$]
			\item Existe un vector $\mathbf{v}_2$ tal que 
			\[
				(A-\lambda I) \mathbf{v}_2 = \mathbf{v}_1,
			\]
			con $\mathbf{v}_1$ y $\mathbf{v}_2$ LI.
			\item Existe un vector $\mathbf{v}_3$ tal que 
			\[
			(A-\lambda I) \mathbf{v}_3 = \mathbf{v}_2,
			\]
			con $\mathbf{v}_1$, $\mathbf{v}_2$ y $\mathbf{v}_3$ LI.
			\item La matriz $C$ cuyas columnas son los vectores $\mathbf{v}_1$, $\mathbf{v}_2$ y $\mathbf{v}_3$ satisface
			\[
			C^{-1} A C = J =
			\left(
			\begin{array}{@{\hspace{0.2\tabcolsep}}c@{\hspace{1.2\tabcolsep}}c@{\hspace{1.2\tabcolsep}}c@{\hspace{0.2\tabcolsep}}}
				\lambda & 1  & 0 \\[1mm]
				0 & \lambda  & 1 \\[1mm]
				0 & 0  & \lambda
			\end{array}
			\right).
			\]
		\end{enumerate}
				
	\end{prop}
	
\end{frame}
}

%%------------------------------------------------------------------------------------------------------

\subsection{}

\begin{frame}\frametitle{Forma canónica de Jordan para matrices $3\times 3$}
	
	\begin{ej}{\textbf{Ejemplo 3}} \justifying
		Halle la forma canónica de Jordan de
		\[
		A = 
		\left(
		\begin{array}{cr@{\hspace{2\tabcolsep}}r}
			0 &  1 & 0 \\[1mm]
			0 &  0 & 1 \\[1mm]
			1 & -3 & 3
		\end{array}
		\right)
		\]
	\end{ej}
	
	\textit{Solución.}	
	
\end{frame}