\section{Dimensión de un espacio vectorial}

\subsection{}

{\nologo
\begin{frame}\frametitle{Dimensión de un espacio vectorial}

\begin{block}{\textbf{Definición 1 (Dimensión)}}
	\justifying
	\begin{enumerate}
		\item[\labelname{$a$}] Si un espacio vectorial $V$ tiene una base con $n$ vectores, entonces al número $n$ se le llama la
		\textbf{\textit{dimensión}} de $V$ y escribimos
		\[
		\dim V = n.
		\]
		\item[\labelname{$b$}] Si $V$ es el espacio vectorial que consiste solamente del vector cero ($V=\{\mathbf{0}\}$), definimos
		la \textbf{\textit{dimensión}} de $V$ como cero.
	\end{enumerate}		
\end{block}

%\vspace{5mm}

\begin{ej}{\textbf{Ejemplo 1 }} \justifying
	En cada uno de los siguientes ejemplos, la dimensión se determina simplemente contando el número de vectores en la 
	base canónica.
	
	\begin{enumerate}
		\item[\labelname{$a$}] $\dim \r^n = n$.
		
		\vspace{1mm}
		\item[\labelname{$b$}] $\dim P_n = n+1$.
		
		\vspace{1mm}
		\item[\labelname{$c$}] $\dim M_{mn} = m\times n$.
	\end{enumerate}
\end{ej}	

\end{frame}
}

%%------------------------------------------------------------------------------------------------------

\subsection{}

{\nologo
\begin{frame}\frametitle{Dimensión de un subespacio}

\begin{block}{\textbf{Definición 1 (Dimensión)}}
	\justifying
	\begin{enumerate}
		\item[\labelname{$a$}] Si un espacio vectorial $V$ tiene una base con $n$ vectores, entonces al número $n$ se le llama la
		\textbf{\textit{dimensión}} de $V$ y escribimos
		
		\vspace{-2mm}
		\[
		\dim V = n.
		\]
		
		\vspace{-1mm}
		\item[\labelname{$b$}] Si $V$ es el espacio vectorial que consiste solamente del vector cero ($V=\{\mathbf{0}\}$), definimos
		la \textbf{\textit{dimensión}} de $V$ como cero.
	\end{enumerate}		
\end{block}

%\vspace{5mm}

\begin{ej}{\textbf{Ejemplo 2 }} \justifying
	Determine la dimensión de cada uno de los siguientes subespacios de $\r^3$:

	\begin{enumerate}
		\item[\labelname{$a$}]  $W = \big\{(2c,c,0)\mid c \ \text{es un número real} \big\}$.
		
		\vspace{1mm}
		\item[\labelname{$b$}] $W = \big\{(b,a-b,a)\mid a \ \text{y} \ b \ \text{son números reales} \big\}$.
	\end{enumerate}
\end{ej}	

\end{frame}
}

%%------------------------------------------------------------------------------------------------------

\subsection{}

\begin{frame}\frametitle{Dimensión de un subespacio}

\begin{block}{\textbf{Definición 1 (Dimensión)}}
	\justifying
	\begin{enumerate}
		\item[\labelname{$a$}] Si un espacio vectorial $V$ tiene una base con $n$ vectores, entonces al número $n$ se le llama la
		\textbf{\textit{dimensión}} de $V$ y escribimos
		
		\vspace{-2mm}
		\[
		\dim V = n.
		\]
		
		\vspace{-1mm}
		\item[\labelname{$b$}] Si $V$ es el espacio vectorial que consiste solamente del vector cero ($V=\{\mathbf{0}\}$), definimos
		la \textbf{\textit{dimensión}} de $V$ como cero.
	\end{enumerate}		
\end{block}

%\vspace{5mm}

\begin{ej}{\textbf{Ejemplo 3 }} \justifying
	Sea $W$ el subespacio de todas las matrices simétricas en $M_{22}$. Halle la dimensión de $W$.
	
\end{ej}	

\end{frame}

%------------------------------------------------------------------------------------------------------

\subsection{}

\begin{frame}\frametitle{Conjuntos LI}
	
	\begin{prop}{\textbf{Propiedad 1}}\justifying
		Sea $V$ un espacio vectorial de dimensión $n$ y $\{\mathbf{v}_1, \mathbf{v}_2, \hdots , \mathbf{v}_m \}$ es un conjunto de vectores en $V$ linealmente independiente, entonces $m\leq n$.
	\end{prop}	
	
	
\end{frame}

%------------------------------------------------------------------------------------------------------

\subsection{}

\begin{frame}\frametitle{Dimensión de un subesapcio}
	
	\begin{prop}{\textbf{Propiedad 2}}\justifying
		Si $V$ un espacio vectorial de dimensión finita y $H$ es un subespacio vectorial de $V$, entonces $\dim H\leq \dim V$. 
	\end{prop}	
	
	%\vspace{5mm}
		
\end{frame}

%------------------------------------------------------------------------------------------------------

\subsection{}

\begin{frame}\frametitle{Comprobación de una base en un espacio $n$-dimensional}

\begin{prop}{\textbf{Propiedad 3}}\justifying
	Sea $V$ un espacio vectorial de dimensión $n$. 
	\begin{enumerate}
		\item[\labelname{$a$}] Si $S=\{\mathbf{v}_1, \mathbf{v}_2, \hdots , \mathbf{v}_n \}$ es un conjunto de vectores en $V$ 
		linealmente independiente (LI), entonces $S$ es una base para $V$.
		\item[\labelname{$b$}] Si $S=\{\mathbf{v}_1, \mathbf{v}_2, \hdots , \mathbf{v}_n \}$ genera a $V$,
		entonces $S$ es una base para $V$.
	\end{enumerate}
\end{prop}	

%\vspace{5mm}

\begin{alertblock}{\textbf{Observación 1}} \justifying
	Como consecuencia del teorema anterior, para verificar si un conjunto $S$ de $n$ vectores en un
	espacio vectorial $V$ de dimensión $n$ es una base para $V$, es suficiente con verificar que $S$ es
	linealmente independiente (LI) o que $S$ genera a $V$.
\end{alertblock}	

\end{frame}

%------------------------------------------------------------------------------------------------------

\subsection{}

{\nologo
\begin{frame}%\frametitle{Comprobación de una base en un espacio $n$-dimensional}

\begin{prop}{\textbf{Propiedad 3}}\justifying
	Sea $V$ un espacio vectorial de dimensión $n$. 
	\begin{enumerate}
		\item[\labelname{$a$}] Si $S=\{\mathbf{v}_1, \mathbf{v}_2, \hdots , \mathbf{v}_n \}$ es un conjunto de vectores en $V$ 
		linealmente independiente (LI), entonces $S$ es una base para $V$.
		\item[\labelname{$b$}] Si $S=\{\mathbf{v}_1, \mathbf{v}_2, \hdots , \mathbf{v}_n \}$ genera a $V$,
		entonces $S$ es una base para $V$.
	\end{enumerate}
\end{prop}	

%\vspace{5mm}

\begin{ej}{\textbf{Ejemplo 4}} \justifying
	Muestre que el conjunto de vectores de $\r^5$,
	{\small 
	\[	
     S = 
	 \left\{ \begin{array}{c@{\hspace{-\tabcolsep}}} \phantom{.}\\ \phantom{.}\\ \phantom{\vdots} \\[1mm] \phantom{.} \end{array} \right.
	 \underbrace{\left(\begin{array}{r} 1\\ 2\\ -1 \\ 3 \\ 4 \end{array} \right)}_{\color{blue}\mathbf{v}_1},\
	 \underbrace{\left(\begin{array}{r} 0\\ 1\\ 3 \\ -2 \\ 3 \end{array} \right)}_{\color{blue}\mathbf{v}_2},\
	 \underbrace{\left(\begin{array}{r} 0\\ 0\\ 2 \\ -1 \\ 5 \end{array} \right)}_{\color{blue}\mathbf{v}_3},\
	 \underbrace{\left(\begin{array}{r} 0\\ 0\\ 0 \\ 2 \\ -3 \end{array} \right)}_{\color{blue}\mathbf{v}_4},\
	 \underbrace{\left(\begin{array}{r} 0\\ 0\\ 0 \\ 0 \\ -2 \end{array} \right)}_{\color{blue}\mathbf{v}_5}
	 \left. \begin{array}{c@{\hspace{-\tabcolsep}}} \phantom{.}\\ \phantom{.}\\ \phantom{\vdots} \\[1mm] \phantom{.} \end{array} \right\}
	\]
	}
	es una base para $\r^5$.
\end{ej}	

\end{frame}
}

%------------------------------------------------------------------------------------------------------

\subsection{}

\begin{frame}\frametitle{Dimensión del espacio solución de un sistema homogéneo}

%\begin{prop}{\textbf{Propiedad 3}}\justifying
%	Sea $V$ un espacio vectorial de dimensión $n$. 
%	\begin{enumerate}
%		\item[\labelname{$a$}] Si $S=\{\mathbf{v}_1, \mathbf{v}_2, \hdots , \mathbf{v}_n \}$ es un conjunto de vectores en $V$ 
%		linealmente independiente (LI), entonces $S$ es una base para $V$.
%		\item[\labelname{$b$}] Si $S=\{\mathbf{v}_1, \mathbf{v}_2, \hdots , \mathbf{v}_n \}$ genera a $V$,
%		entonces $S$ es una base para $V$.
%	\end{enumerate}
%\end{prop}	

%\vspace{5mm}

\begin{ej}{\textbf{Ejemplo 5}} \justifying
	Halle la dimensión del espacio solución del sistema homogéneo
	\[	
	\begin{array}{r@{\hspace{0.8\tabcolsep}}c@{\hspace{0.8\tabcolsep}}r@{\hspace{0.8\tabcolsep}}c@{\hspace{0.8\tabcolsep}}r@{\hspace{0.8\tabcolsep}}c@{\hspace{0.8\tabcolsep}}l@{\hspace{0.8\tabcolsep}}} 
		x & + & 2y & - & z & = & 0 \\[1mm]  
		2x & - & y & + & 3z & = & 0 
	\end{array}
	\]
\end{ej}	

\end{frame}

%------------------------------------------------------------------------------------------------------

\subsection{}

\begin{frame}\frametitle{Dimensión del espacio solución de un sistema homogéneo}

%\begin{prop}{\textbf{Propiedad 3}}\justifying
%	Sea $V$ un espacio vectorial de dimensión $n$. 
%	\begin{enumerate}
%		\item[\labelname{$a$}] Si $S=\{\mathbf{v}_1, \mathbf{v}_2, \hdots , \mathbf{v}_n \}$ es un conjunto de vectores en $V$ 
%		linealmente independiente (LI), entonces $S$ es una base para $V$.
%		\item[\labelname{$b$}] Si $S=\{\mathbf{v}_1, \mathbf{v}_2, \hdots , \mathbf{v}_n \}$ genera a $V$,
%		entonces $S$ es una base para $V$.
%	\end{enumerate}
%\end{prop}	

%\vspace{5mm}

\begin{ej}{\textbf{Ejemplo 6}} \justifying
	Halle la dimensión del espacio solución del sistema homogéneo
	\[	
	\begin{array}{r@{\hspace{0.8\tabcolsep}}c@{\hspace{0.8\tabcolsep}}r@{\hspace{0.8\tabcolsep}}c@{\hspace{0.8\tabcolsep}}r@{\hspace{0.8\tabcolsep}}c@{\hspace{0.8\tabcolsep}}l@{\hspace{0.8\tabcolsep}}} 
	2x & - & y & + & 3z & = & 0 \\[1mm] 
	4x & - & 2y & + & 6z & = & 0 \\[1mm] 
	-6x & + & 3y & - & 9z & = & 0 
	\end{array}
	\]
\end{ej}	

\end{frame}

