\section{Diagonalización}

\subsection{}

{\nologo 
\begin{frame}\frametitle{Recordemos que\ldots }
	
	\begin{prop}{\textbf{Propiedad 1 }}
		\justifying
		Sea $V$ un un espacio vectorial de dimensión finita con bases $\mathcal{B}$ y $\mathcal{B}'$ y sea $T:V\longrightarrow V$ una transformación lineal. Si $A_T$ la matriz de $T$ con respecto la base $\mathcal{B}$ y ${A'}_T$ la matriz de $T$ con respecto a la base $\mathcal{B}'$, entonces
		\[
		{A'}_T=P^{-1}A_TP,
		\]
		donde  $P=P_{\mathcal{B} \leftarrow\mathcal{B}'}$ es la matriz de cambio de base de $\mathcal{B}'$ a $\mathcal{B}$.
	\end{prop}	
	
	\begin{alertblock}{\textbf{Observación 1 }}
		\justifying
		Por la propiedad 1, las matrices que representan a una transformación lineal 
		\[
			T:V\longrightarrow V
		\]
		con respecto a bases distintas de $V$, son semejantes.
	\end{alertblock}
	
\end{frame}
}

% ---------------------------------------------------------------------------------------------------

\subsection{}
%
\begin{frame}\frametitle{Matriz semejante a una matriz diagonal}
	
	\begin{ej}{\textbf{Ejemplo 1}}
		Considere la transformación lineal $T:\r^2\longrightarrow\r^2$ dada por
		\[
		T\left(
		\begin{array}{@{\hspace{0.3\tabcolsep}}c@{\hspace{0.3\tabcolsep}}}
		x  \\[1mm]
		y  
		\end{array}
		\right)
		=
		\left(
		\begin{array}{@{\hspace{0.3\tabcolsep}}c@{\hspace{0.3\tabcolsep}}}
		3x+2y  \\[1mm]
		3x+4y 
		\end{array}
		\right)
		\]
		y las bases de $\r^2$, $\mathcal{B}=\{(1,0), (0,1)\}$ y $\mathcal{B}'=\{(1,-1), (2,3)\}$.
		%	\[
		%		\mathcal{B}_1=\{(1,0), (0,1)\} \quad \text{y} \quad \mathcal{B}_2=\{(1,0), (1,1)\}.
		%	\]
		
		%\vspace{-2mm}
		\begin{enumerate}[$a$]
			\justifying
			\item Encuentre la matriz de representación $A_T$ respecto a la base  $\mathcal{B}$.
			\item Encuentre la matriz de representación ${A'}_T$ respecto a la base  $\mathcal{B}'$.
			\item ¿Qué relación existe entre $A_T$ y ${A'}_T$?
			%\item Halle la matriz de cambio de base $P=P_{\mathcal{B} \leftarrow\mathcal{B}'}$.
			%\item Considere la base $\mathcal{B}_2=\{(1,0), (1,1)\}$ de $\r^2$. Halle la matriz $P$ de cambio de base de $\mathcal{B}_2$ a $\mathcal{B}_1$. 
			%\item Encuentre $P^{-1}$.
			%\item Use la propiedad 3 para hallar ${A'}_T$ respecto a la base $\mathcal{B}'$.
		\end{enumerate}
	\end{ej}
	\textit{Solución.}
	
\end{frame}

% ---------------------------------------------------------------------------------------------------

\subsection{}

\begin{frame}\frametitle{Valores propios de matrices semejantes}

\begin{defi}{\textbf{Definición 1}}\justifying
	\justifying
	Una matriz $A$ de $n\times n$ se dice que \textbf{\textit{diagonalizable}} si es semejante a una matriz
	diagonal $D$. Es decir, existe una matriz invertible $P$  de $n\times n$  tal que 
	\[
		P^{-1}AP = D.
	\]
\end{defi}	

\begin{prop}{\textbf{Propiedad 2}}
	\justifying
	Si $A$ y $B$ son matrices semejantes, entonces tienen los mismos valores propios.
\end{prop}	

\end{frame}

%%------------------------------------------------------------------------------------------------------

\subsection{}

\begin{frame}\frametitle{Valores propios de matrices semejantes}
	
	\begin{prop}{\textbf{Propiedad 2}}
		\justifying
		Si $A$ y $B$ son matrices semejantes, entonces tienen los mismos valores propios.
	\end{prop}	
	
	\begin{ej}{\textbf{Ejemplo 2}}
		Las matrices $A$ y $D$ son semejantes. Halle los valores propios de $A$.
		\[
		A =
		\left(
		\begin{array}{rrr}
		1  &  0 & \phantom{-}0 \\[1mm]
		-1 &  1 & 1 \\[1mm]
		-1 & -2 & 4
		\end{array}
		\right),\quad
		D =
		\left(
		\begin{array}{rrr}
		1 & 0 & 0 \\[1mm]
		0 & 2 & 0 \\[1mm]
		0 & 0 & 3
		\end{array}
		\right).
		\]	
	\end{ej}
	\textit{Solución.}
	
\end{frame}

%%------------------------------------------------------------------------------------------------------

\subsection{}

\begin{frame}\frametitle{Condiciones para que una matriz sea diagonalizable}
	
	\begin{prop}{\textbf{Propiedad 3}}
		\justifying
		Una matriz $A$ de $n\times n$ es diagonalizable si y sólo si tiene $n$ vectores propios 
		linealmente independientes.
	\end{prop}	
	
	\begin{alertblock}{\textbf{Observación 2}}\justifying
		Cuando $A$ es diagonalizable, $P^{-1}AP=D$, con $D$ una matriz diagonal cuyos elementos  
		de la diagonal son los valores propios de $A$ y $P$ una matriz cuyas columnas son 
		respectivamente los $n$ vectores propios LI de A.
	\end{alertblock}
	
\end{frame}

%%------------------------------------------------------------------------------------------------------

\subsection{}

\begin{frame}\frametitle{Ejemplo de una matriz diagonalizable}
			
	\begin{ej}{\textbf{Ejemplo 3}}
		Si es posible, encuentre una matriz $P$ que diagonalice a 
		\[
		A =
		\left(
		\begin{array}{rrr}
		-1 & \phantom{-}0 & 1 \\[1mm]
		 3 &            0 & -3 \\[1mm]
		 1 &            0 & -1
		\end{array}
		\right).
		\]	
	\end{ej}
	\textit{Solución.}
	
\end{frame}

%%------------------------------------------------------------------------------------------------------

\subsection{}

\begin{frame}\frametitle{Procedimiento para diagonalizar una matriz cuadrada}
	
	\begin{ejem}{\textbf{Procedimiento}}\justifying
		\justifying
		Sea $A$ una matriz $n\times n$. 
		\begin{enumerate}\justifying
			\item Halle $n$ vectores propios linealmente independientes $\mathbf{p}_1,\hdots,\mathbf{p}_n$ de $A$,
			con valores propios correspondientes $\lambda_1,\hdots,\lambda_n$. Si no existen $n$ vectores 
			propios linealmente independientes, entonces $A$ \textit{no} es diagonalizable.
			\item Si $A$ tiene $n$ vectores propios linealmente $\mathbf{p}_1,\mathbf{p}_2,\hdots,\mathbf{p}_n$,
			entonces defina
			\[
				P = (\ \mathbf{p}_1 \ | \ \mathbf{p}_2 \ | \ \cdots \ | \ \mathbf{p}_n \ ).
			\]
			\item La matriz diagonal $D=P^{-1}AP$ tendrá los valores propios $\lambda_1,\hdots,\lambda_n$
			en su diagonal principal. El orden de los vectores propios usados para formar a $P$, determina
			el orden en que aparecen los valores propios en la diagonal de $D$.
		\end{enumerate}
	\end{ejem}		
	
\end{frame}

%%------------------------------------------------------------------------------------------------------

\subsection{}

\begin{frame}\frametitle{Ejemplo de una matriz no diagonalizable}
	
	\begin{ej}{\textbf{Ejemplo 4}}
		Determine si la matriz $A$ es diagonalizable.
		\[
		A =
		\left(
		\begin{array}{rr}
		1 & 2 \\[1mm]
		0 & 1
		\end{array}
		\right)
		\]	
	\end{ej}
	\textit{Solución.}
	
\end{frame}

%%------------------------------------------------------------------------------------------------------

\subsection{}

\begin{frame}\frametitle{Ejemplo de una matriz diagonalizable}
	
	\begin{ej}{\textbf{Ejemplo 5}}
		Si es posible, encuentre una matriz $P$ que diagonalice a 
		\[
		A =
		\left(
		\begin{array}{rrr}
		 1 & -1 & -1 \\[1mm]
		 1 &  3 &  1 \\[1mm]
		-3 & 1 & -1
		\end{array}
		\right).
		\]	
	\end{ej}
	\textit{Solución.}
	
\end{frame}

%%------------------------------------------------------------------------------------------------------

\subsection{}

%{\nologo 
\begin{frame}\frametitle{Condiciones para que una matriz sea diagonalizable}
		
%	\begin{prop}{\textbf{Propiedad 4}}
%		\justifying
%		Si $A$ es una matriz $n\times n$ que tiene $n$ valores propios distintos, entonces los respectivos vectores 
%		propios son linealmente independientes.
%	\end{prop}	
		
	\begin{prop}{\textbf{Propiedad 4}}
		\justifying
		Si $A$ es una matriz $n\times n$ que tiene $n$ valores propios distintos, entonces $A$ es diagonalizable.
	\end{prop}		
	%\textit{Demostración.}
	
	\vspace{0cm}
	
	\begin{ej}{\textbf{Ejemplo 6}}
		Determine si la matriz $A$ es diagonalizable.
		\[
		A =
		\left(
		\begin{array}{rrr}
		1 & -2 &  1 \\[1mm]
		0 &  0 &  1 \\[1mm]
		0 &  0 & -3
		\end{array}
		\right)
		\]	
	\end{ej}
	\textit{Solución.}
	
\end{frame}
%}

%%------------------------------------------------------------------------------------------------------

\subsection{}

\begin{frame}\frametitle{Potencia de una matriz}
	
	\begin{ej}{\textbf{Ejemplo 7}}
		Calcule $A^{10}$ si
		\[
		A =
		\left(
		\begin{array}{cc}
			0 & 1 \\[1mm]
			2 & 1 
		\end{array}
		\right).
		\]	
	\end{ej}
	\textit{Solución.}
	
\end{frame}

%%------------------------------------------------------------------------------------------------------

\subsection{}

\begin{frame}\frametitle{Condiciones para que una matriz sea diagonalizable}
	
	\begin{prop}{\textbf{Propiedad 5}}
		\justifying
		Sea $A$ una matriz $n\times n$, con $n$ vectores propios $\lambda_1,\hdots,\lambda_n$.
		Los siguientes enunciados son equivalentes:
		\begin{enumerate}[$a$]\justifying
			\item $A$ es diagonalizable.
			\item La multiplicidad algebraica de $\lambda_i$ es igual a su multiplicidad geométrica de $\lambda_i$,
			para $i=1,\hdots,m$, donde $m$ es el número de raíces distintas del polinomio característico $p(\lambda)$.
		\end{enumerate}
	\end{prop}	
	
	
	
\end{frame}

% ---------------------------------------------------------------------------------------------------

\subsection{}
%
\begin{frame}\frametitle{Diagonalización de un operador}
	
	\begin{ej}{\textbf{Ejemplo 8}}\justifying
		Considere la transformación lineal $T:\r^3\longrightarrow\r^3$ dada por
		\[
			T(x,y,z) = (x-y-z,x+3y+z,-3x+y-z).
		\]
		Si es posible, encuentre una base $\mathcal{B}$ para $\r^3$ tal que la matriz de representación
		de $T$ relativa a $\mathcal{B}$ sea diagonal.
	\end{ej}
	\textit{Solución.}
	
\end{frame}
