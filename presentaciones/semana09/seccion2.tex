\section{Matrices semejantes}

\subsection{}

{\nologo
\begin{frame}\frametitle{Vimos que\ldots }

\vspace{-2mm}
\begin{defi}{\textbf{Definición 1}}\justifying
	Sean $\mathcal{B}=\{\mathbf{u}_1, \mathbf{u}_2, \hdots , \mathbf{u}_n \}$ y $\mathcal{C}=\{\mathbf{v}_1, \mathbf{v}_2, \hdots , \mathbf{v}_n \}$  bases para un espacio  vectorial $V$. A la matriz $n\times n$ cuyas columnas son los vectores coordenados
	
	\vspace{-2mm}
	\[
	\left[ \mathbf{u}_1 \right]_{\mathcal{C}}, \left[ \mathbf{u}_2 \right]_{\mathcal{C}}, \hdots, \left[ \mathbf{u}_n \right]_{\mathcal{C}}
	\]
	se le llama \textbf{\textit{matriz de cambio de base}} de $\mathcal{B}$ a $\mathcal{C}$ y se denota por $P_{\mathcal{C} \leftarrow\mathcal{B}}$. Es decir,	
	
	\vspace{-5mm}
	\[	
	P_{\mathcal{C} \leftarrow\mathcal{B}} =
	\left( 
	\begin{array}{c|c|c|c} \left[ \mathbf{u}_1 \right]_{\mathcal{C}} & \left[ \mathbf{u}_2 \right]_{\mathcal{C}} & \cdots & \left[ \mathbf{u}_n \right]_{\mathcal{C}}
	\end{array} 
	\right)
	\]
\end{defi}	

\vspace{-1mm}

\begin{alertblock}{\textbf{Observación 1 }}\justifying 
Si $\mathcal{B}$ y $\mathcal{C}$ son dos bases de $V$, las columnas de la matriz de cambio de base $P_{\mathcal{C} \leftarrow\mathcal{B}}$ son precisamente los vectores 
coordenados obtenidos al escribir los vectores de la base $\mathcal{B}$ en términos de los de la base $\mathcal{C}$.
\end{alertblock}

\vspace{-1mm}

\begin{prop}{\textbf{Propiedad 1}}
	\justifying
	Si $\mathcal{B}=\{\mathbf{u}_1, \mathbf{u}_2, \hdots , \mathbf{u}_n \}$ y $\mathcal{C}=\{\mathbf{v}_1, \mathbf{v}_2, \hdots , \mathbf{v}_n \}$  bases para un espacio  vectorial $V$ y sea $P_{\mathcal{C} \leftarrow\mathcal{B}}$ la matriz de cambio 
	de base de $\mathcal{B}$ a $\mathcal{C}$, entonces
	
	\vspace{-2mm}
	\[	
		P_{\mathcal{C} \leftarrow\mathcal{B}}\left[ \mathbf{x} \right]_{\mathcal{B}}=\left[ \mathbf{x} \right]_{\mathcal{C}},
		\quad \text{para todo }  \mathbf{x} \ \text{en } V.
	\]	
\end{prop}	

\end{frame}
}

% ---------------------------------------------------------------------------------------------------

\subsection{}

{\nologo
\begin{frame}\frametitle{Vimos que\ldots }
	
	\vspace{-3mm}
	\begin{defi}{\textbf{Definición 1}}\justifying
		Sean $\mathcal{B}=\{\mathbf{u}_1, \mathbf{u}_2, \hdots , \mathbf{u}_n \}$ y $\mathcal{C}=\{\mathbf{v}_1, \mathbf{v}_2, \hdots , \mathbf{v}_n \}$  bases para un espacio  vectorial $V$. A la matriz $n\times n$ cuyas columnas son los vectores coordenados
		\[
		\left[ \mathbf{u}_1 \right]_{\mathcal{C}}, \left[ \mathbf{u}_2 \right]_{\mathcal{C}}, \hdots, \left[ \mathbf{u}_n \right]_{\mathcal{C}}
		\]
		se le llama \textbf{\textit{matriz de cambio de base}} de $\mathcal{B}$ a $\mathcal{C}$ y se denota por $P_{\mathcal{C} \leftarrow\mathcal{B}}$. Es decir,	
		\[	
		P_{\mathcal{C} \leftarrow\mathcal{B}} =
		\left( 
		\begin{array}{c|c|c|c} \left[ \mathbf{u}_1 \right]_{\mathcal{C}} & \left[ \mathbf{u}_2 \right]_{\mathcal{C}} & \cdots & \left[ \mathbf{u}_n \right]_{\mathcal{C}}
		\end{array} 
		\right)
		\]
	\end{defi}	
	
	\vspace{-1mm}
	\begin{prop}{\textbf{Propiedad 2}}
		\justifying
		Sean $\mathcal{B}=\{\mathbf{u}_1, \mathbf{u}_2, \hdots , \mathbf{u}_n \}$ y $\mathcal{C}=\{\mathbf{v}_1, \mathbf{v}_2, \hdots , \mathbf{v}_n \}$  bases para un espacio  vectorial $V$. Sean
		\[
		B = \left( 
		\begin{array}{c|c|c} \left[ \mathbf{u}_1 \right]_{\mathcal{E}}  & \cdots & \left[ \mathbf{u}_n \right]_{\mathcal{E}}
		\end{array} 
		\right)
		\qquad \text{y} \qquad 
		C = \left( 
		\begin{array}{c|c|c} \left[ \mathbf{v}_1 \right]_{\mathcal{E}}  & \cdots & \left[ \mathbf{v}_n \right]_{\mathcal{E}}
		\end{array} 
		\right),
		\]
		donde $\mathcal{E}$ es cualquier base para $V$. Entonces al aplicar eliminación de Gauss-Jordan a la matriz aumentada $(\ C\ |\ B\ )$, se obtiene
		\[	
		\left( 
		\begin{array}{c|c} C & B
		\end{array} 
		\right)
		\longrightarrow
		\left( 
		\begin{array}{c|c} I & P_{\mathcal{C} \leftarrow\mathcal{B}}
		\end{array} 
		\right).
		\]
	\end{prop}	
	
\end{frame}
}

% ---------------------------------------------------------------------------------------------------

\subsection{}

\begin{frame}%\frametitle{Cambio de base}
	
%	\begin{prop}{\textbf{Propiedad 2}}
%		\justifying
%		Sean $\mathcal{B}=\{\mathbf{u}_1, \mathbf{u}_2, \hdots , \mathbf{u}_n \}$ y $\mathcal{C}=\{\mathbf{v}_1, \mathbf{v}_2, \hdots , \mathbf{v}_n \}$  bases para un espacio  vectorial $V$. Sean
%		\[
%		B = \left( 
%		\begin{array}{c|c|c} \left[ \mathbf{u}_1 \right]_{\mathcal{E}}  & \cdots & \left[ \mathbf{u}_n \right]_{\mathcal{E}}
%		\end{array} 
%		\right)
%		\qquad \text{y} \qquad 
%		C = \left( 
%		\begin{array}{c|c|c} \left[ \mathbf{v}_1 \right]_{\mathcal{E}}  & \cdots & \left[ \mathbf{v}_n \right]_{\mathcal{E}}
%		\end{array} 
%		\right),
%		\]
%		donde $\mathcal{E}$ es cualquier base para $V$. Entonces al aplicar eliminación de Gauss-Jordan a la matriz aumentada $(\ C\ |\ B\ )$, se obtiene
%		\[	
%		\left( 
%		\begin{array}{c|c} C & B
%		\end{array} 
%		\right)
%		\longrightarrow
%		\left( 
%		\begin{array}{c|c} I & P_{\mathcal{C} \leftarrow\mathcal{B}}
%		\end{array} 
%		\right).
%		\]
%	\end{prop}	
	
	\begin{ej}{\textbf{Ejemplo 1}}
		\justifying
		Sea $T:\r^2\to \r^2$ la transformación lineal $T(x,y)=(x+3y,2x+2y)$ y considere en $\r^2$ las bases
		$\mathcal{B} = \{(1,0), (0,1)\}$ y $\mathcal{B}' = \{(1,1), (3,-2)\}$. Halle:
		
		%\vspace{-3mm}
		%\begin{multicols}{2}
			\begin{itemize}
				\item[\labelname{$a$}] La matriz de cambio de base $P_{\mathcal{B}' \leftarrow\mathcal{B}}$.
				\item[\labelname{$b$}] $A_T$ respecto a la base $\mathcal{B}$. 
				\item[\labelname{$c$}] $A'_T$ respecto a la base $\mathcal{B}'$.
			\end{itemize}
		%\end{multicols}
	\end{ej}
	\textit{Solución.}
	
\end{frame}

% ---------------------------------------------------------------------------------------------------

\subsection{}

\begin{frame}\frametitle{Problema de diagonalización }
	
	\begin{alertblock}{\textbf{Observación 2 }}\justifying 
		Sea $V$ es un espacio vectorial de dimensión finita con base $\mathcal{B}$ y sea
		\[
			T:V\to V
		\]
		es una transformación lineal, con matriz $A_T$ respecto a la base $\mathcal{B}$. ¿Cómo hallar una base
		$\mathcal{B}'$ de $V$ para la cual ${A'}_T$ sea una matriz diagonal?		
	\end{alertblock}
	
	
\end{frame}

%% ---------------------------------------------------------------------------------------------------
%
%
%\subsection{}
%
%\begin{frame}%\frametitle{Cambio de base}
%	
%	\begin{defi}{\textbf{Definición 1}}\justifying
%		Sean $\mathcal{B}=\{\mathbf{u}_1, \mathbf{u}_2, \hdots , \mathbf{u}_n \}$ y $\mathcal{C}=\{\mathbf{v}_1, \mathbf{v}_2, \hdots , \mathbf{v}_n \}$  bases para un espacio  vectorial $V$. A la matriz $n\times n$ cuyas columnas son los vectores coordenados
%		\[
%		\left[ \mathbf{u}_1 \right]_{\mathcal{C}}, \left[ \mathbf{u}_2 \right]_{\mathcal{C}}, \hdots, \left[ \mathbf{u}_n \right]_{\mathcal{C}}
%		\]
%		se le llama \textbf{\textit{matriz de cambio de base}} de $\mathcal{B}$ a $\mathcal{C}$ y se denota por $P_{\mathcal{C} \leftarrow\mathcal{B}}$. Es decir,	
%		\[	
%		P_{\mathcal{C} \leftarrow\mathcal{B}} =
%		\left( 
%		\begin{array}{c|c|c|c} \left[ \mathbf{u}_1 \right]_{\mathcal{C}} & \left[ \mathbf{u}_2 \right]_{\mathcal{C}} & \cdots & \left[ \mathbf{u}_n \right]_{\mathcal{C}}
%		\end{array} 
%		\right)
%		\]
%	\end{defi}	
%	
%	\begin{ej}{\textbf{Ejemplo 1}}
%		\justifying
%		Encuentre la matriz de cambio de base de $\mathcal{B}$ a $\mathcal{C}$ para las siguientes bases de $\r^2$: 
%		\[
%			\mathcal{B} = \{(-3,2), (4,-2)\} \qquad \text{y} \qquad 
%			\mathcal{C} = \{(-1,2), (2,-2)\}.
%		\]
%	\end{ej}
%	\textit{Solución.}
%	
%\end{frame}

% ---------------------------------------------------------------------------------------------------

\subsection{}

{\nologo
\begin{frame}%\frametitle{Cambio de base}
	
	\vspace{-2mm}
	\begin{prop}{\textbf{Propiedad 3 }}
		\justifying
		Sea $V$ un un espacio vectorial de dimensión finita con bases $\mathcal{B}$ y $\mathcal{B}'$ y sea $T:V\longrightarrow V$ una transformación lineal. Si $A_T$ la matriz de $T$ con respecto la base $\mathcal{B}$ y ${A'}_T$ la matriz de $T$ con respecto a la base $\mathcal{B}'$, entonces
		\[
			{A'}_T=P^{-1}A_TP,
		\]
		donde  $P=P_{\mathcal{B} \leftarrow\mathcal{B}'}$ es la matriz de cambio de base de $\mathcal{B}'$ a $\mathcal{B}$.
	\end{prop}	
	
	\vspace{-1mm}
	\begin{alertblock}{\textbf{Observación 3 }}\justifying 
		
		\begin{center}		
		\begin{tikzpicture}[node distance=1.6cm, auto]
			\node at (0,0.5) {$V$};			
			\node at (3,0.5) {$V$};
			\node (A) {$\mathbf{x}$};
			\node (B) [ right=2.3cm  of A] {$T(\mathbf{x})$};
			\draw[->](A) to node {$T$}(B);
			%\pause \pause 
			\node (C) [below of=A] {$[\mathbf{x}]_{\mathcal{B}'}$};
			\draw[->](A) to node [left] {}(C);
			%\pause \pause 
			\node (D) [below of=B] {$[T(\mathbf{x})]_{\mathcal{B}'}$};
			\draw[->](B) to node [left] {}(D);
			%\pause \pause 
			\draw[->](C) to node {${A'}_T$}(D);						
			%\pause \pause 
			\node at (4.4,-1.6) {$= A'_T[\mathbf{x}]_{\mathcal{B}'}$};
			%\pause \pause 
			\node (E) [below of=C] {$P[\mathbf{x}]_{\mathcal{B}'}$};
			\draw[->](C) to node [left] {$P$}(E);
			%\pause \pause 
			\node (F) [below of=D] {$A_TP[\mathbf{x}]_{\mathcal{B}'}$};
			\draw[->](E) to node [above] {$A_T$}(F);
			%\pause \pause 
			\draw[->](F) to node [right] {$P^{-1}$}(D);
			%\pause \pause 
			\node at (6.4,-1.6) {$=P^{-1}A_TP[\mathbf{x}]_{\mathcal{B}'}$};
		\end{tikzpicture}
		\end{center}
	
	\end{alertblock}

\end{frame}
}

% ---------------------------------------------------------------------------------------------------

\subsection{}
%
\begin{frame}%\frametitle{Cambio de base}

\begin{ej}{\textbf{Ejemplo 2}}
	Considere la transformación lineal $T:\r^2\longrightarrow\r^2$ dada por
	\[
	T\left(
	\begin{array}{@{\hspace{0.3\tabcolsep}}c@{\hspace{0.3\tabcolsep}}}
	x  \\[1mm]
	y  
	\end{array}
	\right)
	=
	\left(
	\begin{array}{@{\hspace{0.3\tabcolsep}}c@{\hspace{0.3\tabcolsep}}}
	2x-2y  \\[1mm]
	-x+3y 
	\end{array}
	\right)
	\]
	y las bases de $\r^2$, $\mathcal{B}=\{(1,0), (0,1)\}$ y $\mathcal{B}'=\{(1,0), (1,1)\}$.
%	\[
%		\mathcal{B}_1=\{(1,0), (0,1)\} \quad \text{y} \quad \mathcal{B}_2=\{(1,0), (1,1)\}.
%	\]
	
	%\vspace{-2mm}
	\begin{enumerate}[$a$]
		\justifying
		\item Encuentre la matriz de representación $A_T$ respecto a la base  $\mathcal{B}$.
		\item Halle la matriz de cambio de base $P=P_{\mathcal{B} \leftarrow\mathcal{B}'}$.
		%\item Considere la base $\mathcal{B}_2=\{(1,0), (1,1)\}$ de $\r^2$. Halle la matriz $P$ de cambio de base de $\mathcal{B}_2$ a $\mathcal{B}_1$. 
		%\item Encuentre $P^{-1}$.
		\item Use la propiedad 3 para hallar ${A'}_T$ respecto a la base $\mathcal{B}'$.
	\end{enumerate}
\end{ej}
\textit{Solución.}

\end{frame}

% ---------------------------------------------------------------------------------------------------

\subsection{}
%
\begin{frame}%\frametitle{Cambio de base}

\begin{ej}{\textbf{Ejemplo $3(a)$ }}
	Sea $T:\r^2\longrightarrow\r^2$ un transformación lineal y considere en $\r^2$ las bases
	\[
	\mathcal{B}=\{(-3,2), (4,-2)\} \qquad \text{y} \qquad \mathcal{B}'=\{(-1,2), (2,-2)\}. 
 	\]
 	Suponga que la matriz de representación $A_T$ respecto a la base 
 	$\mathcal{B}$ es 
 	\[
 	A_T = 
 	\left( 
 	\begin{array}{@{\hspace{1\tabcolsep}}rr}	
 	-2 & 7 \\[2mm] 
 	-3 & 7
 	\end{array} 
 	\right).
 	\]
 	
 	\vspace{-2mm}
	\begin{enumerate}[$a$]
		\justifying
		\item Encuentre la matriz de representación ${A'}_T$ respecto a la base $\mathcal{B}'$.
%		\item Calcule $\left[\mathbf{v}\right]_{\mathcal{B}}$, $\left[T(\mathbf{v})\right]_{\mathcal{B}}$ y $\left[T(\mathbf{v})\right]_{\mathcal{B}'}$, para el vector $\mathbf{v}$ cuyo vector de coordenadas es 
%		\[
%		[\mathbf{v}]_{\mathcal{B}'}=\left(
%		\begin{array}{@{\hspace{0.3\tabcolsep}}c@{\hspace{0.3\tabcolsep}}}
%		-3  \\[1mm]
%		-1  
%		\end{array}
%		\right)
%		\]
%		\item Encuentre una fórmula para $T\left(
%		\begin{array}{@{\hspace{0.3\tabcolsep}}c@{\hspace{0.3\tabcolsep}}}
%		x  \\[1mm]
%		y  
%		\end{array}
%		\right)$.
	\end{enumerate}
\end{ej}
\textit{Solución.}

\end{frame}

% ---------------------------------------------------------------------------------------------------

\subsection{}
%
\begin{frame}%\frametitle{Cambio de base}
	
	\begin{ej}{\textbf{Ejemplo $3(b)$ }}
		Sea $T:\r^2\longrightarrow\r^2$ un transformación lineal y considere en $\r^2$ las bases
		\[
		\mathcal{B}=\{(-3,2), (4,-2)\} \qquad \text{y} \qquad \mathcal{B}'=\{(-1,2), (2,-2)\}. 
		\]
		Suponga que la matriz de representación $A_T$ respecto a la base 
		$\mathcal{B}$ es 
		\[
		A_T = 
		\left( 
		\begin{array}{@{\hspace{1\tabcolsep}}rr}	
		-2 & 7 \\[2mm] 
		-3 & 7
		\end{array} 
		\right).
		\]
		
		\vspace{-2mm}
		\begin{enumerate}[$b$]
			\justifying
%			\item Encuentre la matriz de representación ${A'}_T$ respecto a la base $\mathcal{B}'$.
			\item Calcule $\left[\mathbf{v}\right]_{\mathcal{B}}$, $\left[T(\mathbf{v})\right]_{\mathcal{B}}$ y $\left[T(\mathbf{v})\right]_{\mathcal{B}'}$, para el vector $\mathbf{v}$ cuyo vector de coordenadas es 
			\[
			[\mathbf{v}]_{\mathcal{B}'}=\left(
			\begin{array}{@{\hspace{0.3\tabcolsep}}c@{\hspace{0.3\tabcolsep}}}
			-3  \\[1mm]
			-1  
			\end{array}
			\right)
			\]
%			\item Encuentre una fórmula para $T\left(
%			\begin{array}{@{\hspace{0.3\tabcolsep}}c@{\hspace{0.3\tabcolsep}}}
%			x  \\[1mm]
%			y  
%			\end{array}
%			\right)$.
		\end{enumerate}
	\end{ej}
	\textit{Solución.}
	
\end{frame}

% ---------------------------------------------------------------------------------------------------

\subsection{}
%
\begin{frame}%\frametitle{Cambio de base}
	
	\begin{ej}{\textbf{Ejemplo $3(c)$ }}
		Sea $T:\r^2\longrightarrow\r^2$ un transformación lineal y considere en $\r^2$ las bases
		\[
		\mathcal{B}=\{(-3,2), (4,-2)\} \qquad \text{y} \qquad \mathcal{B}'=\{(-1,2), (2,-2)\}. 
		\]
		Suponga que la matriz de representación $A_T$ respecto a la base 
		$\mathcal{B}$ es 
		\[
		A_T = 
		\left( 
		\begin{array}{@{\hspace{1\tabcolsep}}rr}	
		-2 & 7 \\[2mm] 
		-3 & 7
		\end{array} 
		\right).
		\]
		
		\vspace{-2mm}
		\begin{enumerate}[$c$]
			\justifying
%			\item Encuentre la matriz de representación ${A'}_T$ respecto a la base $\mathcal{B}'$.
%			\item Calcule $\left[\mathbf{v}\right]_{\mathcal{B}}$, $\left[T(\mathbf{v})\right]_{\mathcal{B}}$ y $\left[T(\mathbf{v})\right]_{\mathcal{B}'}$, para el vector $\mathbf{v}$ cuyo vector de coordenadas es 
%			\[
%			[\mathbf{v}]_{\mathcal{B}'}=\left(
%			\begin{array}{@{\hspace{0.3\tabcolsep}}c@{\hspace{0.3\tabcolsep}}}
%			-3  \\[1mm]
%			-1  
%			\end{array}
%			\right)
%			\]
			\item Encuentre una fórmula para $T\left(
			\begin{array}{@{\hspace{0.3\tabcolsep}}c@{\hspace{0.3\tabcolsep}}}
			x  \\[1mm]
			y  
			\end{array}
			\right)$.
		\end{enumerate}
	\end{ej}
	\textit{Solución.}
	
\end{frame}

%% ---------------------------------------------------------------------------------------------------
%
%\subsection{}
%%
%\begin{frame}\frametitle{Cambio de base}
%
%\begin{problock}{\textbf{Ejercicio 1}}
%	\justifying
%	Encuentre una fórmula para $T\left(
%	\begin{array}{@{\hspace{0.3\tabcolsep}}c@{\hspace{0.3\tabcolsep}}}
%	x  \\[1mm]
%	y  
%	\end{array}
%	\right)$, donde $T$ es la transformación lineal del ejemplo anterior.
%\end{problock}
%
%\end{frame}

% ---------------------------------------------------------------------------------------------------

\subsection{}
%
\begin{frame}\frametitle{Matrices similares}
	
	\begin{block}{\textbf{Definición 1 (matrices semejantes)}}
		\justifying
		Sean $A$ y $B$ matrices cuadradas $n\times n$. Se dice que $B$ es \textbf{semejante } a $A$ (o que $B$ es \textbf{similar} a $A$) si existe una matriz invertible $P$ tal que $B=P^{-1}AP$.
	\end{block}
	
	\begin{alertblock}{\textbf{Observación 4 }}		
		\begin{enumerate}[$a$]\justifying
			\item Si $A$ es semejante a $B$, entonces  $B$ es semejante a $A$. 
			\item Si $A$ es semejante a $B$,  entonces se dice que $A$ y $B$ son semejantes. 
			\item $A$ y $B$ son semejantes si existe una matriz invertible $P$ tal que $AP=PB$.
			\item De acuerdo a la definición anterior y la propiedad 3, las matrices de una transformación lineal $T:V\longrightarrow V$ con respecto a bases distintas de $V$, son semejantes.
		\end{enumerate}		
	\end{alertblock}
	

\end{frame}


% ---------------------------------------------------------------------------------------------------

\subsection{}
%
\begin{frame}\frametitle{Matrices similares}
	
	\begin{block}{\textbf{Definición 1 (matrices semejantes)}}
		\justifying
		Sean $A$ y $B$ matrices cuadradas $n\times n$. Se dice que $B$ es \textbf{semejante } a $A$ (o que $B$ es \textbf{similar} a $A$) si existe una matriz invertible $P$ tal que $B=P^{-1}AP$.
	\end{block}
	
	
	\begin{ej}{\textbf{Ejemplo 4}}
		\justifying
		Considere las matrices
		\[
		A = 
		\left( 
		\begin{array}{@{\hspace{1\tabcolsep}}rr}	
		1 & 2 \\[2mm] 
		0 & -1
		\end{array} 
		\right),\ \
		B = 
		\left( 
		\begin{array}{@{\hspace{1\tabcolsep}}rr}	
		1 & 0 \\[2mm] 
		-2 & -1		
		\end{array} 
		\right)
		\quad \text{y} \quad
		P = 
		\left( 
		\begin{array}{@{\hspace{1\tabcolsep}}rr}	
		1 & -1 \\[2mm] 
		1 & 1		
		\end{array} 
		\right).
		\]
		Muestre que la matriz $B$ es semejante a la matriz $A$.
	\end{ej}

	\textit{Solución.}
	
\end{frame}


%% ---------------------------------------------------------------------------------------------------
%
%\subsection{}
%%
%\begin{frame}\frametitle{Matrices semejantes}
%
%\begin{block}{\textbf{Definición 1 (matrices semejantes)}}
%	\justifying
%	Sean $A$ y $B$ matrices cuadradas $n\times n$. Se dice que $B$ es \textbf{semejante } a $A$ (o que $B$ es \textbf{similar} a $A$) si existe una matriz invertible $P$ tal que $B=P^{-1}AP$.
%\end{block}
%
%\begin{prop}{\textbf{Propiedad 4 (propiedades de las matrices similares)}}
%	\justifying
%	Sean $A, B$ y $C$ matrices cuadradas de $n\times n$. Entonces:
%	\begin{enumerate}[$a$]
%		\justifying
%		\item $A$ es semejante a $A$.
%		\item Si $A$ es semejante a $B$, entonces $B$ es semejante a $A$.
%		\item Si $A$ es semejante a $B$ y $B$ es semejante a $C$, entonces $A$ es semejante a $C$.		
%	\end{enumerate}
%\end{prop}
%
%\begin{alertblock}{\textbf{Observación 5 }}
%	\justifying
%	Si $B$ es semejante a $A$, entonces $A$ es semejante a $B$ y en tal caso simplemente se dice que 
%	$A$ y $B$ son semejantes.
%\end{alertblock}
%
%\end{frame}

% ---------------------------------------------------------------------------------------------------

\subsection{}
%
\begin{frame}\frametitle{Matrices semejantes}

\begin{prop}{\textbf{Propiedad 5 (propiedades de las matrices similares)}}
	\justifying
	Sean $A$ y $B$ matrices $n\times n$. Si $A$ y $B$ son semejantes, entonces:
	\begin{enumerate}[$a$]
		\justifying		
		\item $\det A = \det B$.
		\item $A$ es invertible si y sólo si $B$ es invertible.
		\item $A$ y $B$ tienen el mismo rango ($\rho(A)=\rho(B)$).
	\end{enumerate}
\end{prop}

\begin{ej}{\textbf{Ejemplo 5}}
	\justifying
	Muestre que las matrices $A$ y $B$ dadas a continuación no son semejantes.
	\[
	A = 
	\left( 
	\begin{array}{@{\hspace{1\tabcolsep}}rr}	
	1 & 2 \\[2mm] 
	2 & 1
	\end{array} 
	\right)
	\quad \text{y} \quad
	B = 
	\left( 
	\begin{array}{@{\hspace{1\tabcolsep}}rr}	
	2 & 1 \\[2mm] 
	1 & 2
	\end{array} 
	\right).
	\]
\end{ej}
\textit{Solución.}

\end{frame}
