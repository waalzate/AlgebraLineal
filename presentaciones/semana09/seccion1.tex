\section{Representación matricial de transformaciones lineales}

\subsection{}

{\nologo
\begin{frame}\frametitle{Matriz estándar de $T:\r^n\longrightarrow\r^m$}

\vspace{-1mm}
\begin{prop}{\textbf{Propiedad 1}}
	\justifying
	
	Sea $T:\r^n\longrightarrow\r^m$ una transformación lineal tal que
    
    \[
    T(\mathbf{e}_1)=
    \left(
    \begin{array}{@{\hspace{0.3\tabcolsep}}c@{\hspace{0.3\tabcolsep}}}
    a_{11}  \\[1mm]
    a_{21}  \\[1mm]
    \vdots \\[1mm]
    a_{m1}
    \end{array}
    \right)
    ,  \ T(\mathbf{e}_2)=
    \left(
    \begin{array}{@{\hspace{0.3\tabcolsep}}c@{\hspace{0.3\tabcolsep}}}
    a_{12}  \\[1mm]
    a_{22}  \\[1mm]
    \vdots \\[1mm]
    a_{m2}
    \end{array}
    \right),\ \dots\ , 
    \ T(\mathbf{e}_n)=
    \left(
    \begin{array}{@{\hspace{0.3\tabcolsep}}c@{\hspace{0.3\tabcolsep}}}
    a_{1n}  \\[1mm]
    a_{2n}  \\[1mm]
    \vdots \\[1mm]
    a_{mn}
    \end{array}
    \right).
    \]
    Entonces la matriz $m\times n$ cuyas columnas son los $T(\mathbf{e}_j)$, 
    \[
   A_T=\left(
   \begin{array}{@{\hspace{0.1\tabcolsep}}c@{\hspace{1.5\tabcolsep}}c@{\hspace{1.5\tabcolsep}}c@{\hspace{1.5\tabcolsep}}c@{\hspace{0.1\tabcolsep}}}
   a_{11} & a_{12} & \cdots & a_{1n} \\[1mm]
   a_{21} & a_{22} & \cdots & a_{2n} \\[1mm]
   \vdots & \vdots &        & \vdots \\[0mm]
   a_{m1} & a_{m2} & \cdots & a_{mn} \\[1mm]
   \end{array}
   \right)
    \]
    es tal que $T(\mathbf{v})=A_T\mathbf{v}$ para todo $\mathbf{v}\in\r^n$. La matriz $A_T$ es llamada \textbf{matriz estándar} 
    para $T$ o también la \textbf{representación matricial} de $T$.
\end{prop}	

\end{frame}
}

% ---------------------------------------------------------------------------------------------------

\subsection{}

\begin{frame}\frametitle{Matriz estándar de $T:\r^n\longrightarrow\r^m$}
	
	\begin{alertblock}{\textbf{Observación 1 }}
		
		\begin{itemize}
			\justifying
			\item[\labelname{$a$}] Se debe tener cuidado con la aplicación de la Propiedad 1. Solamente es válida para transformaciones lineales de $\r^n$ en $\r^m$.
			\item[\labelname{$b$}] La propiedad anterior nos dice que para hallar la matriz estándar de una transformación lineal que va de $\r^n$ a $\r^m$ basta hallar las imágenes de los vectores de la base estándar de $\r^n$ y ponerlas como columnas de la matriz.
			\item[\labelname{$c$}] Se puede demostrar que la matriz $A$ de la Propiedad 1 es la \textit{única} matriz que satisface que $T(\mathbf{v})=A\mathbf{v}$ para todo $\mathbf{v}\in\r^n$.
		\end{itemize}
		
	\end{alertblock}
	

	
\end{frame}

% ---------------------------------------------------------------------------------------------------

\subsection{}

\begin{frame}\frametitle{Matriz estándar de $T:\r^n\longrightarrow\r^m$}

\begin{ej}{\textbf{Ejemplo 1}}
	\justifying
	Determine la matriz estándar $A$ de la transformación lineal  $T:\r^3\longrightarrow\r^2$ definida por 
		\[
	T
	\left(
	\begin{array}{@{\hspace{0.3\tabcolsep}}c@{\hspace{0.3\tabcolsep}}}
	x  \\[1mm]
	y  \\[1mm]
	z
	\end{array}
	\right)
	=
	\left(
	\begin{array}{@{\hspace{0.3\tabcolsep}}c@{\hspace{0.3\tabcolsep}}}
	x-2y  \\[1mm]
	2x+y \\[1mm]
	\end{array}
	\right)
	\]
	y verifique que $T(\mathbf{v})=A\mathbf{v}$ para todo $\mathbf{v}\in\r^3$.
\end{ej}
\textit{Solución.}

\end{frame}

% ---------------------------------------------------------------------------------------------------

\subsection{}

{\nologo
\begin{frame}%\frametitle{Matriz de transformación para bases no estándar}

\vspace{-2mm}
\begin{prop}{\textbf{Propiedad 2} (Matriz de transformación para bases no estándar)}
	\justifying
	Sean $V$ y $W$ espacios vectoriales de dimensión finita con bases $\mathcal{B}_1$ y $\mathcal{B}_2$, respectivamente, donde $\mathcal{B}_1=\{\mathbf{v}_1, \mathbf{v}_2,\dots, \mathbf{v}_n \}$. Si $T:V\longrightarrow W$ es una transformación lineal tal que
	 \[
	\left[T(\mathbf{v_1})\right]_{\mathcal{B}_2}=
	\left(
	\begin{array}{@{\hspace{0.3\tabcolsep}}c@{\hspace{0.3\tabcolsep}}}
	a_{11}  \\[1mm]
	a_{21}  \\[1mm]
	\vdots \\[1mm]
	a_{m1}
	\end{array}
	\right)
	,  \ \left[T(\mathbf{v_2})\right]_{\mathcal{B}_2}=
	\left(
	\begin{array}{@{\hspace{0.3\tabcolsep}}c@{\hspace{0.3\tabcolsep}}}
	a_{12}  \\[1mm]
	a_{22}  \\[1mm]
	\vdots \\[1mm]
	a_{m2}
	\end{array}
	\right), \dots, 
	\ \left[T(\mathbf{v_n})\right]_{\mathcal{B}_2}=
	\left(
	\begin{array}{@{\hspace{0.3\tabcolsep}}c@{\hspace{0.3\tabcolsep}}}
	a_{1n}  \\[1mm]
	a_{2n}  \\[1mm]
	\vdots \\[1mm]
	a_{mn}
	\end{array}
	\right),
	\]
	entonces la matriz $m\times n$ cuyas $n$ columnas son los vectores $\left[T(\mathbf{v_j})\right]_{\mathcal{B}_2}$,
	\[
	A_T=\left(
	\begin{array}{@{\hspace{0.1\tabcolsep}}c@{\hspace{1.5\tabcolsep}}c@{\hspace{1.5\tabcolsep}}c@{\hspace{1.5\tabcolsep}}c@{\hspace{0.1\tabcolsep}}}
	a_{11} & a_{12} & \cdots & a_{1n} \\[1mm]
	a_{21} & a_{22} & \cdots & a_{2n} \\[1mm]
	\vdots & \vdots &        & \vdots \\[0mm]
	a_{m1} & a_{m2} & \cdots & a_{mn} \\[1mm]
	\end{array}
	\right)
	\]
	es tal que $\left[T(\mathbf{v})\right]_{\mathcal{B}_2}=A_T\left[\mathbf{v}\right]_{\mathcal{B}_1}$ para todo $\mathbf{v}\in V$. Esta matriz es llamada \textbf{la matriz de $T$ con respecto a las bases $\mathcal{B}_1$ y $\mathcal{B}_2$}.
\end{prop}	

\end{frame}
}

% ---------------------------------------------------------------------------------------------------

\subsection{}

\begin{frame}\frametitle{Matriz de transformación para bases no estándar}
	
	\begin{ej}{\textbf{Ejemplo 2 }}
		\justifying
		Considere la transformación lineal  $T:\r^2\longrightarrow\r^2$ definida por 
		\[
		T
		\left(
		\begin{array}{@{\hspace{0.3\tabcolsep}}c@{\hspace{0.3\tabcolsep}}}
		x  \\[1mm]
		y  
		\end{array}
		\right)
		=
		\left(
		\begin{array}{@{\hspace{0.3\tabcolsep}}c@{\hspace{0.3\tabcolsep}}}
		x+y  \\[1mm]
		2x-y 
		\end{array}
		\right)
		\]
		Encuentre la matriz de $T$ con respecto a las bases 
		\[
		\mathcal{B}_1 = \Big\{ \, \underbrace{(1,2)}_{\color{blue}\mathbf{v}_1}, \ \underbrace{(-1,1)}_{\color{blue}\mathbf{v}_2} \, \Big\} \quad \text{y} \quad 
		\mathcal{B}_2 = \Big\{ \, \underbrace{(1,0)}_{\color{blue}\mathbf{w}_1}, \ \underbrace{(0,1)}_{\color{blue}\mathbf{w}_2} \, \Big\}.
		\]
	\end{ej}
	\textit{Solución.}
	
\end{frame}

% ---------------------------------------------------------------------------------------------------

\subsection{}

\begin{frame}\frametitle{Matriz de transformación para bases no estándar}

\begin{ej}{\textbf{Ejemplo 3 }}
	\justifying
	Para la transformación lineal del ejemplo anterior, $T:\r^2\longrightarrow\r^2$ definida por 
	\[
	T
	\left(
	\begin{array}{@{\hspace{0.3\tabcolsep}}c@{\hspace{0.3\tabcolsep}}}
	x  \\[1mm]
	y  
	\end{array}
	\right)
	=
	\left(
	\begin{array}{@{\hspace{0.3\tabcolsep}}c@{\hspace{0.3\tabcolsep}}}
	x+y  \\[1mm]
	2x-y 
	\end{array}
	\right),
	\]
	encuentre $T(\mathbf{v})$, donde $\mathbf{v}=(2,1)$, de dos formas distintas:
	\begin{enumerate}[1.]
		\justifying
		\item[\labelname{$a$}] De manera directa, usando la definición de $T$.
		\item[\labelname{$b$}] Usando la matriz de $T$ con respecto a las bases $\mathcal{B}_1$ y $\mathcal{B}_2$.
	\end{enumerate}
	
	
\end{ej}
\textit{Solución.}

\end{frame}

% ---------------------------------------------------------------------------------------------------

\subsection{}

\begin{frame}\frametitle{Matriz de transformación para otros espacios vectoriales}

\begin{ej}{\textbf{Ejemplo 4 }}
	\justifying
	Sea $D_x:P_2\longrightarrow P_1$ el operador direrencial que transforma un polinomio cuadrático $p$ en su derivada $p'$. Ya sabemos que $D_x$ es una transformación lineal. Determine la matriz de $D_x$ con respecto a las bases
	\[
	\mathcal{B}_1=\{1,x,x^2\} \quad\text{y}\quad \mathcal{B}_2=\{1,x\}.
	\]
	Use la matriz hallada $A_T$ para verificar que $$\left[D_x(p)\right]_{\mathcal{B}_2}=A_T\left[p\right]_{\mathcal{B}_1}$$ con $p=a+bx+cx^2$.
	
\end{ej}
\textit{Solución.}

\end{frame}

% ---------------------------------------------------------------------------------------------------

\subsection{}

\begin{frame}\frametitle{Rango y nulidad de una transformación lineal}
	\vspace{-3mm}
	\begin{prop}{\textbf{Propiedad 3}}
		\justifying
		Sean $V$ y $W$ espacios vectoriales de dimensión finita con $\dim V=n$ y sea $T: V\longrightarrow W$ una trasnfomación lineal cuya matriz con respecto a \textit{ciertas} bases de $V$ y $W$, respectivamente, es $A$. Entonces:
		\begin{enumerate}[1.]
			\justifying
			\item[\labelname{$a$}] $\rho(T)=\rho(A)$.
			\item[\labelname{$b$}] $\nu(T)=\nu(A)$.
			\item[\labelname{$c$}] $\nu(T)+\rho(T)=n$.
		\end{enumerate}
	\end{prop}
	\begin{alertblock}{\textbf{Observación 2}}
		\justifying
		Los ítems (\textit{a}) y (\textit{b}) de la propiedad anterior implican que $\rho(A)$ y $\nu(A)$ son independientes de las bases escogidas para $V$ y $W$.
	\end{alertblock}

\end{frame}

% ---------------------------------------------------------------------------------------------------

\subsection{}

\begin{frame}\frametitle{Rango y nulidad de una transformación lineal}

\begin{ej}{\textbf{Ejemplo 5}}
	\justifying
	Considere la transformación lineal $T: P_3\longrightarrow P_2$ definida por 
	\[
		T\left(a_0+a_1x+a_2x^2+a_3x^3\right)=a_1+a_2x^2.
	\]
	Halle $A_T$ y utilícela para determinar el núcleo y la imagen de $T$.
\end{ej}
\textit{Solución.}

\end{frame}

% ---------------------------------------------------------------------------------------------------

\subsection{}

\begin{frame}\frametitle{Rango y nulidad de una transformación lineal}
	
	\begin{ej}{\textbf{Ejemplo 6}}
		\justifying
		Considere la transformación lineal $T: P_2\longrightarrow P_3$ definida por 
		\[
			(Tp)(x)= xp(x).
		\]
		Halle $A_T$ y utilícela para determinar el núcleo y la imagen de $T$.
	\end{ej}
	\textit{Solución.}
	
\end{frame}

% ---------------------------------------------------------------------------------------------------

\subsection{}

\begin{frame}%\frametitle{Composición de transformaciones lineales}

	%\vspace{-3mm}
	\begin{block}{\textbf{Definición 1 (Composición de transformaciones lineales)}}
		\justifying
		Sean $T:\r^n\longrightarrow\r^m$ y $S:\r^m\longrightarrow\r^p$ transformaciones lineales. La \textbf{composición} de $S$ con $T$, denotada por $S\circ T$, es una función de $\r^n$ en $\r^p$ definida por 
		\[
		(S\circ T)(\mathbf{v})=S(T(\mathbf{v}))
		\]
		para todo $\mathbf{v}\in \r^n$.
	\end{block}	

	\begin{ej}{\textbf{Ejemplo 7}}
		\justifying
		Considere las transformaciones lineales $T:\r^2\longrightarrow\r^3$ y $S:\r^3\longrightarrow\r^4$ definidas por
		\[
			T(x,y) = (x,2x-y,3x+4y) \quad \text{y} \quad S(x,y,z) = (2x+z,3y-z,x-y,x+y+z).
		\]		
		Encuentre $S\circ T:\r^2\longrightarrow\r^4$.
	\end{ej}
	\textit{Solución.}
	
\end{frame}

% ---------------------------------------------------------------------------------------------------

\subsection{}

\begin{frame}%\frametitle{Composición de transformaciones lineales}
	
	\begin{prop}{\textbf{Propiedad 4}}
		\justifying
		Sean $T:\r^n\longrightarrow\r^m$ y $S:\r^m\longrightarrow\r^p$ transformaciones lineales. Entonces la composición $S\circ T: \r^n\longrightarrow\r^p$ es una transformación lineal. Más aún, si $A_{T}$ es la matriz estándar para $T$ y 
		$A_{S}$ es la matriz estándar para $S$, entonces la matriz estándar para $S\circ T$ es $A_{S} A_{T}$, es decir,
		\[
		A_{S\circ T} = A_{S} A_{T}
		\]
	\end{prop}

	\begin{ej}{\textbf{Ejemplo 8}}
		\justifying
		Considere las transformaciones lineales $T_1$ y $T_2$ de $\r^3\longrightarrow\r^3$ definidas por
		\[
		T_1(x,y,z) = (2x+y,0,x+z) \quad \text{y} \quad T_2(x,y,z) = (x-y,z,y).
		\]		
		Encuentre las matrices estándar para  $T_1\circ T_2$ y $T_2\circ T_1$.
	\end{ej}
	\textit{Solución.}
	
\end{frame}

% ---------------------------------------------------------------------------------------------------

\subsection{}

%\begin{frame}\frametitle{Composición de transformaciones lineales}
%	\begin{prop}{\textbf{Propiedad 4}}
%		\justifying
%		Sean $T:\r^n\longrightarrow\r^m$ y $S:\r^m\longrightarrow\r^p$ transformaciones lineales. Entonces la composición $S\circ T: \r^n\longrightarrow\r^p$ es una transformación lineal. Más aún, si $A$ es la matriz estándar para $T$ y $B$ es la matriz estándar para $S$, entonces la matriz estándar para $S\circ T$ es $BA$.
%	\end{prop}
%	
%	\begin{ej}{\textbf{Ejemplo 6}}
%		\justifying
%		Considere las transformaciones lineales $T:\r^2\longrightarrow\r^3$ y $S:\r^3\longrightarrow\r^4$ definidas por
%		$$
%		\begin{array}{ccc}
%		T
%		\left(
%		\begin{array}{@{\hspace{0.3\tabcolsep}}c@{\hspace{0.3\tabcolsep}}}
%		x  \\[1mm]
%		y  
%		\end{array}
%		\right)
%		=
%		\left(
%		\begin{array}{@{\hspace{0.3\tabcolsep}}c@{\hspace{0.3\tabcolsep}}}
%		x \\[1mm]
%		2x-y\\[1mm]
%		3x+4y 
%		\end{array}
%		\right)&\text{ y } & S
%		\left(
%		\begin{array}{@{\hspace{0.3\tabcolsep}}c@{\hspace{0.3\tabcolsep}}}
%		x  \\[1mm]
%		y  \\[1mm]
%		z
%		\end{array}
%		\right)
%		=
%		\left(
%		\begin{array}{@{\hspace{0.3\tabcolsep}}c@{\hspace{0.3\tabcolsep}}}
%		2x+z\\[1mm]
%		3y-z\\[1mm]
%		x-y \\ [1mm]
%		x+y+z
%		\end{array}
%		\right)
%		\end{array}
%		$$
%		Encuentre $S\circ T:\r^2\longrightarrow\r^4$.
%	\end{ej}
%\end{frame}

% ---------------------------------------------------------------------------------------------------

\subsection{}

{\nologo
\begin{frame}\frametitle{Inversa de una transformación lineal}
	%\vspace{-3mm}
	\begin{block}{\textbf{Definición 2}}
		Sea $T:\r^n\longrightarrow\r^n$ una transformación lineal. Decimos que $T$ es \textbf{invertible} si existe una \textit{función} $S:\r^n\longrightarrow\r^n$  tal que
		\[
		S\circ T=I \quad\text{y}\quad T\circ S=I
		\]
		donde $I$ es la transformación identidad. En este caso se dice que $S$ es la \textbf{inversa} de $T$ o 
		también que $T$ es la \textbf{inversa} de $S$.
	\end{block}	
	
	\begin{alertblock}{\textbf{Observación 3}}
		
		\begin{itemize}
			\justifying
			\item[\labelname{$a$}] La condición dada en la definición anterior equivale a 
			\[
			S(T(\mathbf{v}))=\mathbf{v} \quad\text{y}\quad T(S(\mathbf{v}))=\mathbf{v} \quad \text{para todo} \quad \mathbf{v}\in\r^n.
			\] 
			\item[\labelname{$b$}] No toda transformación lineal $T$ tiene inversa. 
			\item[\labelname{$c$}] ¿Cuándo una transformación lineal $T$ tiene inversa? 
			\item[\labelname{$d$}] $S\circ T = I \  \Rightarrow \ A_{S\circ T} = A_SA_T = A_I = I_n
			\  \Rightarrow \  {A_S}^{-1} = A_T \text{ y } {A_T}^{-1} = A_S$
%			\item No toda transformación lineal $T$ tiene inversa. 
%			\item Si $T$ es invertible, su inversa es única y es denotada por $T^{-1}$ (teorema).
%			\item Una transformación lineal $T:\r^n\longrightarrow\r^n$ es invertible si y sólo si es uno a uno y sobre.
		\end{itemize}
	\end{alertblock}
\end{frame}
}

% ---------------------------------------------------------------------------------------------------

\subsection{}

\begin{frame}%\frametitle{Inversa de una transformación lineal}
	
	\begin{prop}{\textbf{Propiedad 5}}
		\justifying
		Sea $T:\r^n\longrightarrow\r^n$ una transformación lineal invertible. Entonces su función inversa $T^{-1}:\r^n\longrightarrow\r^n$ es también una transformación lineal. Más aún, si $A_T$ es la matriz estándar para $T$, entonces $A_T$ es invertible y ${A_T}^{-1}$ es la matriz estándar para $T^{-1}$, es decir, ${A_T}^{-1}=A_{T^{-1}}$.
	\end{prop}	
	
	\begin{ej}{\textbf{Ejemplo 9}}
		\justifying
		Considere la transformación lineal $T:\r^3\longrightarrow\r^3$ definida por
		\[
			T(x,y,z) = (2x+3y+z,3x+3y+z,2x+4y+z).	
		\]		
		Muestre que $T$ es invertible y encuentre su inversa $T^{-1}$.
	\end{ej}
	\textit{Solución.}
	
\end{frame}