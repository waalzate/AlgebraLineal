\section{Espacio nulo de una matriz}

\subsection{}

\begin{frame}\frametitle{Soluciones de un sistema lineal homogéneo}

\begin{defi}{\textbf{Definición 1}}
	Sea $A$ una matriz $m\times n$. Al conjunto de todas las soluciones del sistema lineal
	homogéneo
	\[
		A \mathbf{x} =  \mathbf{0}
	\]
	se le denomina el \textbf{\textit{espacio nulo}} de $A$ y se denota por $N_A$:
	\[
    	N_A = \{ \mathbf{x}\in \r^n \mid A \mathbf{x} =  \mathbf{0} \}	
	\]
	A la dimensión del espacio nulo de $A$ se le denomina \textbf{\textit{nulidad}} de $A$ y se denota por $\nu(A)$:
	\[
		\nu(A) = \dim N_A
	\]
\end{defi}	

%\begin{ej}{\textbf{Ejemplo 1}}
%	Encuentre el espacio nulo de la matriz
%	\[
%	A = 
%	\left( 
%	\begin{array}{@{\hspace{\tabcolsep}}rrrr}	
%	1 & \phantom{-}2 & -2 & \phantom{-}1 \\[1mm] 
%	3 & 6 & -5 & 4 \\[1mm] 
%	1 &  2 & 0 & 3
%	\end{array} 
%	\right)
%	\]
%\end{ej}

\end{frame}

%%------------------------------------------------------------------------------------------------------

\subsection{}

\begin{frame}\frametitle{Soluciones de un sistema lineal homogéneo}
			
	\begin{ej}{\textbf{Ejemplo 1}}
		Encuentre el espacio nulo de la matriz
		\[
		A = 
		\left( 
		\begin{array}{@{\hspace{\tabcolsep}}rrrr}	
		1 & \phantom{-}2 & -2 & \phantom{-}1 \\[1mm] 
		3 & 6 & -5 & 4 \\[1mm] 
		1 &  2 & 0 & 3
		\end{array} 
		\right)
		\]
	\end{ej}
	\textit{Solución}
	
\end{frame}

%%------------------------------------------------------------------------------------------------------


\subsection{}

{\nologo
\begin{frame}\frametitle{Rango y nulidad de una matriz}

\vspace{-3mm}
\begin{defi}{\textbf{Definición 1}}
	Sea $A$ una matriz $m\times n$. Al conjunto de todas las soluciones del sistema lineal
	homogéneo
	\[
	A \mathbf{x} =  \mathbf{0}
	\]
	se le denomina el \textbf{\textit{espacio nulo}} de $A$ y se denota por $N_A$:
	\[
	N_A = \{ \mathbf{x}\in \r^n \mid A \mathbf{x} =  \mathbf{0} \}	
	\]
	A la dimensión del espacio nulo de $A$ se le denomina \textbf{\textit{nulidad}} de $A$ y se denota por $\nu(A)$:
	\[
	\nu(A) = \dim N_A
	\]
\end{defi}	

\vspace{-2mm}

\begin{prop}{\textbf{Propiedad 1}}
	\justifying
	Sea $A$ una matriz $m\times n$. Entonces
	\[
		\rho(A) + \nu(A) = n
	\]
\end{prop}	

\vspace{-2mm}

\begin{alertblock}{\textbf{Observación 1}}
	La propiedad 4 nos dice que ``rango $+$ nulidad $=$ número de columnas''.
\end{alertblock}

\end{frame}
}

% ------------------------------------------------------------------------------------------------------

\subsection{}

\begin{frame}%\frametitle{Espacio renglón y espacio columna de una matriz}

\begin{ej}{\textbf{Ejemplo 2}}
	Considere la matriz
		\[
		A = 
		\left( 
		\begin{array}{@{\hspace{0.1\tabcolsep}}rrrrr}	
		 1 & \phantom{-1}0 & \phantom{1}-2 & \phantom{-1}1 & 0 \\[2mm] 
		 0 & -1 & -3 & 1 & 3 \\[2mm] 
		-2 & -1 & 1 & -1 & 3 \\[2mm] 
		 0 & 3 & 9 & 0 & -12 
		\end{array} 
		\right)
		\]
	
	\vspace{-1mm}	
	\begin{enumerate}
		\item[\labelname{$a$}] Encuentre el rango y la nulidad de $A$. \\[2mm]
		%\item[\labelname{$b$}] Encuentre un subconjunto de vectores columna de $A$ que formen una base para $C_A$.\\[2mm]
		%\item[\labelname{$c$}] Si es posible, escriba la tercera columna de $A$ como combinación lineal de las dos primeras.
	\end{enumerate}
\end{ej}
\textit{Solución}

\end{frame}

% ------------------------------------------------------------------------------------------------------

\subsection{}

\begin{frame}%\frametitle{Espacio renglón y espacio columna de una matriz}
	
	\begin{ej}{\textbf{Ejemplo 2}}
		Considere la matriz
		\[
		A = 
		\left( 
		\begin{array}{@{\hspace{0.1\tabcolsep}}rrrrr}	
		1 & \phantom{-1}0 & \phantom{1}-2 & \phantom{-1}1 & 0 \\[2mm] 
		0 & -1 & -3 & 1 & 3 \\[2mm] 
		-2 & -1 & 1 & -1 & 3 \\[2mm] 
		0 & 3 & 9 & 0 & -12 
		\end{array} 
		\right)
		\]
		
		\vspace{-1mm}	
		\begin{enumerate}
			%\item[\labelname{$a$}] Encuentre el rango y la nulidad de $A$. \\[2mm]
			\item[\labelname{$b$}] Encuentre un subconjunto de vectores columna de $A$ que formen una base para $C_A$.\\[2mm]
			%\item[\labelname{$c$}] Si es posible, escriba la tercera columna de $A$ como combinación lineal de las dos primeras.
		\end{enumerate}
	\end{ej}
	\textit{Solución}
	
\end{frame}

% ------------------------------------------------------------------------------------------------------

\subsection{}

\begin{frame}%\frametitle{Espacio renglón y espacio columna de una matriz}
	
	\begin{ej}{\textbf{Ejemplo 2}}
		Considere la matriz
		\[
		A = 
		\left( 
		\begin{array}{@{\hspace{0.1\tabcolsep}}rrrrr}	
		1 & \phantom{-1}0 & \phantom{1}-2 & \phantom{-1}1 & 0 \\[2mm] 
		0 & -1 & -3 & 1 & 3 \\[2mm] 
		-2 & -1 & 1 & -1 & 3 \\[2mm] 
		0 & 3 & 9 & 0 & -12 
		\end{array} 
		\right)
		\]
		
		\vspace{-1mm}	
		\begin{enumerate}
			%\item[\labelname{$a$}] Encuentre el rango y la nulidad de $A$. \\[2mm]
			%\item[\labelname{$b$}] Encuentre un subconjunto de vectores columna de $A$ que formen una base para $C_A$.\\[2mm]
			\item[\labelname{$c$}] Si es posible, escriba la tercera columna de $A$ como combinación lineal de las dos primeras.
		\end{enumerate}
	\end{ej}
	\textit{Solución}
	
\end{frame}

%%------------------------------------------------------------------------------------------------------

\subsection{}

{\nologo
\begin{frame}\frametitle{Consistencia de un sistema de ecuaciones lineales}

\vspace{-3mm}

\begin{columns}[c]
	\begin{column}{1.15\textwidth}

\[
	A\mathbf{x} = \mathbf{b} 
	\qquad \Longleftrightarrow \qquad 
	{\color{blue} 
	\left(
	\begin{array}{@{\hspace{0.1\tabcolsep}}c@{\hspace{1.5\tabcolsep}}c@{\hspace{1.5\tabcolsep}}c@{\hspace{1.5\tabcolsep}}c@{\hspace{0.1\tabcolsep}}}
	a_{11} & a_{12} & \cdots & a_{1n} \\[1mm]
	a_{21} & a_{22} & \cdots & a_{2n} \\[1mm]
	\vdots & \vdots &        & \vdots \\[0mm]
	a_{m1} & a_{m2} & \cdots & a_{mn} \\[1mm]
	\end{array}
	\right)
	\left(
	\begin{array}{@{\hspace{0.1\tabcolsep}}c@{\hspace{0.1\tabcolsep}}}
	x_{1} \\[1mm]
	x_{2} \\[1mm]
	\vdots \\[0mm]
	x_{m} \\[1mm]
	\end{array}
	\right)
	}
	=
	{\color{red}
	\left(
	\begin{array}{@{\hspace{0.1\tabcolsep}}c@{\hspace{0.1\tabcolsep}}}
	b_{1} \\[1mm]
	b_{2} \\[1mm]
	\vdots \\[0mm]
	b_{m} \\[1mm]
	\end{array}
	\right)
}
\]

\vspace{1mm}

\[
{\color{blue}
\left(
\begin{array}{@{\hspace{0.1\tabcolsep}}c@{\hspace{1.5\tabcolsep}}c@{\hspace{1.5\tabcolsep}}c@{\hspace{1.5\tabcolsep}}c@{\hspace{0.1\tabcolsep}}}
a_{11} & a_{12} & \cdots & a_{1n} \\[1mm]
a_{21} & a_{22} & \cdots & a_{2n} \\[1mm]
\vdots & \vdots &        & \vdots \\[0mm]
a_{m1} & a_{m2} & \cdots & a_{mn} \\[1mm]
\end{array}
\right)
\left(
\begin{array}{@{\hspace{0.1\tabcolsep}}c@{\hspace{0.1\tabcolsep}}}
x_{1} \\[1mm]
x_{2} \\[1mm]
\vdots \\[0mm]
x_{m} \\[1mm]
\end{array}
\right)}\ 
=\ 
x_1
\left(
\begin{array}{@{\hspace{0.1\tabcolsep}}c@{\hspace{0.1\tabcolsep}}}
a_{11} \\[1mm]
a_{21} \\[1mm]
\vdots \\[0mm]
a_{m1} \\[1mm]
\end{array}
\right)\ 
+\ 
%x_2
%\left(
%\begin{array}{@{\hspace{0.1\tabcolsep}}c@{\hspace{0.1\tabcolsep}}}
%a_{12} \\[1mm]
%a_{22} \\[1mm]
%\vdots \\[0mm]
%a_{m2} \\[1mm]
%\end{array}
%\right)\ 
%+\ 
\cdots\ 
+\ 
x_n
\left(
\begin{array}{@{\hspace{0.1\tabcolsep}}c@{\hspace{0.1\tabcolsep}}}
a_{1n} \\[1mm]
a_{2n} \\[1mm]
\vdots \\[0mm]
a_{mn} \\[1mm]
\end{array}
\right)
=
{\color{red}
\left(
\begin{array}{@{\hspace{0.1\tabcolsep}}c@{\hspace{0.1\tabcolsep}}}
b_{1} \\[1mm]
b_{2} \\[1mm]
\vdots \\[0mm]
b_{m} \\[1mm]
\end{array}
\right)
}
\]

\vspace{2mm}
\[
A\mathbf{x} = \mathbf{b} 
\quad \Longleftrightarrow \quad 
{\color{red}
\left(
\begin{array}{@{\hspace{0.1\tabcolsep}}c@{\hspace{0.1\tabcolsep}}}
b_{1} \\[1mm]
b_{2} \\[1mm]
\vdots \\[0mm]
b_{m} \\[1mm]
\end{array}
\right)
}
\in 
\text{ gen}
\left\{
\left(
\begin{array}{@{\hspace{0.1\tabcolsep}}c@{\hspace{0.1\tabcolsep}}}
a_{11} \\[1mm]
a_{21} \\[1mm]
\vdots \\[0mm]
a_{m1} \\[1mm]
\end{array}
\right)%\ 
%,
%\left(
%\begin{array}{@{\hspace{0.1\tabcolsep}}c@{\hspace{0.1\tabcolsep}}}
%a_{12} \\[1mm]
%a_{22} \\[1mm]
%\vdots \\[0mm]
%a_{m2} \\[1mm]
%\end{array}
%\right)\ 
,
\hdots\ 
,
\left(
\begin{array}{@{\hspace{0.1\tabcolsep}}c@{\hspace{0.1\tabcolsep}}}
a_{1n} \\[1mm]
a_{2n} \\[1mm]
\vdots \\[0mm]
a_{mn} \\[1mm]
\end{array}
\right)
\right\}
\quad \Longleftrightarrow \quad 
\mathbf{b} \in C_A
\]

\end{column}
\end{columns}

%	\begin{array}{@{\hspace{0.1\tabcolsep}}c@{\hspace{\tabcolsep}}c@{\hspace{\tabcolsep}}c@{\hspace{\tabcolsep}}c@{\hspace{\tabcolsep}}c@{\hspace{\tabcolsep}}c@{\hspace{\tabcolsep}}c@{\hspace{\tabcolsep}}c@{\hspace{\tabcolsep}}c@{\hspace{0.1\tabcolsep}}}
%	a_{11}x_1 & + & a_{12}x_2 & + & \cdots & + & a_{1n}x_n & = & b_1 \\[1mm]
%	a_{21}x_1 & + & a_{22}x_2 & + & \cdots & + & a_{2n}x_n & = & b_2 \\[1mm]
%	\vdots          &  & \vdots          &  &  &  & \vdots          &   & \vdots \\[1mm]
%	a_{m1}x_1 & + & a_{m2}x_2 & + & \cdots & + & a_{mn}x_n & = & b_m 
%	\end{array}

\end{frame}
}

%%------------------------------------------------------------------------------------------------------

\subsection{}

\begin{frame}%\frametitle{Consistencia de un sistema de ecuaciones lineales}

\begin{prop}{\textbf{Propiedad 2}}
	\justifying
	Un sistema de ecuaciones lineales $A\mathbf{x}=\mathbf{b}$ es consistente si y sólo si $\mathbf{b}$ está en el espacio columna de $A$.
	Esto ocurrirá si y sólo si $A$ y la matriz aumentada $(A,b)$ tiene el mismo rango.
\end{prop}	

\vspace{-1mm}

\begin{ej}{\textbf{Ejemplo 3}}
	Determine si el sistema de ecuaciones lineales
	\[
		\begin{array}{@{\hspace{0.1\tabcolsep}}c@{\hspace{\tabcolsep}}c@{\hspace{\tabcolsep}}c@{\hspace{\tabcolsep}}c@{\hspace{\tabcolsep}}c@{\hspace{\tabcolsep}}c@{\hspace{\tabcolsep}}r@{\hspace{0.1\tabcolsep}}}
		x_1 &  + & x_2  & - & x_3 & = & -1 \\[1mm]
		x_1 &    &      & + & x_3 & = & 3 \\[1mm]
		3x_1 & + & 2x_2 & - & x_3 & = & 1 \\[1mm]
		\end{array}
	\]
	
	\vspace{-2mm}
	tiene solución.
\end{ej}
\textit{Solución}

\end{frame}

%%------------------------------------------------------------------------------------------------------

\subsection{}

\begin{frame}\frametitle{Sistema de ecuaciones lineales con matriz de coeficientes cuadrada}

\begin{prop}{\textbf{Propiedad 3}}
	%\justifying
	Sea $A$ una matriz $n\times n$. Entonces las siguientes afirmaciones son equivalentes:
	\begin{enumerate}[$a$]
		\item $A$ es no singular (invertible).
		\item $A\mathbf{x}=\mathbf{b}$ tiene solución única para cada $\mathbf{b}$.
		\item $A\mathbf{x}=\mathbf{0}$ tiene solamente la solución trivial.
		\item $A$ es equivalente (por renglones) a $I_n$.
		\item $\det A \neq 0$.
		\item $\rho(A) = n$.
		\item Los $n$ vectores renglones de $A$ son linealmente independientes.
		\item Los $n$ vectores columna de $A$ son linealmente independientes.
	\end{enumerate}
\end{prop}	

\end{frame}