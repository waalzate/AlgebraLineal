\section{Espacio renglón y espacio columna de una matriz}

\subsection{}

{\nologo
\begin{frame}\frametitle{Vectores renglón (fila) y vectores columna}

\begin{columns}[c]
	\begin{column}{1.15\textwidth}
		\[
		\begin{array}{@{\hspace{0.1\tabcolsep}}c@{\hspace{2\tabcolsep}}c@{\hspace{2\tabcolsep}}c@{\hspace{0.1\tabcolsep}}}
		\text{\textbf{Matriz }} A & \text{\textbf{Vectores renglón de }} A & \text{\textbf{Vectores columna de} } A \\[-8mm] 
		{\left(
		\begin{array}{@{\hspace{0.1\tabcolsep}}c@{\hspace{1.5\tabcolsep}}c@{\hspace{1.5\tabcolsep}}c@{\hspace{1.5\tabcolsep}}c@{\hspace{0.1\tabcolsep}}}
		a_{11} & a_{12} & \cdots & a_{1n} \\[1mm]
		a_{21} & a_{22} & \cdots & a_{2n} \\[1mm]
		\vdots & \vdots &        & \vdots \\[0mm]
		a_{m1} & a_{m2} & \cdots & a_{mn} \\[1mm]
		\end{array}
		\right)
		\atop		
		}
		& 
		\begin{array}{@{\hspace{0.1\tabcolsep}}c@{\hspace{0.5\tabcolsep}}c@{\hspace{0.5\tabcolsep}}l@{\hspace{0.1\tabcolsep}}}
		\mathbf{r}_1 & = & (a_{11},a_{12},\hdots, a_{1n})\\[1mm]
		\mathbf{r}_2 & = & (a_{21},a_{22},\hdots, a_{2n})\\
		& \vdots & \\[1mm]
		\mathbf{r}_m  & = & (a_{m1},a_{m2},\hdots, a_{mn})\\[1.1cm]
		\end{array}
		& 
		{
		\begin{array}{@{\hspace{0.1\tabcolsep}}c@{\hspace{1.3\tabcolsep}}c@{\hspace{1.3\tabcolsep}}c@{\hspace{1.3\tabcolsep}}c@{\hspace{0.1\tabcolsep}}}\\[8mm]
		\underbrace{\!\!
		\left(
		\begin{array}{@{\hspace{0.1\tabcolsep}}c@{\hspace{0.1\tabcolsep}}}
		a_{11}\\[1mm]
		a_{21}\\[1mm]
		\vdots \\[0mm]
		a_{m1}
		\end{array}
		\right)
		\!\!}_{\mathbf{c}_1}
		& 
		\underbrace{\!\!
		\left(
		\begin{array}{@{\hspace{0.1\tabcolsep}}c@{\hspace{0.1\tabcolsep}}}
		a_{12}\\[1mm]
		a_{22}\\[1mm]
		\vdots \\[0mm]
		a_{m2}
		\end{array}
		\right)
		\!\!}_{\mathbf{c}_2}
		& 
		\cdots
		& 
		\underbrace{\!\!
		\left(
		\begin{array}{@{\hspace{0.1\tabcolsep}}c@{\hspace{0.1\tabcolsep}}}
		a_{1n}\\[1mm]
		a_{2n}\\[1mm]
		\vdots \\[0mm]
		a_{mn}
		\end{array}
		\right)
		\!\!}_{\mathbf{c}_n}
		\end{array}
		\atop
		}				
		\end{array}		
		\]
	\end{column}
\end{columns}

\vspace{-2mm}

\begin{ej}{\textbf{Ejemplo 1}}
	
	\vspace{-2mm}
	
	\[
	\begin{array}{@{\hspace{0.1\tabcolsep}}c@{\hspace{4\tabcolsep}}c@{\hspace{4\tabcolsep}}c@{\hspace{0.1\tabcolsep}}}
	A & \text{\textbf{Vectores renglón}} & \text{\textbf{Vectores columna} }  \\[-8mm] 
	{\left(
		\begin{array}{@{\hspace{0.1\tabcolsep}}r@{\hspace{1.5\tabcolsep}}r@{\hspace{1.5\tabcolsep}}r@{\hspace{0.1\tabcolsep}}}
		 0 & \phantom{-}1 & -1 \\[1mm]
		-2 & 3 & 4 \\[1mm]
		\end{array}
		\right)
		\atop		
	}
	& 
	\begin{array}{@{\hspace{0.1\tabcolsep}}c@{\hspace{0.5\tabcolsep}}c@{\hspace{0.5\tabcolsep}}l@{\hspace{0.1\tabcolsep}}}
	\mathbf{r}_1 & = & (0,1,-1)\\[1mm]
	\mathbf{r}_2 & = & (-2,3,4)\\[6mm]
	\end{array}
	& 
	{
		\begin{array}{@{\hspace{0.1\tabcolsep}}c@{\hspace{1.3\tabcolsep}}c@{\hspace{1.3\tabcolsep}}c@{\hspace{0.1\tabcolsep}}}\\[8mm]
		\underbrace{\!\!
			\left(
			\begin{array}{@{\hspace{0.1\tabcolsep}}r@{\hspace{0.1\tabcolsep}}}
			0\\[1mm]
			-2
			\end{array}
			\right)
			\!\!}_{\mathbf{c}_1}
		& 
		\underbrace{\!\!
			\left(
			\begin{array}{@{\hspace{0.5\tabcolsep}}c@{\hspace{0.5\tabcolsep}}}
			1\\[1mm]
			3
			\end{array}
			\right)
			\!\!}_{\mathbf{c}_2}
		& 
		\underbrace{\!\!
			\left(
			\begin{array}{@{\hspace{0.1\tabcolsep}}r@{\hspace{0.1\tabcolsep}}}
			-1\\[1mm]
			4
			\end{array}
			\right)
			\!\!}_{\mathbf{c}_3}
		\end{array}
		\atop
	}				
	\end{array}		
	\]
\end{ej}	

\end{frame}
}

%%------------------------------------------------------------------------------------------------------

\subsection{}

{\nologo
\begin{frame}\frametitle{Espacio renglón y espacio columna de una matriz}

\begin{columns}[c]
	\begin{column}{1.15\textwidth}
		\[
		\begin{array}{@{\hspace{0.1\tabcolsep}}c@{\hspace{2\tabcolsep}}c@{\hspace{2\tabcolsep}}c@{\hspace{0.1\tabcolsep}}}
		\text{\textbf{Matriz }} A & \text{\textbf{Vectores renglón de }} A & \text{\textbf{Vectores columna de} } A \\[-8mm] 
		{\left(
			\begin{array}{@{\hspace{0.1\tabcolsep}}c@{\hspace{1.5\tabcolsep}}c@{\hspace{1.5\tabcolsep}}c@{\hspace{1.5\tabcolsep}}c@{\hspace{0.1\tabcolsep}}}
			a_{11} & a_{12} & \cdots & a_{1n} \\[1mm]
			a_{21} & a_{22} & \cdots & a_{2n} \\[1mm]
			\vdots & \vdots &        & \vdots \\[0mm]
			a_{m1} & a_{m2} & \cdots & a_{mn} \\[1mm]
			\end{array}
			\right)
			\atop		
		}
		& 
		\begin{array}{@{\hspace{0.1\tabcolsep}}c@{\hspace{0.5\tabcolsep}}c@{\hspace{0.5\tabcolsep}}l@{\hspace{0.1\tabcolsep}}}
		\mathbf{r}_1 & = & (a_{11},a_{12},\hdots, a_{1n})\\[1mm]
		\mathbf{r}_2 & = & (a_{21},a_{22},\hdots, a_{2n})\\
		& \vdots & \\[1mm]
		\mathbf{r}_m  & = & (a_{m1},a_{m2},\hdots, a_{mn})\\[1.1cm]
		\end{array}
		& 
		{
			\begin{array}{@{\hspace{0.1\tabcolsep}}c@{\hspace{1.3\tabcolsep}}c@{\hspace{1.3\tabcolsep}}c@{\hspace{1.3\tabcolsep}}c@{\hspace{0.1\tabcolsep}}}\\[8mm]
			\underbrace{\!\!
				\left(
				\begin{array}{@{\hspace{0.1\tabcolsep}}c@{\hspace{0.1\tabcolsep}}}
				a_{11}\\[1mm]
				a_{21}\\[1mm]
				\vdots \\[0mm]
				a_{m1}
				\end{array}
				\right)
				\!\!}_{\mathbf{c}_1}
			& 
			\underbrace{\!\!
				\left(
				\begin{array}{@{\hspace{0.1\tabcolsep}}c@{\hspace{0.1\tabcolsep}}}
				a_{12}\\[1mm]
				a_{22}\\[1mm]
				\vdots \\[0mm]
				a_{m2}
				\end{array}
				\right)
				\!\!}_{\mathbf{c}_2}
			& 
			\cdots
			& 
			\underbrace{\!\!
				\left(
				\begin{array}{@{\hspace{0.1\tabcolsep}}c@{\hspace{0.1\tabcolsep}}}
				a_{1n}\\[1mm]
				a_{2n}\\[1mm]
				\vdots \\[0mm]
				a_{mn}
				\end{array}
				\right)
				\!\!}_{\mathbf{c}_n}
			\end{array}
			\atop
		}				
		\end{array}		
		\]
	\end{column}
\end{columns}

\vspace{-5mm}


\begin{defi}{\textbf{Definición 1}}
	Sea $A$ una matriz ${\color{blue}m}\times {\color{red}n}$.
	\begin{enumerate}
		\item[\labelname{$a$}] El \textbf{\textit{espacio renglón}} de $A$ es el subespacio de $\r^{\color{red}n}$ generado por los renglones de $A$:
		\[
			R_A = \text{gen}\, \{ \mathbf{r}_1, \mathbf{r}_2, \hdots,\mathbf{r}_{\color{blue}m} \}
		\]
		\item[\labelname{$b$}] El \textbf{\textit{espacio columna}} de $A$ es el subespacio de $\r^{\color{blue}m}$ generado por las columnas de $A$:
		\[
			C_A = \text{gen}\, \{ \mathbf{c}_1, \mathbf{c}_2, \hdots,\mathbf{c}_{\color{red}n} \}
		\]
	\end{enumerate}
\end{defi}	

\end{frame}
}

%%------------------------------------------------------------------------------------------------------

\subsection{}

{\nologo
\begin{frame}%\frametitle{Espacio renglón y espacio columna de una matriz}

\begin{defi}{\textbf{Definición 1}}
	Sea $A$ una matriz ${\color{blue}m}\times {\color{red}n}$.
	\begin{enumerate}
		\item[\labelname{$a$}] El \textbf{\textit{espacio renglón}} de $A$ es el subespacio de $\r^{\color{red}n}$ generado por los renglones de $A$:
		\[
		R_A = \text{gen}\, \{ \mathbf{r}_1, \mathbf{r}_2, \hdots,\mathbf{r}_{\color{blue}m} \}
		\]
		\item[\labelname{$b$}] El \textbf{\textit{espacio columna}} de $A$ es el subespacio de $\r^{\color{blue}m}$ generado por las columnas de $A$:
		\[
		C_A = \text{gen}\, \{ \mathbf{c}_1, \mathbf{c}_2, \hdots,\mathbf{c}_{\color{red}n} \}
		\]
	\end{enumerate}
\end{defi}	

%\vspace{3mm}

\begin{ej}{\textbf{Ejemplo  2}}
	Determine el \textit{espacio renglón} y el \textit{espacio columna} de la matriz
		\[
		A = 
		\left( 
		\begin{array}{cc}	
		1 & 2 \\ 
		3 & 6 
		\end{array} 
		\right)
		\]
\end{ej}
\textit{Solución.}

\end{frame}
}

%%------------------------------------------------------------------------------------------------------

\subsection{}

{\nologo
\begin{frame}%\frametitle{Espacio renglón y espacio columna de una matriz}

\begin{defi}{\textbf{Definición 1}}
	Sea $A$ una matriz ${\color{blue}m}\times {\color{red}n}$.
	\begin{enumerate}
		\item[\labelname{$a$}] El \textbf{\textit{espacio renglón}} de $A$ es el subespacio de $\r^{\color{red}n}$ generado por los renglones de $A$:
		\[
		R_A = \text{gen}\, \{ \mathbf{r}_1, \mathbf{r}_2, \hdots,\mathbf{r}_{\color{blue}m} \}
		\]
		\item[\labelname{$b$}] El \textbf{\textit{espacio columna}} de $A$ es el subespacio de $\r^{\color{blue}m}$ generado por las columnas de $A$:
		\[
		C_A = \text{gen}\, \{ \mathbf{c}_1, \mathbf{c}_2, \hdots,\mathbf{c}_{\color{red}n} \}
		\]
	\end{enumerate}
\end{defi}	

%\vspace{3mm}

\begin{ej}{\textbf{Ejemplo  3}}
	Halle una base para el \textit{espacio renglón} de la matriz
	\[
	A = 
	\left( 
	\begin{array}{rrrr}	
	1 & -2 & \phantom{-}5 & \phantom{-}3 \\ 
	-3 & 4 & 0 & 1 \\
	0 & 0 & 0 & 0
	\end{array} 
	\right)
	\]
\end{ej}
\textit{Solución.}

\end{frame}
}

%%------------------------------------------------------------------------------------------------------

\subsection{}

\begin{frame}%\frametitle{Espacio renglón y espacio columna de una matriz}
	
	%\begin{defi}{\textbf{Definición 1}}
	%	Sea $A$ una matriz ${\color{blue}m}\times {\color{red}n}$.
	%	\begin{enumerate}
	%		\item[\labelname{$a$}] El \textbf{\textit{espacio renglón}} de $A$ es el subespacio de $\r^{\color{red}n}$ generado por los renglones de $A$:
	%		\[
	%		R_A = \text{gen}\, \{ \mathbf{r}_1, \mathbf{r}_2, \hdots,\mathbf{r}_{\color{blue}m} \}
	%		\]
	%		\item[\labelname{$b$}] El \textbf{\textit{espacio columna}} de $A$ es el subespacio de $\r^{\color{blue}m}$ generado por las columnas de $A$:
	%		\[
	%		C_A = \text{gen}\, \{ \mathbf{c}_1, \mathbf{c}_2, \hdots,\mathbf{c}_{\color{red}n} \}
	%		\]
	%	\end{enumerate}
	%\end{defi}	
	
	\vspace{0mm}
	
	\begin{ej}{\textbf{Ejemplo  4}}
		Considere la matriz
		\[
		A = 
		\left( 
		\begin{array}{rr}	
		1 & -1  \\
		0 &  1 \\
		3 & -3
		\end{array} 
		\right).
		\]
		\begin{enumerate}
			\item[\labelname{$a$}] Determine si $\mathbf{b} = 
			\left( 
			\begin{array}{r}	
			1 \\
			2 \\
			3 
			\end{array} 
			\right)$ está en $C_A$ y describa a $C_A$.
			%\item[\labelname{$b$}] Determine si $\mathbf{w} = (4, 5)$ está en $R_A$ y describa a $R_A$.
			%\item[\labelname{$c$}] Describa a $R_A$ y a $C_A$.
		\end{enumerate}
	\end{ej}
	\textit{Solución.}
	
\end{frame}

%%------------------------------------------------------------------------------------------------------

\subsection{}

\begin{frame}%\frametitle{Espacio renglón y espacio columna de una matriz}

%\begin{defi}{\textbf{Definición 1}}
%	Sea $A$ una matriz ${\color{blue}m}\times {\color{red}n}$.
%	\begin{enumerate}
%		\item[\labelname{$a$}] El \textbf{\textit{espacio renglón}} de $A$ es el subespacio de $\r^{\color{red}n}$ generado por los renglones de $A$:
%		\[
%		R_A = \text{gen}\, \{ \mathbf{r}_1, \mathbf{r}_2, \hdots,\mathbf{r}_{\color{blue}m} \}
%		\]
%		\item[\labelname{$b$}] El \textbf{\textit{espacio columna}} de $A$ es el subespacio de $\r^{\color{blue}m}$ generado por las columnas de $A$:
%		\[
%		C_A = \text{gen}\, \{ \mathbf{c}_1, \mathbf{c}_2, \hdots,\mathbf{c}_{\color{red}n} \}
%		\]
%	\end{enumerate}
%\end{defi}	

\vspace{0mm}

\begin{ej}{\textbf{Ejemplo  4}}
	Considere la matriz
	\[
	A = 
	\left( 
	\begin{array}{rr}	
	1 & -1  \\
	0 &  1 \\
	3 & -3
	\end{array} 
	\right).
	\]
	\begin{enumerate}
%		\item[\labelname{$a$}] Determine si $\mathbf{b} = 
%		\left( 
%		\begin{array}{r}	
%		1 \\
%		2 \\
%		3 
%		\end{array} 
%		\right)$ está en $C_A$ y describa a $C_A$.
		\item[\labelname{$b$}] Determine si $\mathbf{w} = (4, 5)$ está en $R_A$ y describa a $R_A$.
		%\item[\labelname{$c$}] Describa a $R_A$ y a $C_A$.
	\end{enumerate}
\end{ej}
\textit{Solución.}

\end{frame}

%%------------------------------------------------------------------------------------------------------

\subsection{}

{\nologo
\begin{frame}%\frametitle{Espacio renglón y espacio columna de una matriz}

\begin{defi}{\textbf{Definición 1}}
	Sea $A$ una matriz ${\color{blue}m}\times {\color{red}n}$.
	\begin{enumerate}
		\item[\labelname{$a$}] El \textbf{\textit{espacio renglón}} de $A$ es el subespacio de $\r^{\color{red}n}$ generado por los renglones de $A$:
		\[
		R_A = \text{gen}\, \{ \mathbf{r}_1, \mathbf{r}_2, \hdots,\mathbf{r}_{\color{blue}m} \}
		\]
		\item[\labelname{$b$}] El \textbf{\textit{espacio columna}} de $A$ es el subespacio de $\r^{\color{blue}m}$ generado por las columnas de $A$:
		\[
		C_A = \text{gen}\, \{ \mathbf{c}_1, \mathbf{c}_2, \hdots,\mathbf{c}_{\color{red}n} \}
		\]
	\end{enumerate}
\end{defi}	

\vspace{-1mm}

\begin{prop}{\textbf{Propiedad 1 (base para el espacio renglón)}}
	\justifying
	Si $A$ y $B$ son matrices $m\times n$ equivalentes por renglones, entonces el espacio
	renglón de $A$ es igual al espacio renglón de $B$, es decir, $R_A = R_B$.
\end{prop}	

\vspace{-1mm}

\begin{alertblock}{\textbf{Observación 1}}
	\begin{enumerate}
		\item[\labelname{$a$}] La propiedad 1 establece que el \textit{espacio renglón} de una matriz
		no se modifica por la aplicación de operaciones elementales en los renglones.	
		\item[\labelname{$b$}] La aplicación de operaciones elementales en los renglones de una matriz
		\textit{puede} modificar el \textit{espacio columna}.	
	\end{enumerate}
\end{alertblock}
	
\end{frame}
}

%%------------------------------------------------------------------------------------------------------

\subsection{}

\begin{frame}%\frametitle{Base para el espacio renglón de una matriz}

\begin{defi}{\textbf{Definición 1}}
	Sea $A$ una matriz ${\color{blue}m}\times {\color{red}n}$.
	\begin{enumerate}
		\item[\labelname{$a$}] El \textbf{\textit{espacio renglón}} de $A$ es el subespacio de $\r^{\color{red}n}$ generado por los renglones de $A$:
		\[
		R_A = \text{gen}\, \{ \mathbf{r}_1, \mathbf{r}_2, \hdots,\mathbf{r}_{\color{blue}m} \}
		\]
		\item[\labelname{$b$}] El \textbf{\textit{espacio columna}} de $A$ es el subespacio de $\r^{\color{blue}m}$ generado por las columnas de $A$:
		\[
		C_A = \text{gen}\, \{ \mathbf{c}_1, \mathbf{c}_2, \hdots,\mathbf{c}_{\color{red}n} \}
		\]
	\end{enumerate}
\end{defi}	

%\vspace{3mm}

\begin{prop}{\textbf{Propiedad 2 (base para el espacio renglón)}}
	\justifying
%	Si $A$ es una matriz $m\times n$ y $B$ es la matriz que resulta al reducir $A$ a su forma escalonada reducida, 
%	entonces los renglones no nulos de $B$ forman una base para el espacio renglón de $A$.
	Si una matriz $A$ es equivalente por renglones a una matriz $B$ que está en forma escalonada,
	entonces los renglones no nulos de $B$ forman una base para el espacio renglón de $A$.
\end{prop}	

\end{frame}

%%------------------------------------------------------------------------------------------------------

\subsection{}

\begin{frame}%\frametitle{Base para el espacio renglón de una matriz}

\begin{prop}{\textbf{Propiedad 2 (base para el espacio renglón)}}
	\justifying
	Si una matriz $A$ es equivalente por renglones a una matriz $B$ que está en forma escalonada,
	entonces los renglones no nulos de $B$ forman una base para el espacio renglón de $A$.
\end{prop}	

%\vspace{3mm}

\begin{ej}{\textbf{Ejemplo  5}}
	Halle una base para el \textit{espacio renglón} de la matriz
	\[
	A = 
	\left( 
	\begin{array}{rrrr}	
	1 & -1 & \phantom{-}1 & -1 \\[1mm]
	2 &  0 & 0 & 1 \\[1mm]
	4 & -2 & 2 & 1 \\[1mm]
	7 & -3 & 3 & -1
	\end{array} 
	\right)
	\]
\end{ej}
\textit{Solución}

\end{frame}

%%------------------------------------------------------------------------------------------------------

\subsection{}

\begin{frame}%\frametitle{Base para el espacio renglón de una matriz}

\begin{prop}{\textbf{Propiedad 2 (base para el espacio renglón)}}
	\justifying
	Si una matriz $A$ es equivalente por renglones a una matriz $B$ que está en forma escalonada,
	entonces los renglones no nulos de $B$ forman una base para el espacio renglón de $A$.
\end{prop}	

%\vspace{3mm}

\begin{ej}{\textbf{Ejemplo  6}}
	Halle una base para el subespacio de $\r^3$ generado por
	\[
	S = \Big\{ \, \underbrace{(-1,2,5)}_{\color{blue}\mathbf{v}_1}, \ \underbrace{(3,0,3)}_{\color{blue}\mathbf{v}_2}, \ 
	\underbrace{(5,1,8)}_{\color{blue}\mathbf{v}_3} \, \Big\},
	\]
\end{ej}
\textit{Solución}

\end{frame}

%%------------------------------------------------------------------------------------------------------

\subsection{}

\begin{frame}\frametitle{Base para el espacio columna de una matriz}
	
	\begin{prop}{\textbf{Propiedad 3}}
		\justifying
		Si $A$ es una matriz $m\times n$, entonces $C_A = R_{A^T}$.
	\end{prop}	
	
	%\vspace{3mm}
	
	\begin{alertblock}{\textbf{Observación 2}}
		Para hallar una base del espacio columna de una matriz $A$ podemos aplicar la propiedad 2 a la matriz $A^T$.
	\end{alertblock}
	
	%\vspace{3mm}
	
	
\end{frame}

%%------------------------------------------------------------------------------------------------------

\subsection{}

\begin{frame}%\frametitle{Base para el espacio columna de una matriz}

\begin{prop}{\textbf{Propiedad 3}}
	\justifying
	Si $A$ es una matriz $m\times n$, entonces $C_A = R_{A^T}$.
\end{prop}	

%\vspace{3mm}

\begin{ej}{\textbf{Ejemplo  7}}
	Halle una base para el \textit{espacio columna} de la matriz
	\[
	A = 
	\left( 
	\begin{array}{@{\hspace{0.1\tabcolsep}}rrrr}	
	 1 & \phantom{-}3 & 1 &  3 \\[1mm] 
	 0 & 1 & 1 &  0 \\[1mm] 
	-3 & 0 & 6 & -1 \\[1mm] 
	 3 & 4 & -2 & 1 \\[1mm] 
	 2 & 0 & -4 & -2
	\end{array} 
	\right)
	\]
\end{ej}
\textit{Solución}

\end{frame}

%%------------------------------------------------------------------------------------------------------

\subsection{}

\begin{frame}\frametitle{Base para el espacio columna de una matriz}

\begin{ej}{\textbf{Ejemplo  8}}
	Compare las relaciones de dependencia entre las columnas de la matriz $A$ y 
	las columnas de una matriz escalonada de $A$.
	\[
	A = 
	\left( 
	\begin{array}{@{\hspace{0.1\tabcolsep}}rrrr}	
	1 & \phantom{-}3 & 1 &  3 \\[1mm] 
	0 & 1 & 1 &  0 \\[1mm] 
	-3 & 0 & 6 & -1 \\[1mm] 
	3 & 4 & -2 & 1 \\[1mm] 
	2 & 0 & -4 & -2
	\end{array} 
	\right)
	\]
\end{ej}
\textit{Solución}


%\begin{alertblock}{\textbf{Observación 3}}
%	\begin{enumerate}\justifying
%		\item[\labelname{$a$}] La aplicación de operaciones elementales en los renglones de una matriz
%		\textit{puede} cambiar el \textit{espacio columna}.	
%		\item[\labelname{$b$}] La aplicación de operaciones elementales en los renglones de una matriz
%		\textit{no} cambia las relaciones de dependencia entre las columnas.
%		\item[\labelname{$b$}] Una base para $C_A$ está formada por las columnas de $A$ que correspondan 
%		a las columnas que contienen los 1 pivote en la matriz escalonada.
%	\end{enumerate}
%\end{alertblock}

\end{frame}

%%------------------------------------------------------------------------------------------------------

\subsection{}

\begin{frame}\frametitle{Base para el espacio columna de una matriz}

\begin{alertblock}{\textbf{Observación 3}}
	\begin{enumerate}\justifying
		\item[\labelname{$a$}] La aplicación de operaciones elementales en los renglones de una matriz
		\textit{puede} cambiar el \textit{espacio columna}.	
		\item[\labelname{$b$}] La aplicación de operaciones elementales en los renglones de una matriz
		\textit{no} cambia las relaciones de dependencia entre las columnas.
		\item[\labelname{$b$}] Una base para $C_A$ está formada por las columnas de $A$ que correspondan 
		a las columnas que contienen los 1 pivote en la matriz escalonada.
	\end{enumerate}
\end{alertblock}


\end{frame}

%%------------------------------------------------------------------------------------------------------

\subsection{}

\begin{frame}\frametitle{Base para el espacio columna de una matriz}
		
	\begin{ej}{\textbf{Ejemplo  9}}
		Halle una base para el espacio columna de 
		\[
		A = 
		\left( 
		\begin{array}{@{\hspace{0.1\tabcolsep}}rrrr}	
		1 & \phantom{-}3 & 1 &  3 \\[1mm] 
		0 & 1 & 1 &  0 \\[1mm] 
		-3 & 0 & 6 & -1 \\[1mm] 
		3 & 4 & -2 & 1 \\[1mm] 
		2 & 0 & -4 & -2
		\end{array} 
		\right)
		\]
	\end{ej}
	\textit{Solución}
	
\end{frame}

%%------------------------------------------------------------------------------------------------------

\subsection{}

{\nologo
\begin{frame}%\frametitle{Rango de una matriz}

\begin{prop}{\textbf{Propiedad 4}}
	\justifying
	Si $A$ es una matriz $m\times n$, entonces el espacio renglón de $A$ y el espacio columna de $A$
	tienen la misma dimensión.
\end{prop}	

%\vspace{3mm}

\begin{defi}{\textbf{Definición 2 (rango de una matriz)}}
	La dimensión del espacio renglón (o columna) de una matriz se llama el \textbf{\textit{rango}}
	de $A$ y se denota por $\rho(A)$.
\end{defi}	

%\vspace{3mm}

%\begin{alertblock}{\textbf{Observación 2}}
%	Para hallar una base del espacio columna de una matriz $A$ podemos aplicar la propiedad 2 a la matriz $A^T$.
%\end{alertblock}
%
%\vspace{3mm}

\begin{ej}{\textbf{Ejemplo 10}}
	Halle el rango de la matriz
	\[
	A = 
	\left( 
	\begin{array}{@{\hspace{\tabcolsep}}rrrr}	
	1 & -2 & \phantom{-}0 & 1 \\[1mm] 
	0 &  1 & 5 &  -3 \\[1mm] 
	0 &  1 & 3 & 5
	\end{array} 
	\right)
	\]
\end{ej}
\textit{Solución}

\end{frame}
}