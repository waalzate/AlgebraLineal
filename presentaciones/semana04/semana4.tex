% Este documento LaTeX fue diseado por profesores  del Instituto de Matemtáicas 
% de la Universidad de Antioqua (https://www.matematicasudea.co/). Usted puede modificarlo
% y personalizarlo a su gusto bajo los términos de la licencia de documentacin libre GNU.
% http://es.wikipedia.org/w/index.php?title=Licencia_de_documentaci%C3%B3n_libre_de_GNU&oldid=15717448

%\documentclass[envcountsect,serif,9pt,t]{beamer}

\documentclass[envcountsect,serif,9pt,t]{beamer}

%\usepackage{scrextend}
%\changefontsizes{8pt}

\usepackage{etex}

% \setbeamertemplate{navigation symbols}{}

% \setbeamertemplate{title page}[default][colsep=-4bp,rounded=false,shadow=\beamer@themerounded@shadow]

% \setbeamertemplate{blocks}[rounded][shadow=false]

% \setbeamertemplate{title page}[default][colsep=-4bp,rounded=true]


% \setbeamertemplate{frametitle}[default][colsep=-4bp,rounded=false,shadow=false]

\usetheme{Frankfurt}

\usepackage{tcolorbox}

\usepackage{lipsum}

% \hypersetup{pdfpagemode=FullScreen}

% \usecolortheme[rgb={0,0.39,0}]{structure}

%\documentclass[serif,9pt]{beamer}
%\usepackage[T1]{fontenc} % Needed for Type1 Concrete
%\usepackage[charter]{mathdesign}

%\usepackage[pdftex]{graphicx}

% https://www.overleaf.com/learn/latex/Spanish
\usepackage[utf8]{inputenc}
\usepackage[spanish, es-tabla]{babel}

\usepackage{amsmath,amssymb}
%\usepackage[dvips]{graphicx}
\usepackage{multicol}
\usepackage{textcomp}
\usepackage[makeroom]{cancel}
\usepackage{multicol}
\usepackage{colortbl}
\usepackage{pstricks,pst-plot}

\usepackage{eso-pic}

%\usepackage{subfig}
\usepackage{xcolor}
%\usepackage{float}

\usepackage{graphicx}

\usepackage{tikz,tkz-euclide,tikz-3dplot}
% \usetikzlibrary{shadows,decorations.pathmorphing,shapes,arrows,patterns,matrix,positioning,plotmarks,shapes.geometric,shapes.arrows,arrows.meta}
\tikzset{>=latex}
%\usetkzobj{all}

\usepackage{gensymb}

\usepackage{cancel}

% Para las tablas nice
% \usepackage{tikz}
% \usepackage{pgfplots}
% \usetikzlibrary{shadows}

% Para gráficas
\usepackage{graphicx}
\usepackage{caption}
\usepackage{subcaption}

% Para justificar texto
\usepackage{ragged2e}
% \justifying

% https://ctan.math.illinois.edu/macros/latex/contrib/longfbox/longfbox.html
\usepackage{longfbox}

\decimalpoint

\definecolor{azul}{RGB}{0,92,162}

\setbeamercolor{structure}{fg=azul}

%%%%%%%%%%%%%%%%%%%%%%%%%%%%% Definiciones %%%%%%%%%%%%%%%%%%%%%%%%%%%%%%%%%

% https://tex.stackexchange.com/questions/53781/how-can-i-include-the-logo-in-some-slides-and-remove-in-others-using-beamer/53783
\newcommand{\nologo}{\setbeamertemplate{logo}{}} % command to set the logo to nothing

% https://tex.stackexchange.com/questions/123924/indexed-letters-inside-circles
\newcommand\encircle[1]{%
	\tikz[baseline=(X.base)] 
	\node (X) [draw, shape=circle, inner sep=0] {\strut #1};}

% http://tex.stackexchange.com/questions/161301/latex-beamer-enumerate-with-manual-numbers
\newcommand{\labelname}[1]{
	\def\insertenumlabel{#1}%
	\usebeamertemplate{enumerate item}%
}

\newenvironment<>{defi}[1]{%
	\begin{actionenv}#2%
		\def\insertblocktitle{#1}%
		\par%
		\mode<presentation>{}
		\usebeamertemplate{block begin}}
	{\par\usebeamertemplate{block end}\end{actionenv}
}

\newenvironment<>{prop}[1]{%
	\begin{actionenv}#2%
		\def\insertblocktitle{#1}%
		\par%
		\mode<presentation>{%
			\setbeamercolor{block title}{fg=white,bg=green!45!black}
			\setbeamercolor{block body}{fg=black,bg=green!4} 
			%        \setbeamercolor{itemize item}{fg=orange!20!black}crist
			%\setbeamertemplate{itemize item}[triangle]
			\setbeamercolor{item projected}{fg=white,bg=green!60!black}
			%        \setbeamercolor{item projected}{fg=white,bg=orange!100}
		}%
		\usebeamertemplate{block begin}}
	{\par\usebeamertemplate{block end}\end{actionenv}
}

\newenvironment<>{obs}[1]{%
	\begin{actionenv}#2%
		\def\insertblocktitle{#1}%
		\par%
		\mode<presentation>{%
			\setbeamercolor{block title}{fg=white,bg=orange!90!black}
			\setbeamercolor{block body}{fg=black,bg=orange!5}
			%        \setbeamercolor{itemize item}{fg=orange!20!black}
			%\setbeamertemplate{itemize item}[triangle]
			\setbeamercolor{item projected}{fg=white,bg=orange!80!black}
			%        \setbeamercolor{item projected}{fg=white,bg=orange!100}
		}%
		\usebeamertemplate{block begin}}
	{\par\usebeamertemplate{block end}\end{actionenv}
}

\newenvironment<>{ejer}[1]{%
	\begin{actionenv}#2%
		\def\insertblocktitle{#1}%
		\par%
		\mode<presentation>{%
			\setbeamercolor{block title}{fg=white,bg=purpura!40!black}
			\setbeamercolor{block body}{fg=black,bg=columbia!5}
			%        \setbeamercolor{itemize item}{fg=orange!20!black}
			%\setbeamertemplate{itemize item}[triangle]
			\setbeamercolor{item projected}{fg=white,bg=purpura!40!black}
		}%
		\usebeamertemplate{block begin}}
	{\par\usebeamertemplate{block end}\end{actionenv}
}

\newenvironment<>{ej}[1]{%
	\begin{actionenv}#2%
		\def\insertblocktitle{#1}%
		\par%
		\mode<presentation>{%
			\setbeamercolor{block title}{fg=white,bg=orange!90!black}
			\setbeamercolor{block body}{fg=black,bg=orange!5}
			%        \setbeamercolor{itemize item}{fg=orange!20!black}
			%\setbeamertemplate{itemize item}[triangle]
			\setbeamercolor{item projected}{fg=white,bg=orange!80!black}
			%        \setbeamercolor{item projected}{fg=white,bg=orange!100}
		}%
		\usebeamertemplate{block begin}}
	{\par\usebeamertemplate{block end}\end{actionenv}
}

\newenvironment<>{ejem}[1]{%
	\begin{actionenv}#2%
		\def\insertblocktitle{#1}%
		\par%
		\mode<presentation>{%
			\setbeamercolor{block title}{fg=white,bg=red!40!black}
			\setbeamercolor{block body}{fg=black,bg=columbia!5}
			%        \setbeamercolor{itemize item}{fg=orange!20!black}
			%\setbeamertemplate{itemize item}[triangle]
			\setbeamercolor{item projected}{fg=white,bg=purpura!40!black}
		}%
		\usebeamertemplate{block begin}}
	{\par\usebeamertemplate{block end}\end{actionenv}
}

\newenvironment<>{problock}[1]{%
	\begin{actionenv}#2%
		\def\insertblocktitle{#1}%
		\par%
		\mode<presentation>{%
			\setbeamercolor{block title}{fg=white,bg=orange!50!black}
			\setbeamercolor{block body}{fg=black,bg=olive!30}
			\setbeamercolor{itemize item}{fg=orange!20!black}
			\setbeamertemplate{itemize item}[triangle]
			\setbeamercolor{item projected}{fg=white,bg=orange!20!black}
		}%
		\usebeamertemplate{block begin}}
	{\par\usebeamertemplate{block end}\end{actionenv}
}

% \definecolor{darkred}{rgb}{0.8,0,0}
% 
% \newenvironment<>{ejer}[1]{%
%   \begin{actionenv}#2%
%       \def\insertblocktitle{#1}%
%       \par%
%       \mode<presentation>{%
%        \setbeamercolor{block title}{fg=white,bg=orange!20!black}
% %        \setbeamercolor{block body}{fg=black,bg=olive!50}
%        \setbeamercolor{block body}{fg=black,bg=darkred}
%        \setbeamercolor{itemize item}{fg=orange!20!black}
%        \setbeamertemplate{itemize item}[triangle]
%      }%
%       \usebeamertemplate{block begin}}
%     {\par\usebeamertemplate{block end}\end{actionenv}
% }

% http://tex.stackexchange.com/questions/66655/numbered-definitions-theorems-in-beamer
% \setbeamertemplate{theorems}[numbered]
\setbeamertemplate{theorems}[ams style] 
\setbeamertemplate{caption}[numbered]

% \newtheorem{teo}{Teorema}[section]
% \newtheorem{lema}{Lema}[section]
% \newtheorem{pro}{Proposición}[section]
% \newtheorem{defi}{Definici\'on}[section]
% \newtheorem{ej}{Ejemplo}[section]
% \newtheorem{obs}{Observación}[section]

%\newtheorem{}{Teorema}[section]
%\newtheorem{proposition}[theorem]{Proposición}

\def\R{\mathbb{R}}
\def\r{\mathbb{R}}
\def\rn{\mathbb{R}^n}
\def\u{\mathbf{u}}
\def\u{\mathbf{u}}
\def\v{\mathbf{v}}
\def\w{\mathbf{w}}
\def\x{\mathbf{x}}
\def\y{\mathbf{y}}
\def\p{\mathbb{P}}
\def\f{\mathbb{F}}
\def\N{\mathbb{N}}
\def\n{\mathbb{N}}
\def\q{\mathbb{Q}}
\def\Q{\mathbb{Q}}
\def\c{\mathbb{C}}
\def\C{\mathbb{C}}
\def\z{\mathbb{Z}}
\def\Z{\mathbb{Z}}

\def\sen{\mathop{\mbox{\normalfont sen}}\nolimits}
\def\sign{\mathop{\mbox{\normalfont sign}}\nolimits}
\def\intt{\mathop{\mbox{\normalfont int}}\nolimits}
\def\diag{\mathop{\mbox{\normalfont diag}}\nolimits}
\def\arcsen{\mathop{\mbox{\normalfont arcsen}}\nolimits}
\def\ln{\mathop{\mbox{\normalfont ln}}\nolimits}
\def\tr{\mathop{\mbox{\normalfont tr}}\nolimits}
\def\erf{\mathop{\mbox{\normalfont erf}}\nolimits}
\def\erfc{\mathop{\mbox{\normalfont erfc}}\nolimits}

\def\op{\mathop{\mbox{\normalfont op}}\nolimits}
\def\ady{\mathop{\mbox{\normalfont ady}}\nolimits}
\def\hip{\mathop{\mbox{\normalfont hip}}\nolimits}

% \def\.{\mathop{\mbox{\normalfont .}}\nolimits}
\def\rem{\mathop{\mbox{\normalfont rem}}\nolimits}
\def\divi{\mathop{\mbox{\normalfont div}}\nolimits}
\def\modd{\mathop{\mbox{\normalfont mod}}\nolimits}
\def\oo{\mathop{\mbox{ \normalfont O }}\nolimits}
\def\yy{\mathop{\mbox{ \normalfont Y }}\nolimits}
\def\no{\mathop{\mbox{\normalfont NO }}\nolimits}

%%%%%%%%%%%%%%%%%%%%%%%%%%%%%%%%%%%%%%%%%%%%%%%%%%%%%%%%%%%%%%%%%%%%%%%

\newpsobject{grilla}{psgrid}{subgriddiv=1,griddots=10,gridlabels=6pt}
\newpsobject{grillaA}{psgrid}{subgriddiv=1,griddots=6,gridlabels=0pt,gridcolor=grilla-gray}
\newpsobject{grillaB}{psgrid}{subgriddiv=1,griddots=10,gridlabels=0pt,gridcolor=grilla-gray}

%%%%%%%%%%%%%%%%%%%%%%%%%%%%%%%%%%%%%%%%%%%%%%%%%%%%%%%%%%%%%%%%%%%%%%%%%%%%


\newcommand{\myblue}{\only{\color{blue}}}
\newcommand{\myred}{\only{\color{red}}}
\newcommand{\mypurpura}{\only{\color{purpura}}}
\newcommand{\myorange}{\only{\color{orange}}}
\newcommand{\mymagenta}{\only{\color{magenta}}}
\newcommand{\mycyan}{\only{\color{cyan}}}
\newcommand{\mygreen}{\only{\color{olive}}}
\newcommand{\myyellow}{\only{\color{yellow}}}
\newcommand{\mygray}{\only{\color{gray}}}
\def\colorize<#1>{\temporal<#1>{\color{red!50}}{\color{black}}{\color{black!50}}}

%========================================================

\xdefinecolor{dorado}{cmyk}{0,0.10,0.84,0}
\xdefinecolor{melon}{cmyk}{0,0.29,0.84,0}
\xdefinecolor{naranja}{cmyk}{0,0.42,1,0}
\xdefinecolor{durazno}{cmyk}{0,0.46,0.50,0}
\xdefinecolor{fresa}{cmyk}{0,1,0.50,0}
\xdefinecolor{ladrillo}{cmyk}{0,0.77,0.87,0}
\xdefinecolor{violeta}{cmyk}{0.07,0.90,0,0.34}
\xdefinecolor{columbia}{cmyk}{0.39, 0.13, 0.00, 0.00}
\xdefinecolor{purpura}{cmyk}{0.45,0.86,0,0}
\xdefinecolor{aguamarina}{cmyk}{0.85,0,0.33,0}
\xdefinecolor{esmeralda}{cmyk}{0.91,0,0.88,0.12}
\xdefinecolor{pino}{cmyk}{0.92,0,0.59,0.25}
\xdefinecolor{oliva}{cmyk}{0.64,0,0.95,0.40}
\xdefinecolor{canela}{cmyk}{0.14,0.42,0.56,0}
\xdefinecolor{marron}{cmyk}{0,0.72,1,0.45}
\xdefinecolor{cafe}{cmyk}{0,0.81,1,0.60}
\xdefinecolor{gris-claro}{cmyk}{0,0,0,0.30}
\xdefinecolor{gris-oscuro}{cmyk}{0,0,0,0.50}
\definecolor{darkgreen}{rgb}{0,0.39,0}
\xdefinecolor{darkblue}{rgb}{0,0,0.55}

\definecolor{verde}{HTML}{006400} % ver8de

\definecolor{light-gray}{gray}{0.94}
\definecolor{gris}{gray}{0.92}
\definecolor{grilla-gray}{gray}{0.77}

%========================================================

\def\octave{\mathop{\mbox{\normalfont \hspace{.1em}\texttt{Octave} \hspace{-.3em}\raisebox{.4ex}{}}}\nolimits}
\def\goctave{\mathop{\mbox{\normalfont \hspace{.1em}\texttt{GNU Octave} \hspace{-.3em}\raisebox{.4ex}{}}}\nolimits}

\beamersetuncovermixins{\opaqueness<1>{10}}{\opaqueness<2->{10}}

\begin{document}

\title{Álgebra lineal -- Semana 4 \\[2mm] Espacios fundamentales de una matriz}
\author{
	% \href{mailto:alejandro.piedrahita@udea.edu.co}{Alejandro Piedrahita H.} \\[6mm]
	\texorpdfstring{Grupo EMAC\newline{\footnotesize \url{grupoemac@udea.edu.co} } }{} \\[6mm]
	Facultad de Ciencias Exactas y Naturales \\[0mm] %\vspace{0.03cm}
	\href{https://www.matematicasudea.co/}{\color{blue}Instituto de Matemáticas} \\[0mm] %\vspace{0.09cm}
	Universidad de Antioquia}

\date{\today}
\logo{\fbox{\includegraphics[scale=0.12]{imagenes/udea}}} 
%\logo{\fbox{\includegraphics[width=2cm, height=1.4cm]{imagenes/profe}}\phantom{xx} } 

\begin{frame}
 \titlepage
\end{frame}

% Remueve el logo de la página de inicio:
% https://tex.stackexchange.com/questions/56966/prevent-logo-from-title-page-in-beamer-class?rq=1
%\frame[plain]{\titlepage}

%-----------------------------------------------------------------------------------------------------------------


% \begin{frame}
% \frametitle{Contenido}\tableofcontents
% \end{frame} 

%-----------------------------------------------------------------------------------------------------------------

\section{Conjuntos generadores}

\subsection{}

\begin{frame}\frametitle{Combinación lineal}

\begin{block}{\textbf{Definición 1 (Combinación lineal)}}
	\justifying
	Un vector $\mathbf{v}$ en un espacio vectorial $V$ se dice que es \textbf{\textit{combinación lineal}} 
	de los vectores
	\[
		\mathbf{v_1}, \mathbf{v_2}, \hdots, \mathbf{v_k}
	\]
	de $V$, si existen escalares $c_1,c_2, \hdots,c_k$ tales que
	\[
		\mathbf{v} = c_1\mathbf{v_1} + c_2\mathbf{v_2} + \hdots + c_k\mathbf{v_k}.
	\]
\end{block}

%\vspace{5mm}
%
%\begin{ej}{\textbf{Ejemplo 1 }} \justifying
%	Muestre que en el conjunto de vectores de $\r^3$,
%	\[
%		S = \{ \, \underbrace{(1,3,1)}_{\color{blue}\mathbf{v_1}}, \ \underbrace{(0,1,2)}_{\color{blue}\mathbf{v_2}}, \ 
%		       \underbrace{(1,0,-5)}_{\color{blue}\mathbf{v_3}} \, \},
%	\]
%	el vector $\mathbf{v_1}$ es \textbf{\textit{combinación lineal}} de los vectores $\mathbf{v_2}$ y $\mathbf{v_3}$.
%\end{ej}	


\end{frame}

%%------------------------------------------------------------------------------------------------------

\subsection{}

\begin{frame}\frametitle{Combinación lineal}

\begin{block}{\textbf{Definición 1 (Combinación lineal)}}
	\justifying
	Un vector $\mathbf{v}$ en un espacio vectorial $V$ se dice que es \textbf{\textit{combinación lineal}} 
	de los vectores $\mathbf{v_1}, \mathbf{v_2}, \hdots, \mathbf{v_k}$ de $V$ si existen escalares $c_1,c_2, \hdots,c_k$ tales que
	
	\vspace{-2mm}
	\[
	\mathbf{v} = c_1\mathbf{v_1} + c_2\mathbf{v_2} + \hdots + c_k\mathbf{v_k}
	\]
\end{block}

\begin{ej}{\textbf{Ejemplo 1 }} \justifying
	Muestre que en el conjunto de vectores de $\r^3$,
	\[
	S = \{ \, \underbrace{(1,3,1)}_{\color{blue}\mathbf{v_1}}, \ \underbrace{(0,1,2)}_{\color{blue}\mathbf{v_2}}, \ 
	\underbrace{(1,0,-5)}_{\color{blue}\mathbf{v_3}} \, \},
	\]
	el vector $\mathbf{v_1}$ es \textbf{\textit{combinación lineal}} de los vectores $\mathbf{v_2}$ y $\mathbf{v_3}$.
\end{ej}	

\textit{Solución.}

\end{frame}

%%------------------------------------------------------------------------------------------------------

\subsection{}

\begin{frame}\frametitle{Combinación lineal}

\begin{ej}{\textbf{Ejemplo 2}} \justifying
	Muestre que en el conjunto de vectores de $M_{22}$,
	\[
	S = \Bigg\{ \, \underbrace{ \left( \begin{array}{cc}	0 & 8 \\ 2 & 1 \end{array} \right) }_{\color{blue}\mathbf{v_1}}, \ \underbrace{  \left( \begin{array}{cc}	0 & 2 \\ 1 & 0 \end{array} \right) }_{\color{blue}\mathbf{v_2}}, \ 
	\underbrace{ \left( \begin{array}{rr}  -1 & 3 \\ 1 & 2 \end{array} \right) }_{\color{blue}\mathbf{v_3}}, \ 
	\underbrace{ \left( \begin{array}{rr}  -2 & 0 \\ 1 & 3 \end{array} \right) }_{\color{blue}\mathbf{v_4}} \, \Bigg\},
	\]
	el vector $\mathbf{v_1}$ es \textbf{\textit{combinación lineal}} de los vectores $\mathbf{v_2},\mathbf{v_3}$ y $\mathbf{v_4}$.
\end{ej}	

\textit{Solución.}

\end{frame}

%%------------------------------------------------------------------------------------------------------

\subsection{}

\begin{frame}\frametitle{Combinación lineal}

\begin{ej}{\textbf{Ejemplo 3}} \justifying
	En $V=\r^3$, escriba al vector $\mathbf{v}=(1,1,1)$ como combinación lineal de los vectores en el conjunto
	\[
		S = \{ \, \underbrace{(1,2,3)}_{\color{blue}\mathbf{v_1}}, \ \underbrace{(0,1,2)}_{\color{blue}\mathbf{v_2}}, \ 
		\underbrace{(-1,0,1)}_{\color{blue}\mathbf{v_3}} \, \}.
	\]
\end{ej}	

\textit{Solución.}

\end{frame}

%%------------------------------------------------------------------------------------------------------

\subsection{}

\begin{frame}\frametitle{Combinación lineal}

\begin{ej}{\textbf{Ejemplo 4}} \justifying
	En $V=\r^3$, escriba al vector $\mathbf{w}=(1,-2,2)$ como combinación lineal de los vectores en el conjunto
	\[
	S = \{ \, \underbrace{(1,2,3)}_{\color{blue}\mathbf{v_1}}, \ \underbrace{(0,1,2)}_{\color{blue}\mathbf{v_2}}, \ 
	\underbrace{(-1,0,1)}_{\color{blue}\mathbf{v_3}} \, \}.
	\]
\end{ej}	

\textit{Solución.}

\end{frame}

%------------------------------------------------------------------------------------------------------

\subsection{}

\begin{frame}\frametitle{Conjuntos generadores}

\begin{block}{\textbf{Definición 2 (Conjunto generador)}}
	\justifying
	Se dice que un conjunto de vectores $S=\{\mathbf{v}_1,\mathbf{v}_2,\hdots,\mathbf{v}_k\}$ en un espacio 
	vectorial $V$ \textbf{\textit{genera}} a $V$ si todo vector en $V$ se puede escribir como combinación lineal
	de los vectores en $S$. Es decir, para todo vector $\mathbf{v}\in V$, existen escalares $c_1,c_2,\hdots,c_k$
	tales que 
	
	\vspace{-3mm}
	\[
		\mathbf{v} = c_1\mathbf{v}_1 + c_2\mathbf{v}_2 + \cdots + c_k\mathbf{v}_k
	\]
	
\end{block}

\vspace{0mm}

\begin{ej}{\textbf{Ejemplo 5}} \justifying
	El conjunto de vectores
	
	\vspace{-2mm}
	\[
	S = \{ \, \underbrace{(1,0,0)}_{\mathbf{i}}, \ \underbrace{(0,1,0)}_{\mathbf{j}}, \ 
	\underbrace{(0,0,1)}_{\mathbf{k}} \, \}.
	\]
	
	\vspace{-3mm}
	\textbf{\textit{genera}} a $\r^3$.
\end{ej}

\textit{Solución.}	
	
\end{frame}

%------------------------------------------------------------------------------------------------------

\subsection{}

\begin{frame}\frametitle{Conjuntos generadores de polinomios}

\begin{block}{\textbf{Definición 2 (Conjunto generador)}}
	\justifying
	Se dice que un conjunto de vectores $S=\{\mathbf{v}_1,\mathbf{v}_2,\hdots,\mathbf{v}_k\}$ en un espacio 
	vectorial $V$ \textbf{\textit{genera}} a $V$ si todo vector en $V$ se puede escribir como combinación lineal
	de los vectores en $S$. Es decir, para todo vector $\mathbf{v}\in V$, existen escalares $c_1,c_2,\hdots,c_k$
	tales que 
	
	\vspace{-3mm}
	\[
	\mathbf{v} = c_1\mathbf{v}_1 + c_2\mathbf{v}_2 + \cdots + c_k\mathbf{v}_k
	\]
	
\end{block}

\vspace{0mm}

\begin{ej}{\textbf{Ejemplo 6}} \justifying
	El conjunto de vectores
	\[
	S = \left\{ \, 1, x, x^2 \right\}
	\]
	\textbf{\textit{genera}} a $P_2$.
\end{ej}	

\textit{Solución}.

\end{frame}

%------------------------------------------------------------------------------------------------------

\subsection{}

\begin{frame}\frametitle{Conjuntos generadores de polinomios}

\begin{block}{\textbf{Definición 2 (Conjunto generador)}}
	\justifying
	Se dice que un conjunto de vectores $S=\{\mathbf{v}_1,\mathbf{v}_2,\hdots,\mathbf{v}_k\}$ en un espacio 
	vectorial $V$ \textbf{\textit{genera}} a $V$ si todo vector en $V$ se puede escribir como combinación lineal
	de los vectores en $S$. Es decir, para todo vector $\mathbf{v}\in V$, existen escalares $c_1,c_2,\hdots,c_k$
	tales que 
	
	\vspace{-3mm}
	\[
	\mathbf{v} = c_1\mathbf{v}_1 + c_2\mathbf{v}_2 + \cdots + c_k\mathbf{v}_k
	\]
	
\end{block}

\vspace{0mm}

\begin{ej}{\textbf{Ejemplo 7}} \justifying
	Ningún conjunto finito de polinomios genera a $P$.
\end{ej}	

\textit{Solución}.

\end{frame}

%------------------------------------------------------------------------------------------------------

\subsection{}

\begin{frame}\frametitle{Conjuntos generador de $\r^3$}

\begin{ej}{\textbf{Ejemplo 8}} \justifying
	El conjunto de vectores
	\[
		S = \{ \, \underbrace{(1,2,3)}_{\color{blue}\mathbf{v_1}}, \ \underbrace{(0,1,2)}_{\color{blue}\mathbf{v_2}}, \ 
		\underbrace{(-2,0,1)}_{\color{blue}\mathbf{v_3}} \, \},
	\]
	\textbf{\textit{genera}} a $\r^3$.
\end{ej}	

\textit{Solución}.

\end{frame}

%------------------------------------------------------------------------------------------------------

\subsection{}

\begin{frame}\frametitle{Conjuntos generador de $M_{22}$}

\begin{ej}{\textbf{Ejemplo 9}} \justifying
	El conjunto de vectores
	\[
		S = \Bigg\{ \, \underbrace{ \left( \begin{array}{cc} 1 & 0 \\ 0 & 0 \end{array} \right) }_{\color{blue}\mathbf{v_1}}, \ \underbrace{  \left( \begin{array}{cc}	0 & 1 \\ 0 & 0 \end{array} \right) }_{\color{blue}\mathbf{v_2}}, \ 
		\underbrace{ \left( \begin{array}{rr}   0 & 0 \\ 1 & 0 \end{array} \right) }_{\color{blue}\mathbf{v_3}}, \ 
		\underbrace{ \left( \begin{array}{rr}   0 & 0 \\ 0 & 1 \end{array} \right) }_{\color{blue}\mathbf{v_4}} \, \Bigg\},
	\]
	\textbf{\textit{genera}} a $M_{22}$.
\end{ej}	

\textit{Solución}.

\end{frame}

%------------------------------------------------------------------------------------------------------

\subsection{}

{\nologo
\begin{frame}\frametitle{Espacio generado por un conjunto de vectores}

\begin{block}{\textbf{Definición 3 (Espacio generado)}}
	\justifying
	El \textbf{\textit{espacio generado}} por un conjunto de vectores 
	\[
	S=\{\mathbf{v}_1,\mathbf{v}_2,\hdots,\mathbf{v}_k\}
	\]
	en un espacio vectorial $V$, se define como el conjunto de TODAS las combinaciones lineales de $S$.
	Al \textbf{\textit{espacio generado}} por $S$ se le denota por 
	\[
	\text{gen}\,(S) \qquad \text{ó} \qquad \text{gen}\, \{\mathbf{v}_1,\mathbf{v}_2,\hdots,\mathbf{v}_k\}.
	\]
	\[
	\text{gen}\,(S) = \Big\{ \, c_1\mathbf{v}_1 + c_2\mathbf{v}_2 + \cdots + c_k\mathbf{v}_k \mid c_1, c_2,\hdots,c_k  
	\text{ son números reales } \, \Big\}.
	\]
\end{block}

\begin{ej}{\textbf{Ejemplo 10}} \justifying
	Halle el espacio generado por el vector $\mathbf{v}=(2,1)$ de $\r^2$.
\end{ej}	

\textit{Solución}.

\end{frame}
}

%------------------------------------------------------------------------------------------------------

\subsection{}

{\nologo
\begin{frame}\frametitle{Espacio generado por un conjunto de vectores}

\begin{block}{\textbf{Definición 3 (Espacio generado)}}
	\justifying
	El \textbf{\textit{espacio generado}} por un conjunto de vectores 
	\[
	S=\{\mathbf{v}_1,\mathbf{v}_2,\hdots,\mathbf{v}_k\}
	\]
	en un espacio vectorial $V$, se define como el conjunto de TODAS las combinaciones lineales de $S$.
	Al \textbf{\textit{espacio generado}} por $S$ se le denota por 
	\[
	\text{gen}\,(S) \qquad \text{ó} \qquad \text{gen}\, \{\mathbf{v}_1,\mathbf{v}_2,\hdots,\mathbf{v}_k\}.
	\]
	\[
	\text{gen}\,(S) = \Big\{ \, c_1\mathbf{v}_1 + c_2\mathbf{v}_2 + \cdots + c_k\mathbf{v}_k \mid c_1, c_2,\hdots,c_k  
	\text{ son números reales } \, \Big\}.
	\]
\end{block}

\begin{prop}{\textbf{Propiedad 1}}
	Si $\mathbf{v}_1,\mathbf{v}_2,\hdots,\mathbf{v}_k$ son vectores en un espacio vectorial $V$, entonces 
	\[
		\text{gen}\, \{\mathbf{v}_1,\mathbf{v}_2,\hdots,\mathbf{v}_k\}
	\]
	es un subespacio vectorial de $V$.
\end{prop}	

\end{frame}
}

%------------------------------------------------------------------------------------------------------

\subsection{}

{\nologo
\begin{frame}\frametitle{Espacio generado por un conjunto de vectores}

\vspace{-2mm}	

	\begin{block}{\textbf{Definición 3 (Espacio generado)}}
		\justifying
		El \textbf{\textit{espacio generado}} por un conjunto de vectores 
		\[
		S=\{\mathbf{v}_1,\mathbf{v}_2,\hdots,\mathbf{v}_k\}
		\]
		en un espacio vectorial $V$, se define como el conjunto de TODAS las combinaciones lineales de $S$.
		Al \textbf{\textit{espacio generado}} por $S$ se le denota por 
		\[
		\text{gen}\,(S) \qquad \text{ó} \qquad \text{gen}\, \{\mathbf{v}_1,\mathbf{v}_2,\hdots,\mathbf{v}_k\}.
		\]
		\[
		\text{gen}\,(S) = \Big\{ \, c_1\mathbf{v}_1 + c_2\mathbf{v}_2 + \cdots + c_k\mathbf{v}_k \mid c_1, c_2,\hdots,c_k  
		\text{ son números reales } \, \Big\}.
		\]
	\end{block}
	
	\begin{ej}{\textbf{Ejemplo 11}} \justifying
		Demuestre que si $\mathbf{v}_1,\mathbf{v}_2,\hdots,\mathbf{v}_m$ son vectores que generan un espacio vectorial $V$, entonces para todo vector $\mathbf{w}$ en $V$, los vectores
		$\mathbf{w}, \mathbf{v}_1,\mathbf{v}_2,\hdots,\mathbf{v}_m$  también generan a $V$.
	\end{ej}
	\textit{Solución.}
	
\end{frame}
}

%------------------------------------------------------------------------------------------------------

\subsection{}

{\nologo
\begin{frame}\frametitle{Espacio generado por un conjunto de vectores}
	
	\vspace{-2mm}
	\begin{block}{\textbf{Definición 3 (Espacio generado)}}
		\justifying
		El \textbf{\textit{espacio generado}} por un conjunto de vectores 
		\[
		S=\{\mathbf{v}_1,\mathbf{v}_2,\hdots,\mathbf{v}_k\}
		\]
		en un espacio vectorial $V$, se define como el conjunto de TODAS las combinaciones lineales de $S$.
		Al \textbf{\textit{espacio generado}} por $S$ se le denota por 
		\[
		\text{gen}\,(S) \qquad \text{ó} \qquad \text{gen}\, \{\mathbf{v}_1,\mathbf{v}_2,\hdots,\mathbf{v}_k\}.
		\]
		\[
		\text{gen}\,(S) = \Big\{ \, c_1\mathbf{v}_1 + c_2\mathbf{v}_2 + \cdots + c_k\mathbf{v}_k \mid c_1, c_2,\hdots,c_k  
		\text{ son números reales } \, \Big\}.
		\]
	\end{block}
	
	\begin{ej}{\textbf{Ejemplo 12}} \justifying
		Demuestre que si $\mathbf{v}_1,\mathbf{v}_2,\hdots,\mathbf{v}_m$ son vectores que generan un espacio vectorial $V$ y que si uno de los vectores $\mathbf{v}_k$ es combinación lineal del resto, entonces los vectores $\mathbf{v}_1,\hdots,\mathbf{v}_m$ sin el vector $\mathbf{v}_k$ también generan a $V$.
	\end{ej}
	\textit{Solución.}
	
\end{frame}
}

%------------------------------------------------------------------------------------------------------

\subsection{}

{\nologo
\begin{frame}\frametitle{Espacio generado por un conjunto de vectores}
	
	\vspace{-2mm}
	\begin{block}{\textbf{Definición 3 (Espacio generado)}}
		\justifying
		El \textbf{\textit{espacio generado}} por un conjunto de vectores 
		\[
		S=\{\mathbf{v}_1,\mathbf{v}_2,\hdots,\mathbf{v}_k\}
		\]
		en un espacio vectorial $V$, se define como el conjunto de TODAS las combinaciones lineales de $S$.
		Al \textbf{\textit{espacio generado}} por $S$ se le denota por 
		\[
		\text{gen}\,(S) \qquad \text{ó} \qquad \text{gen}\, \{\mathbf{v}_1,\mathbf{v}_2,\hdots,\mathbf{v}_k\}.
		\]
		\[
		\text{gen}\,(S) = \Big\{ \, c_1\mathbf{v}_1 + c_2\mathbf{v}_2 + \cdots + c_k\mathbf{v}_k \mid c_1, c_2,\hdots,c_k  
		\text{ son números reales } \, \Big\}.
		\]
	\end{block}
	
	\begin{ej}{\textbf{Ejemplo 13}} \justifying
		Demuestre que si $\mathbf{v}_1,\mathbf{v}_2,\hdots,\mathbf{v}_m$ son vectores que generan un espacio vectorial $V$ y que si uno de los vectores $\mathbf{v}_k$ es el vector cero, entonces los vectores $\mathbf{v}_1,\hdots,\mathbf{v}_m$ sin el vector cero también generan a $V$.
	\end{ej}
	\textit{Solución.}
	
\end{frame}
}


\section{Espacios vectoriales}

\subsection{}

{\nologo
\begin{frame}\frametitle{Espacios vectoriales (reales)}

\vspace{-3mm}
%\begin{exampleblock}{\textbf{Definición 2}}
\begin{block}{\textbf{Definición 5 (Espacio vectorial)}}	
	\justifying
	Sea $V$ un conjunto (no vacío) en el que están definidas dos operaciones (\textbf{suma de vectores} y 
	\textbf{multiplicación por escalar}). Se dice que $V$ es un {\color{red} espacio vectorial (real)} si para todo
	$\mathbf{u}, \mathbf{v}$ y $\mathbf{w}$ en $V$ y todo escalar (número real) $c$ y $d$ en $\r$, se 
	cumplen las siguientes propiedades: 
	\begin{multicols}{2}		
		\begin{enumerate}			
			\justifying
			\item $\mathbf{u}+\mathbf{v}$ está en $V$. \\[4mm]			
			\item $\mathbf{u}+\mathbf{v} = \mathbf{v}+\mathbf{u}$. \\[3mm]			
			\item $(\mathbf{u}+\mathbf{v})+\mathbf{w} = \mathbf{u}+(\mathbf{v}+\mathbf{w})$. \\[4mm]			
			\item Existe en $V$ un vector cero $\mathbf{0}$ tal que
			\[
			\mathbf{u}+\mathbf{0} = \mathbf{u}.
			\]
			
			\vspace{2mm}	
			\item Para cada $\mathbf{u}$, existe en $V$ un vector denotado por $-\mathbf{u}$ tal que
			\[
			\mathbf{u}+(-\mathbf{u}) = \mathbf{0}.
			\]	
			\columnbreak
			\item $c\mathbf{u}$ está en $V$. \\[4mm]			
			\item $c(\mathbf{u}+\mathbf{v}) = c\mathbf{u} + c\mathbf{v}$. \\[3mm]
			\item $(c+d)\mathbf{u} = c\mathbf{u} + d\mathbf{u}$. \\[4mm]
			\item $c(d\mathbf{u}) = (cd)\mathbf{u}$. \\[1.1cm]
			\item $1\mathbf{u} = \mathbf{u}$.
		\end{enumerate}		
	\end{multicols}
	
	\vspace{-8mm}
\end{block}

\end{frame}
}

%%------------------------------------------------------------------------------------------------------

\subsection{}

{\nologo
\begin{frame}\frametitle{$\rn$ con las operaciones estándar es un espacio vectorial}

%\begin{exampleblock}{\textbf{Definición 2}}
\begin{block}{\textbf{Axiomas de un espacio vectorial (real) $V$}}	
	\justifying
	Para todo $\mathbf{u}, \mathbf{v}$ y $\mathbf{w}$ en $V$ y todo escalar $c$ y $d$ en $\r$, se 
	cumplen las siguientes propiedades: 
	
	\vspace{-3mm}
	\begin{multicols}{2}		
		\begin{enumerate}			
			\justifying
			\item $\mathbf{u}+\mathbf{v}$ está en $V$. %\\[4mm]			
			\item $\mathbf{u}+\mathbf{v} = \mathbf{v}+\mathbf{u}$. %\\[3mm]			
			\item $(\mathbf{u}+\mathbf{v})+\mathbf{w} = \mathbf{u}+(\mathbf{v}+\mathbf{w})$. %\\[4mm]			
			\item Existe en $V$ un vector cero $\mathbf{0}$ tal que $\mathbf{u}+\mathbf{0} = \mathbf{u}$.		
			\item Para cada $\mathbf{u}$, existe un vector $-\mathbf{u}$ tal que
			$ \mathbf{u}+(-\mathbf{u}) = \mathbf{0}$.
			\columnbreak
			\item $c\mathbf{u}$ está en $V$. 
			\item $c(\mathbf{u}+\mathbf{v}) = c\mathbf{u} + c\mathbf{v}$.
			\item $(c+d)\mathbf{u} = c\mathbf{u} + d\mathbf{u}$.
			\item $c(d\mathbf{u}) = (cd)\mathbf{u}$.
			\item $1\mathbf{u} = \mathbf{u}$.
		\end{enumerate}		
	\end{multicols}
	
	\vspace{-2mm}
\end{block}

\begin{ej}{\textbf{Ejemplo 1}}\justifying
	$\rn$ con las operaciones de \textit{suma} y \textit{multiplicación por escalar} estándar es un espacio vectorial.
\end{ej}

\end{frame}
}

%%------------------------------------------------------------------------------------------------------

\subsection{}

{\nologo
\begin{frame}\frametitle{El espacio vectorial de todas las matrices $2\times 3$}

%\begin{exampleblock}{\textbf{Definición 2}}
\begin{block}{\textbf{Axiomas de un espacio vectorial (real) $V$}}	
	\justifying
	Para todo $\mathbf{u}, \mathbf{v}$ y $\mathbf{w}$ en $V$ y todo escalar $c$ y $d$ en $\r$, se 
	cumplen las siguientes propiedades: 
	
	\vspace{-3mm}
	\begin{multicols}{2}		
		\begin{enumerate}			
			\justifying
			\item $\mathbf{u}+\mathbf{v}$ está en $V$. %\\[4mm]			
			\item $\mathbf{u}+\mathbf{v} = \mathbf{v}+\mathbf{u}$. %\\[3mm]			
			\item $(\mathbf{u}+\mathbf{v})+\mathbf{w} = \mathbf{u}+(\mathbf{v}+\mathbf{w})$. %\\[4mm]			
			\item Existe en $V$ un vector cero $\mathbf{0}$ tal que $\mathbf{u}+\mathbf{0} = \mathbf{u}$.		
			\item Para cada $\mathbf{u}$, existe un vector $-\mathbf{u}$ tal que
			$ \mathbf{u}+(-\mathbf{u}) = \mathbf{0}$.
			\columnbreak
			\item $c\mathbf{u}$ está en $V$. 
			\item $c(\mathbf{u}+\mathbf{v}) = c\mathbf{u} + c\mathbf{v}$.
			\item $(c+d)\mathbf{u} = c\mathbf{u} + d\mathbf{u}$.
			\item $c(d\mathbf{u}) = (cd)\mathbf{u}$.
			\item $1\mathbf{u} = \mathbf{u}$.
		\end{enumerate}		
	\end{multicols}
	
	\vspace{-2mm}
\end{block}

\begin{ej}{\textbf{Ejemplo 2}}\justifying
	El conjunto $M_{23}$ de todas las matrices $2\times 3$, con las operaciones de suma de matrices
	y multiplicación por escalares es un espacio vectorial.
\end{ej}

\end{frame}
}

%%------------------------------------------------------------------------------------------------------

\subsection{}

{\nologo
\begin{frame}%\frametitle{El espacio vectorial de todos los poliomios de grado menor o igual que 2}

\vspace{-2.5mm}
\begin{ej}{\textbf{Ejemplo 3}}\justifying
	Considere el conjunto $P_2$ de todos los poliomios de la forma 
	\[
		p(x) = a_2x^2 + a_1x + a_0,
	\]
	donde $a_0,a_1,a_2$ son números reales. La \textit{suma} de dos polinomios 
	\[
		p(x) = a_2x^2 + a_1x + a_0 \qquad \text{y} \qquad q(x) = b_2x^2 + b_1x + b_0
	\]
	se define como
	\[
		(p + q)(x) = (a_2+b_2)x^2 + (a_1+b_1)x + (a_0+b_0)
	\]
	y la \textit{multiplicación por escalar} del polinomio $p(x) = a_2x^2 + a_1x^1 + a_0$ por el escalar $c$ se define como
	\[
		(cp)(x) = ca_2x^2 + ca_1x + ca_0.
	\]
	Demuestre que $P_2$ es un espacio vectorial.
\end{ej}

\vspace{-2mm}
\begin{alertblock}{\textbf{Observación 1}}
	$P_n$ se define como el conjunto de todos los polinomios de grado menor o igual que $n$, junto con el 
	polinomio cero.
\end{alertblock}

\end{frame}
}

%%------------------------------------------------------------------------------------------------------

\subsection{}

\begin{frame}%\frametitle{El espacio vectorial de todos los poliomios de grado menor o igual que 2}


\vspace{-2mm}
\begin{ej}{\textbf{Ejemplo 4}}\justifying
	Considere el conjunto $\mathcal{F}$ de todas las funciones de valor real definidas en la recta numérica.
	La \textit{suma} de dos funciones $f$ y $g$ en $\mathcal{F}$ se define como
	\[
		(f+g)(x) = f(x) + g(x)
	\]
	y la \textit{multiplicación por escalar} de una función $f$ en $\mathcal{F}$ por el escalar $c$ se define como
	\[
		(cf)(x) = cf(x).
	\]
	Demuestre que $\mathcal{F}$ es un espacio vectorial.
\end{ej}

\vspace{-1mm}

\begin{figure}	
	\begin{subfigure}[b]{0.45\textwidth}
		\centering
		\begin{tikzpicture}[thick,scale=0.4, every node/.style={scale=0.6}]%[scale=.8,font=\scriptsize]
		% axis
		\draw[help lines,black,dotted] (-5,-3) grid (5,5);
		\draw[thick,-latex] (-5,0) -- (5,0) node[below] {\large $x$};
		\draw[thick,-latex] (0,-3) -- (0,5) node[above] {\large $y$};
		% Ticks
		\foreach \x in {1,...,4}
		\draw (\x,1pt) -- (\x,-3pt) node[anchor=north] {\x};
		\foreach \x in {-4,...,-1}		
		\draw (\x,1pt) -- (\x,-3pt) node[anchor=north] {\x};	
		\foreach \y in {1,...,4}
		\draw (1pt,\y) -- (-3pt,\y) node[anchor=east] {\y}; 		
		\foreach \y in {-2,...,-1}
		\draw (1pt,\y) -- (-3pt,\y) node[anchor=east] {\y}; 
		% f
		\draw[line width=0.2mm,samples=200,draw=blue,domain=-5:5] plot(\x,{cos(2\x r)-sin(2*\x r)-cos(3*\x r)+0.5});
		\fill[color=blue,draw] (-1,3) node[above] { $f$};	
		% 0.5f
		\draw[line width=0.2mm,samples=200,draw=verde,domain=-5:5] plot(\x,{0.5*(cos(2\x r)-sin(2*\x r)-cos(3*\x r)+0.5)});	
		\fill[color=verde,draw] (-3,2.5) node[above] { $0.5f$};	
		\draw [line width=0.1mm,color=verde,->] (-2.5,2.7) -- (-1,1.5);
		% 1.5f
		\draw[line width=0.2mm,samples=200,draw=red,domain=-5:5] plot(\x,{1.5*(cos(2\x r)-sin(2*\x r)-cos(3*\x r)+0.5))});		
		\fill[color=red,draw] (-1,4.5) node[left] { $1.5f$};	
		\end{tikzpicture}
		\caption{${\color{blue}f}, {\color{verde}0.5f}, {\color{red}1.5f}$}
	\end{subfigure}
	\hfill
	\begin{subfigure}[b]{0.45\textwidth}
		\centering
		\begin{tikzpicture}[thick,scale=0.4, every node/.style={scale=0.6}]%[scale=.8,font=\scriptsize]
		% axis
		\draw[help lines,black,dotted] (-5,-3) grid (5,5);
		\draw[thick,-latex] (-5,0) -- (5,0) node[below] {\large $x$};
		\draw[thick,-latex] (0,-3) -- (0,5) node[above] {\large $y$};
		% Ticks
		\foreach \x in {1,...,4}
		\draw (\x,1pt) -- (\x,-3pt) node[anchor=north] {\x};
		\foreach \x in {-4,...,-1}		
		\draw (\x,1pt) -- (\x,-3pt) node[anchor=north] {\x};	
		\foreach \y in {1,...,4}
		\draw (1pt,\y) -- (-3pt,\y) node[anchor=east] {\y}; 		
		\foreach \y in {-2,...,-1}
		\draw (1pt,\y) -- (-3pt,\y) node[anchor=east] {\y}; 	
		% f
		\draw[line width=0.2mm,samples=200,draw=blue,domain=-5:5] plot(\x,{cos(\x r)-sin(2*\x r)-cos(3*\x r)});		
		\fill[color=blue,draw] (1,0.6) node[above] { $f$};	
		% g
		\draw[line width=0.2mm,samples=200,draw=verde,domain=-5:5] plot(\x,{sin(\x r)+cos(\x r)+1});
		\fill[color=verde,draw] (2.2,1.5) node[above] { $g$};	
		% f+g
		\draw[line width=0.2mm,samples=200,draw=red,domain=-5:5] plot(\x,{cos(\x r)-sin(2*\x r)-cos(3*\x r)+sin(\x r)+cos(\x r)+1});		
		\fill[color=red,draw] (1,3) node[above] { $f+g$};	
		\end{tikzpicture}
		\caption{${\color{blue} f}, {\color{verde} g}, {\color{red} f+g}$}
	\end{subfigure}
	%	\caption{Pictures of animals}\label{fig:animals}
\end{figure}

\end{frame}

%%------------------------------------------------------------------------------------------------------

\subsection{}

{\nologo
\begin{frame}\frametitle{Un conjunto que no es espacio vectorial}

%\begin{exampleblock}{\textbf{Definición 2}}
\begin{block}{\textbf{Axiomas de un espacio vectorial (real) $V$}}	
	\justifying
	Para todo $\mathbf{u}, \mathbf{v}$ y $\mathbf{w}$ en $V$ y todo escalar $c$ y $d$ en $\r$, se 
	cumplen las siguientes propiedades: 
	
	\vspace{-3mm}
	\begin{multicols}{2}		
		\begin{enumerate}			
			\justifying
			\item $\mathbf{u}+\mathbf{v}$ está en $V$. %\\[4mm]			
			\item $\mathbf{u}+\mathbf{v} = \mathbf{v}+\mathbf{u}$. %\\[3mm]			
			\item $(\mathbf{u}+\mathbf{v})+\mathbf{w} = \mathbf{u}+(\mathbf{v}+\mathbf{w})$. %\\[4mm]			
			\item Existe en $V$ un vector cero $\mathbf{0}$ tal que $\mathbf{u}+\mathbf{0} = \mathbf{u}$.		
			\item Para cada $\mathbf{u}$, existe un vector $-\mathbf{u}$ tal que
			$ \mathbf{u}+(-\mathbf{u}) = \mathbf{0}$.
			\columnbreak
			\item $c\mathbf{u}$ está en $V$. 
			\item $c(\mathbf{u}+\mathbf{v}) = c\mathbf{u} + c\mathbf{v}$.
			\item $(c+d)\mathbf{u} = c\mathbf{u} + d\mathbf{u}$.
			\item $c(d\mathbf{u}) = (cd)\mathbf{u}$.
			\item $1\mathbf{u} = \mathbf{u}$.
		\end{enumerate}		
	\end{multicols}
	
	\vspace{-2mm}
\end{block}

\begin{ej}{\textbf{Ejemplo 5}}\justifying
	El conjunto $\z$ de todos los números enteros con las operaciones usuales de suma
	y multiplicación por escalar (producto de enteros) \textbf{no} es un espacio vectorial.
\end{ej}

\end{frame}
}

%%------------------------------------------------------------------------------------------------------

\subsection{}

{\nologo
\begin{frame}\frametitle{Un conjunto que no es espacio vectorial}

%\begin{exampleblock}{\textbf{Definición 2}}
\begin{block}{\textbf{Axiomas de un espacio vectorial (real) $V$}}	
	\justifying
	Para todo $\mathbf{u}, \mathbf{v}$ y $\mathbf{w}$ en $V$ y todo escalar $c$ y $d$ en $\r$, se 
	cumplen las siguientes propiedades: 
	
	\vspace{-3mm}
	\begin{multicols}{2}		
		\begin{enumerate}			
			\justifying
			\item $\mathbf{u}+\mathbf{v}$ está en $V$. %\\[4mm]			
			\item $\mathbf{u}+\mathbf{v} = \mathbf{v}+\mathbf{u}$. %\\[3mm]			
			\item $(\mathbf{u}+\mathbf{v})+\mathbf{w} = \mathbf{u}+(\mathbf{v}+\mathbf{w})$. %\\[4mm]			
			\item Existe en $V$ un vector cero $\mathbf{0}$ tal que $\mathbf{u}+\mathbf{0} = \mathbf{u}$.		
			\item Para cada $\mathbf{u}$, existe un vector $-\mathbf{u}$ tal que
			$ \mathbf{u}+(-\mathbf{u}) = \mathbf{0}$.
			\columnbreak
			\item $c\mathbf{u}$ está en $V$. 
			\item $c(\mathbf{u}+\mathbf{v}) = c\mathbf{u} + c\mathbf{v}$.
			\item $(c+d)\mathbf{u} = c\mathbf{u} + d\mathbf{u}$.
			\item $c(d\mathbf{u}) = (cd)\mathbf{u}$.
			\item $1\mathbf{u} = \mathbf{u}$.
		\end{enumerate}		
	\end{multicols}
	
	\vspace{-2mm}
\end{block}

\begin{ej}{\textbf{Ejemplo 6}}\justifying
	Sea $V=\r^2$ con la definición usual de suma, pero la multiplicación por escalar es la siguiente:
	\[
		c(x_1,x_2) = (cx_1,0).
	\]
	Demuestre que $V$ \textbf{no} es espacio vectorial.
\end{ej}

\end{frame}
}

%------------------------------------------------------------------------------------------------------

\subsection{}

\begin{frame}\frametitle{Propiedades de la multiplicación escalar}

\begin{prop}{\textbf{Propiedad 3}}
	\justifying
	Sea $\mathbf{v}$ un vector de un espacio vectorial $V$ y $c$ un escalar. Entonces:
	\begin{enumerate}
		\item[\labelname{$a$}] $0\mathbf{v} = \mathbf{0}$.
		\item[\labelname{$b$}] $c \mathbf{0} = \mathbf{0}$.
		\item[\labelname{$c$}] Si $c \mathbf{v} = \mathbf{0}$, entonces $c=0$ ó $\mathbf{v} = \mathbf{0}$.
		\item[\labelname{$d$}] $(-1)\mathbf{v} = -\mathbf{v}$.
	\end{enumerate}
\end{prop}

\end{frame}

 
\section{Referencias}

%\begin{frame}[allowframebreaks,c]\frametitle{Bibliografía}
\begin{frame}\frametitle{Bibliografía}

\begin{thebibliography}{99}

\setbeamertemplate{bibliography item}[book]
\bibitem[]{mejia}
Clara Mejía
\newblock {\em Álgebra lineal elemental y aplicaciones}
\newblock Ude@, 2006.

\vspace{2mm}

\setbeamertemplate{bibliography item}[book]
\bibitem[]{grossman}
Stanley Grossman
\newblock {\em Álgebra lineal}
\newblock McGraw-Hill Interamericana, Edición 8, 2019. 

\vspace{2mm}

\setbeamertemplate{bibliography item}[book]
\bibitem[]{poole}
David Poole
\newblock {\em Álgebra lineal: una introducción moderna}
\newblock Cengage Learning Editores, 2011. 

\vspace{2mm}

\setbeamertemplate{bibliography item}[book]
\bibitem[]{Kolman}
Bernard Kolman
\newblock {\em Álgebra lineal}
\newblock Pearson Educación, 2006.

\vspace{2mm}

\setbeamertemplate{bibliography item}[book]
\bibitem[]{Larson}
Ron Larson
\newblock {\em Fundamentos de Álgebra lineal}
\newblock Cengage Learning Editores, 2010. 

\end{thebibliography}


\end{frame}




\end{document}