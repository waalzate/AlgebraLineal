\section{Conjuntos generadores}

\subsection{}

\begin{frame}\frametitle{Combinación lineal}

\begin{block}{\textbf{Definición 1 (Combinación lineal)}}
	\justifying
	Un vector $\mathbf{v}$ en un espacio vectorial $V$ se dice que es \textbf{\textit{combinación lineal}} 
	de los vectores
	\[
		\mathbf{v_1}, \mathbf{v_2}, \hdots, \mathbf{v_k}
	\]
	de $V$, si existen escalares $c_1,c_2, \hdots,c_k$ tales que
	\[
		\mathbf{v} = c_1\mathbf{v_1} + c_2\mathbf{v_2} + \hdots + c_k\mathbf{v_k}.
	\]
\end{block}

%\vspace{5mm}
%
%\begin{ej}{\textbf{Ejemplo 1 }} \justifying
%	Muestre que en el conjunto de vectores de $\r^3$,
%	\[
%		S = \{ \, \underbrace{(1,3,1)}_{\color{blue}\mathbf{v_1}}, \ \underbrace{(0,1,2)}_{\color{blue}\mathbf{v_2}}, \ 
%		       \underbrace{(1,0,-5)}_{\color{blue}\mathbf{v_3}} \, \},
%	\]
%	el vector $\mathbf{v_1}$ es \textbf{\textit{combinación lineal}} de los vectores $\mathbf{v_2}$ y $\mathbf{v_3}$.
%\end{ej}	


\end{frame}

%%------------------------------------------------------------------------------------------------------

\subsection{}

\begin{frame}\frametitle{Combinación lineal}

\begin{block}{\textbf{Definición 1 (Combinación lineal)}}
	\justifying
	Un vector $\mathbf{v}$ en un espacio vectorial $V$ se dice que es \textbf{\textit{combinación lineal}} 
	de los vectores $\mathbf{v_1}, \mathbf{v_2}, \hdots, \mathbf{v_k}$ de $V$ si existen escalares $c_1,c_2, \hdots,c_k$ tales que
	
	\vspace{-2mm}
	\[
	\mathbf{v} = c_1\mathbf{v_1} + c_2\mathbf{v_2} + \hdots + c_k\mathbf{v_k}
	\]
\end{block}

\begin{ej}{\textbf{Ejemplo 1 }} \justifying
	Muestre que en el conjunto de vectores de $\r^3$,
	\[
	S = \{ \, \underbrace{(1,3,1)}_{\color{blue}\mathbf{v_1}}, \ \underbrace{(0,1,2)}_{\color{blue}\mathbf{v_2}}, \ 
	\underbrace{(1,0,-5)}_{\color{blue}\mathbf{v_3}} \, \},
	\]
	el vector $\mathbf{v_1}$ es \textbf{\textit{combinación lineal}} de los vectores $\mathbf{v_2}$ y $\mathbf{v_3}$.
\end{ej}	

\textit{Solución.}

\end{frame}

%%------------------------------------------------------------------------------------------------------

\subsection{}

\begin{frame}\frametitle{Combinación lineal}

\begin{ej}{\textbf{Ejemplo 2}} \justifying
	Muestre que en el conjunto de vectores de $M_{22}$,
	\[
	S = \Bigg\{ \, \underbrace{ \left( \begin{array}{cc}	0 & 8 \\ 2 & 1 \end{array} \right) }_{\color{blue}\mathbf{v_1}}, \ \underbrace{  \left( \begin{array}{cc}	0 & 2 \\ 1 & 0 \end{array} \right) }_{\color{blue}\mathbf{v_2}}, \ 
	\underbrace{ \left( \begin{array}{rr}  -1 & 3 \\ 1 & 2 \end{array} \right) }_{\color{blue}\mathbf{v_3}}, \ 
	\underbrace{ \left( \begin{array}{rr}  -2 & 0 \\ 1 & 3 \end{array} \right) }_{\color{blue}\mathbf{v_4}} \, \Bigg\},
	\]
	el vector $\mathbf{v_1}$ es \textbf{\textit{combinación lineal}} de los vectores $\mathbf{v_2},\mathbf{v_3}$ y $\mathbf{v_4}$.
\end{ej}	

\textit{Solución.}

\end{frame}

%%------------------------------------------------------------------------------------------------------

\subsection{}

\begin{frame}\frametitle{Combinación lineal}

\begin{ej}{\textbf{Ejemplo 3}} \justifying
	En $V=\r^3$, escriba al vector $\mathbf{v}=(1,1,1)$ como combinación lineal de los vectores en el conjunto
	\[
		S = \{ \, \underbrace{(1,2,3)}_{\color{blue}\mathbf{v_1}}, \ \underbrace{(0,1,2)}_{\color{blue}\mathbf{v_2}}, \ 
		\underbrace{(-1,0,1)}_{\color{blue}\mathbf{v_3}} \, \}.
	\]
\end{ej}	

\textit{Solución.}

\end{frame}

%%------------------------------------------------------------------------------------------------------

\subsection{}

\begin{frame}\frametitle{Combinación lineal}

\begin{ej}{\textbf{Ejemplo 4}} \justifying
	En $V=\r^3$, escriba al vector $\mathbf{w}=(1,-2,2)$ como combinación lineal de los vectores en el conjunto
	\[
	S = \{ \, \underbrace{(1,2,3)}_{\color{blue}\mathbf{v_1}}, \ \underbrace{(0,1,2)}_{\color{blue}\mathbf{v_2}}, \ 
	\underbrace{(-1,0,1)}_{\color{blue}\mathbf{v_3}} \, \}.
	\]
\end{ej}	

\textit{Solución.}

\end{frame}

%------------------------------------------------------------------------------------------------------

\subsection{}

\begin{frame}\frametitle{Conjuntos generadores}

\begin{block}{\textbf{Definición 2 (Conjunto generador)}}
	\justifying
	Se dice que un conjunto de vectores $S=\{\mathbf{v}_1,\mathbf{v}_2,\hdots,\mathbf{v}_k\}$ en un espacio 
	vectorial $V$ \textbf{\textit{genera}} a $V$ si todo vector en $V$ se puede escribir como combinación lineal
	de los vectores en $S$. Es decir, para todo vector $\mathbf{v}\in V$, existen escalares $c_1,c_2,\hdots,c_k$
	tales que 
	
	\vspace{-3mm}
	\[
		\mathbf{v} = c_1\mathbf{v}_1 + c_2\mathbf{v}_2 + \cdots + c_k\mathbf{v}_k
	\]
	
\end{block}

\vspace{0mm}

\begin{ej}{\textbf{Ejemplo 5}} \justifying
	El conjunto de vectores
	
	\vspace{-2mm}
	\[
	S = \{ \, \underbrace{(1,0,0)}_{\mathbf{i}}, \ \underbrace{(0,1,0)}_{\mathbf{j}}, \ 
	\underbrace{(0,0,1)}_{\mathbf{k}} \, \}.
	\]
	
	\vspace{-3mm}
	\textbf{\textit{genera}} a $\r^3$.
\end{ej}

\textit{Solución.}	
	
\end{frame}

%------------------------------------------------------------------------------------------------------

\subsection{}

\begin{frame}\frametitle{Conjuntos generadores de polinomios}

\begin{block}{\textbf{Definición 2 (Conjunto generador)}}
	\justifying
	Se dice que un conjunto de vectores $S=\{\mathbf{v}_1,\mathbf{v}_2,\hdots,\mathbf{v}_k\}$ en un espacio 
	vectorial $V$ \textbf{\textit{genera}} a $V$ si todo vector en $V$ se puede escribir como combinación lineal
	de los vectores en $S$. Es decir, para todo vector $\mathbf{v}\in V$, existen escalares $c_1,c_2,\hdots,c_k$
	tales que 
	
	\vspace{-3mm}
	\[
	\mathbf{v} = c_1\mathbf{v}_1 + c_2\mathbf{v}_2 + \cdots + c_k\mathbf{v}_k
	\]
	
\end{block}

\vspace{0mm}

\begin{ej}{\textbf{Ejemplo 6}} \justifying
	El conjunto de vectores
	\[
	S = \left\{ \, 1, x, x^2 \right\}
	\]
	\textbf{\textit{genera}} a $P_2$.
\end{ej}	

\textit{Solución}.

\end{frame}

%------------------------------------------------------------------------------------------------------

\subsection{}

\begin{frame}\frametitle{Conjuntos generadores de polinomios}

\begin{block}{\textbf{Definición 2 (Conjunto generador)}}
	\justifying
	Se dice que un conjunto de vectores $S=\{\mathbf{v}_1,\mathbf{v}_2,\hdots,\mathbf{v}_k\}$ en un espacio 
	vectorial $V$ \textbf{\textit{genera}} a $V$ si todo vector en $V$ se puede escribir como combinación lineal
	de los vectores en $S$. Es decir, para todo vector $\mathbf{v}\in V$, existen escalares $c_1,c_2,\hdots,c_k$
	tales que 
	
	\vspace{-3mm}
	\[
	\mathbf{v} = c_1\mathbf{v}_1 + c_2\mathbf{v}_2 + \cdots + c_k\mathbf{v}_k
	\]
	
\end{block}

\vspace{0mm}

\begin{ej}{\textbf{Ejemplo 7}} \justifying
	Ningún conjunto finito de polinomios genera a $P$.
\end{ej}	

\textit{Solución}.

\end{frame}

%------------------------------------------------------------------------------------------------------

\subsection{}

\begin{frame}\frametitle{Conjuntos generador de $\r^3$}

\begin{ej}{\textbf{Ejemplo 8}} \justifying
	El conjunto de vectores
	\[
		S = \{ \, \underbrace{(1,2,3)}_{\color{blue}\mathbf{v_1}}, \ \underbrace{(0,1,2)}_{\color{blue}\mathbf{v_2}}, \ 
		\underbrace{(-2,0,1)}_{\color{blue}\mathbf{v_3}} \, \},
	\]
	\textbf{\textit{genera}} a $\r^3$.
\end{ej}	

\textit{Solución}.

\end{frame}

%------------------------------------------------------------------------------------------------------

\subsection{}

\begin{frame}\frametitle{Conjuntos generador de $M_{22}$}

\begin{ej}{\textbf{Ejemplo 9}} \justifying
	El conjunto de vectores
	\[
		S = \Bigg\{ \, \underbrace{ \left( \begin{array}{cc} 1 & 0 \\ 0 & 0 \end{array} \right) }_{\color{blue}\mathbf{v_1}}, \ \underbrace{  \left( \begin{array}{cc}	0 & 1 \\ 0 & 0 \end{array} \right) }_{\color{blue}\mathbf{v_2}}, \ 
		\underbrace{ \left( \begin{array}{rr}   0 & 0 \\ 1 & 0 \end{array} \right) }_{\color{blue}\mathbf{v_3}}, \ 
		\underbrace{ \left( \begin{array}{rr}   0 & 0 \\ 0 & 1 \end{array} \right) }_{\color{blue}\mathbf{v_4}} \, \Bigg\},
	\]
	\textbf{\textit{genera}} a $M_{22}$.
\end{ej}	

\textit{Solución}.

\end{frame}

%------------------------------------------------------------------------------------------------------

\subsection{}

{\nologo
\begin{frame}\frametitle{Espacio generado por un conjunto de vectores}

\begin{block}{\textbf{Definición 3 (Espacio generado)}}
	\justifying
	El \textbf{\textit{espacio generado}} por un conjunto de vectores 
	\[
	S=\{\mathbf{v}_1,\mathbf{v}_2,\hdots,\mathbf{v}_k\}
	\]
	en un espacio vectorial $V$, se define como el conjunto de TODAS las combinaciones lineales de $S$.
	Al \textbf{\textit{espacio generado}} por $S$ se le denota por 
	\[
	\text{gen}\,(S) \qquad \text{ó} \qquad \text{gen}\, \{\mathbf{v}_1,\mathbf{v}_2,\hdots,\mathbf{v}_k\}.
	\]
	\[
	\text{gen}\,(S) = \Big\{ \, c_1\mathbf{v}_1 + c_2\mathbf{v}_2 + \cdots + c_k\mathbf{v}_k \mid c_1, c_2,\hdots,c_k  
	\text{ son números reales } \, \Big\}.
	\]
\end{block}

\begin{ej}{\textbf{Ejemplo 10}} \justifying
	Halle el espacio generado por el vector $\mathbf{v}=(2,1)$ de $\r^2$.
\end{ej}	

\textit{Solución}.

\end{frame}
}

%------------------------------------------------------------------------------------------------------

\subsection{}

{\nologo
\begin{frame}\frametitle{Espacio generado por un conjunto de vectores}

\begin{block}{\textbf{Definición 3 (Espacio generado)}}
	\justifying
	El \textbf{\textit{espacio generado}} por un conjunto de vectores 
	\[
	S=\{\mathbf{v}_1,\mathbf{v}_2,\hdots,\mathbf{v}_k\}
	\]
	en un espacio vectorial $V$, se define como el conjunto de TODAS las combinaciones lineales de $S$.
	Al \textbf{\textit{espacio generado}} por $S$ se le denota por 
	\[
	\text{gen}\,(S) \qquad \text{ó} \qquad \text{gen}\, \{\mathbf{v}_1,\mathbf{v}_2,\hdots,\mathbf{v}_k\}.
	\]
	\[
	\text{gen}\,(S) = \Big\{ \, c_1\mathbf{v}_1 + c_2\mathbf{v}_2 + \cdots + c_k\mathbf{v}_k \mid c_1, c_2,\hdots,c_k  
	\text{ son números reales } \, \Big\}.
	\]
\end{block}

\begin{prop}{\textbf{Propiedad 1}}
	Si $\mathbf{v}_1,\mathbf{v}_2,\hdots,\mathbf{v}_k$ son vectores en un espacio vectorial $V$, entonces 
	\[
		\text{gen}\, \{\mathbf{v}_1,\mathbf{v}_2,\hdots,\mathbf{v}_k\}
	\]
	es un subespacio vectorial de $V$.
\end{prop}	

\end{frame}
}

%------------------------------------------------------------------------------------------------------

\subsection{}

{\nologo
\begin{frame}\frametitle{Espacio generado por un conjunto de vectores}

\vspace{-2mm}	

	\begin{block}{\textbf{Definición 3 (Espacio generado)}}
		\justifying
		El \textbf{\textit{espacio generado}} por un conjunto de vectores 
		\[
		S=\{\mathbf{v}_1,\mathbf{v}_2,\hdots,\mathbf{v}_k\}
		\]
		en un espacio vectorial $V$, se define como el conjunto de TODAS las combinaciones lineales de $S$.
		Al \textbf{\textit{espacio generado}} por $S$ se le denota por 
		\[
		\text{gen}\,(S) \qquad \text{ó} \qquad \text{gen}\, \{\mathbf{v}_1,\mathbf{v}_2,\hdots,\mathbf{v}_k\}.
		\]
		\[
		\text{gen}\,(S) = \Big\{ \, c_1\mathbf{v}_1 + c_2\mathbf{v}_2 + \cdots + c_k\mathbf{v}_k \mid c_1, c_2,\hdots,c_k  
		\text{ son números reales } \, \Big\}.
		\]
	\end{block}
	
	\begin{ej}{\textbf{Ejemplo 11}} \justifying
		Demuestre que si $\mathbf{v}_1,\mathbf{v}_2,\hdots,\mathbf{v}_m$ son vectores que generan un espacio vectorial $V$, entonces para todo vector $\mathbf{w}$ en $V$, los vectores
		$\mathbf{w}, \mathbf{v}_1,\mathbf{v}_2,\hdots,\mathbf{v}_m$  también generan a $V$.
	\end{ej}
	\textit{Solución.}
	
\end{frame}
}

%------------------------------------------------------------------------------------------------------

\subsection{}

{\nologo
\begin{frame}\frametitle{Espacio generado por un conjunto de vectores}
	
	\vspace{-2mm}
	\begin{block}{\textbf{Definición 3 (Espacio generado)}}
		\justifying
		El \textbf{\textit{espacio generado}} por un conjunto de vectores 
		\[
		S=\{\mathbf{v}_1,\mathbf{v}_2,\hdots,\mathbf{v}_k\}
		\]
		en un espacio vectorial $V$, se define como el conjunto de TODAS las combinaciones lineales de $S$.
		Al \textbf{\textit{espacio generado}} por $S$ se le denota por 
		\[
		\text{gen}\,(S) \qquad \text{ó} \qquad \text{gen}\, \{\mathbf{v}_1,\mathbf{v}_2,\hdots,\mathbf{v}_k\}.
		\]
		\[
		\text{gen}\,(S) = \Big\{ \, c_1\mathbf{v}_1 + c_2\mathbf{v}_2 + \cdots + c_k\mathbf{v}_k \mid c_1, c_2,\hdots,c_k  
		\text{ son números reales } \, \Big\}.
		\]
	\end{block}
	
	\begin{ej}{\textbf{Ejemplo 12}} \justifying
		Demuestre que si $\mathbf{v}_1,\mathbf{v}_2,\hdots,\mathbf{v}_m$ son vectores que generan un espacio vectorial $V$ y que si uno de los vectores $\mathbf{v}_k$ es combinación lineal del resto, entonces los vectores $\mathbf{v}_1,\hdots,\mathbf{v}_m$ sin el vector $\mathbf{v}_k$ también generan a $V$.
	\end{ej}
	\textit{Solución.}
	
\end{frame}
}

%------------------------------------------------------------------------------------------------------

\subsection{}

{\nologo
\begin{frame}\frametitle{Espacio generado por un conjunto de vectores}
	
	\vspace{-2mm}
	\begin{block}{\textbf{Definición 3 (Espacio generado)}}
		\justifying
		El \textbf{\textit{espacio generado}} por un conjunto de vectores 
		\[
		S=\{\mathbf{v}_1,\mathbf{v}_2,\hdots,\mathbf{v}_k\}
		\]
		en un espacio vectorial $V$, se define como el conjunto de TODAS las combinaciones lineales de $S$.
		Al \textbf{\textit{espacio generado}} por $S$ se le denota por 
		\[
		\text{gen}\,(S) \qquad \text{ó} \qquad \text{gen}\, \{\mathbf{v}_1,\mathbf{v}_2,\hdots,\mathbf{v}_k\}.
		\]
		\[
		\text{gen}\,(S) = \Big\{ \, c_1\mathbf{v}_1 + c_2\mathbf{v}_2 + \cdots + c_k\mathbf{v}_k \mid c_1, c_2,\hdots,c_k  
		\text{ son números reales } \, \Big\}.
		\]
	\end{block}
	
	\begin{ej}{\textbf{Ejemplo 13}} \justifying
		Demuestre que si $\mathbf{v}_1,\mathbf{v}_2,\hdots,\mathbf{v}_m$ son vectores que generan un espacio vectorial $V$ y que si uno de los vectores $\mathbf{v}_k$ es el vector cero, entonces los vectores $\mathbf{v}_1,\hdots,\mathbf{v}_m$ sin el vector cero también generan a $V$.
	\end{ej}
	\textit{Solución.}
	
\end{frame}
}
