\section{Independencia lineal}

\subsection{}

{\nologo
\begin{frame}\frametitle{Independencia lineal}

\begin{alertblock}{\textbf{Observación 1}}
	\begin{enumerate}\justifying
		\item[\labelname{$a$}] Dado un conjunto de vectores $S=\{\mathbf{v}_1,\mathbf{v}_2,\hdots,\mathbf{v}_k\}$,
		la ecuación
		\[
		c_1\mathbf{v}_1+c_2\mathbf{v}_2+\hdots+c_k\mathbf{v}_k = \mathbf{0}
		\]
		siempre tiene la \textbf{solución trivial}
		\[
		c_1 =0,\ c_2 =0,\ \hdots,\ c_k=0.
		\]
		
		\medskip
		\item[\labelname{$b$}] En el ejemplo 1 vimos que  $\mathbf{v}_1=3\mathbf{v}_2+\mathbf{v}_3$, donde  
		\[
		\mathbf{v_1}=(1,3,1),\ \mathbf{v_2}=(0,1,2) \ \text{ y } \  \mathbf{v_3}=(1,0,-5)
		\]
		y por tanto la ecuación
		\[
		c_1\mathbf{v}_1+c_2\mathbf{v}_2+c_3\mathbf{v}_3 = \mathbf{0}	
		\]
		tiene la \textbf{solución NO trivial}
		\[
		c_1 =1,\ c_2 =-3,\  c_3=-1.
		\]
		
	\end{enumerate}
\end{alertblock}

\end{frame}
}

%------------------------------------------------------------------------------------------------------

\subsection{}

\begin{frame}\frametitle{Independencia lineal}

\begin{block}{\textbf{Definición 1 (Independencia lineal)}}
\justifying
Sea $S=\{\mathbf{v}_1,\mathbf{v}_2,\hdots,\mathbf{v}_k\}$ un conjunto de vectores
en un espacio vectorial $V$.

\begin{enumerate}%\justifying
	\item[\labelname{$a$}] Se dice que $S$ es \textbf{\textit{linealmente independiente}} (LI) si
	la ecuación vectorial
	\begin{equation}\label{LI}
	c_1\mathbf{v}_1+c_2\mathbf{v}_2+\hdots+c_k\mathbf{v}_k = \mathbf{0}
	\end{equation}
	tiene solamente la \textbf{solución trivial}
	\[
	c_1 =0,\ c_2 =0,\ \hdots,\ c_k=0.
	\]
	
	\item[\labelname{$b$}] Si existen soluciones no triviales de \eqref{LI}, se dice que $S$ es \textbf{\textit{linealmente dependiente}} (LD).
\end{enumerate}		
\end{block}

\end{frame}

%------------------------------------------------------------------------------------------------------

\subsection{}

{\nologo
\begin{frame}\frametitle{Conjuntos linealmente dependientes}

\vspace{-2mm}
\begin{block}{\textbf{Definición 1 (Independencia lineal)}}
	\justifying
	Sea $S=\{\mathbf{v}_1,\mathbf{v}_2,\hdots,\mathbf{v}_k\}$ un conjunto de vectores
	en un espacio vectorial $V$.
	
	\begin{enumerate}%\justifying
		\item[\labelname{$a$}] Se dice que $S$ es \textbf{\textit{linealmente independiente}} (LI) si
		la ecuación vectorial
		\begin{equation}\tag{1}
		c_1\mathbf{v}_1+c_2\mathbf{v}_2+\hdots+c_k\mathbf{v}_k = \mathbf{0}
		\end{equation}
		tiene solamente la \textbf{solución trivial}
		\[
		c_1 =0,\ c_2 =0,\ \hdots,\ c_k=0.
		\]
		
		\item[\labelname{$b$}] Si existen soluciones no triviales de \eqref{LI}, se dice que $S$ es \textbf{\textit{linealmente dependiente}} (LD).
	\end{enumerate}		
\end{block}

\begin{ej}{\textbf{Ejemplo 1}} \justifying
Muestre que el conjunto de vectores $S=\{ (1,2), (2,4) \}$ en $\r^2$ es linealmente dependiente (LD).
\end{ej}	
\textit{Solución}

\end{frame}
}

%------------------------------------------------------------------------------------------------------

\subsection{}

{\nologo
\begin{frame}\frametitle{Conjuntos linealmente dependientes}

\vspace{-2mm}
\begin{block}{\textbf{Definición 1 (Independencia lineal)}}
	\justifying
	Sea $S=\{\mathbf{v}_1,\mathbf{v}_2,\hdots,\mathbf{v}_k\}$ un conjunto de vectores
	en un espacio vectorial $V$.
	
	\begin{enumerate}%\justifying
		\item[\labelname{$a$}] Se dice que $S$ es \textbf{\textit{linealmente independiente}} (LI) si
		la ecuación vectorial
		\begin{equation}\tag{1}
		c_1\mathbf{v}_1+c_2\mathbf{v}_2+\hdots+c_k\mathbf{v}_k = \mathbf{0}
		\end{equation}
		tiene solamente la \textbf{solución trivial}
		\[
		c_1 =0,\ c_2 =0,\ \hdots,\ c_k=0.
		\]
		
		\item[\labelname{$b$}] Si existen soluciones no triviales de \eqref{LI}, se dice que $S$ es \textbf{\textit{linealmente dependiente}} (LD).
	\end{enumerate}		
\end{block}

\begin{ej}{\textbf{Ejemplo 2}} \justifying
	Muestre que el conjunto de vectores $S=\{ (0,0), (1,2) \}$ en $\r^2$ es linealmente dependiente (LD).
\end{ej}	
\textit{Solución}

\end{frame}
}

%------------------------------------------------------------------------------------------------------



\subsection{}

{\nologo
\begin{frame}\frametitle{Conjuntos linealmente dependientes}

\vspace{-2mm}
\begin{block}{\textbf{Definición 1 (Independencia lineal)}}
	\justifying
	Sea $S=\{\mathbf{v}_1,\mathbf{v}_2,\hdots,\mathbf{v}_k\}$ un conjunto de vectores
	en un espacio vectorial $V$.
	
	\begin{enumerate}%\justifying
		\item[\labelname{$a$}] Se dice que $S$ es \textbf{\textit{linealmente independiente}} (LI) si
		la ecuación vectorial
		\begin{equation}\tag{1}
		c_1\mathbf{v}_1+c_2\mathbf{v}_2+\hdots+c_k\mathbf{v}_k = \mathbf{0}
		\end{equation}
		tiene solamente la \textbf{solución trivial}
		\[
		c_1 =0,\ c_2 =0,\ \hdots,\ c_k=0.
		\]
		
		\item[\labelname{$b$}] Si existen soluciones no triviales de \eqref{LI}, se dice que $S$ es \textbf{\textit{linealmente dependiente}} (LD).
	\end{enumerate}		
\end{block}

\begin{ej}{\textbf{Ejemplo 3}} \justifying
	Muestre que el conjunto de vectores $S=\{ (1,0), (0,1), (-2,5) \}$ en $\r^2$ es linealmente dependiente (LD).
\end{ej}	
\textit{Solución}

\end{frame}
}

%------------------------------------------------------------------------------------------------------

\subsection{}

{\nologo
\begin{frame}\frametitle{Conjuntos linealmente dependientes}

\begin{block}{\textbf{Definición 1 (Independencia lineal)}}
	\justifying
	Sea $S=\{\mathbf{v}_1,\mathbf{v}_2,\hdots,\mathbf{v}_k\}$ un conjunto de vectores
	en un espacio vectorial $V$.
	
	\begin{enumerate}%\justifying
		\item[\labelname{$a$}] Se dice que $S$ es \textbf{\textit{linealmente independiente}} (LI) si
		la ecuación vectorial
		\begin{equation}\tag{1}
		c_1\mathbf{v}_1+c_2\mathbf{v}_2+\hdots+c_k\mathbf{v}_k = \mathbf{0}
		\end{equation}
		tiene solamente la \textbf{solución trivial}
		\[
		c_1 =0,\ c_2 =0,\ \hdots,\ c_k=0.
		\]
		
		\item[\labelname{$b$}] Si existen soluciones no triviales de \eqref{LI}, se dice que $S$ es \textbf{\textit{linealmente dependiente}} (LD).
	\end{enumerate}		
\end{block}

\begin{prop}{\textbf{Propiedad 1}}
	Dos vectores en un espacio vectorial son linealmente dependientes si y sólo si uno de ellos es un múltiplo escalar del otro.
\end{prop}

\end{frame}
}

%------------------------------------------------------------------------------------------------------

\subsection{}

\begin{frame}\frametitle{Conjuntos linealmente independientes}

\begin{ej}{\textbf{Ejemplo 4}} \justifying
Determine si el conjunto de vectores
\[
S = \{ \, \underbrace{(1,2,3)}_{\color{blue}\mathbf{v_1}}, \ \underbrace{(0,1,2)}_{\color{blue}\mathbf{v_2}}, \ 
\underbrace{(-2,0,1)}_{\color{blue}\mathbf{v_3}} \, \}
\]
en $V=\r^3$ es linealmente dependiente (LD) o independiente (LI).
\end{ej}	
\textit{Solución.}

\end{frame}

%------------------------------------------------------------------------------------------------------

\subsection{}

{\nologo
\begin{frame}\frametitle{Test para la independencia lineal}

\begin{ejem}{\textbf{Procedimiento 1}}
	Sea $S=\{\mathbf{v}_1,\mathbf{v}_2,\hdots,\mathbf{v}_k\}$ un conjunto de vectores
	en un espacio vectorial $V$. Para determinar si $S$ es LI o LD efectúe los siguientes pasos:
	\begin{enumerate}
		\item[\labelname{$a$}] A partir de la ecuación vectorial
		\[
			c_1\mathbf{v}_1+c_2\mathbf{v}_2+\hdots+c_k\mathbf{v}_k = \mathbf{0}
		\]
		escriba un sistema homogéneo de ecuaciones lineales en las variables $c_1,\hdots,c_k$.
		\item[\labelname{$b$}] Utilice eliminación gaussiana para resolver el sistema.
		\item[\labelname{$c$}] Si el sistema tiene solamente la solución trivial 
		\[	
			c_1 =0,\ c_2 =0,\ \hdots,\ c_k=0,
		\]
		entonces el conjunto $S$ es linealmente independiente (LI).
		\item[\labelname{$d$}] Si el sistema tiene soluciones no triviales, entonces el conjunto $S$ es linealmente dependiente (LD). 
	\end{enumerate}
\end{ejem}	

\end{frame}
}

%------------------------------------------------------------------------------------------------------

\subsection{}

\begin{frame}\frametitle{Test para la independencia lineal}

\begin{ej}{\textbf{Ejemplo 5}} \justifying
	Determine si el siguiente conjunto de vectores en $P_2$ es LI o LD.
	
	\vspace{-2mm}
	\[
		\bigg\{ \underbrace{1+x-2x^2}_{\color{blue}\mathbf{v_1}},\ \underbrace{2+5x-x^2}_{\color{blue}\mathbf{v_2}},
		\ \underbrace{x+x^2}_{\color{blue}\mathbf{v_3}} \bigg\}
	\]
\end{ej}	
\textit{Solución.}

\end{frame}

%------------------------------------------------------------------------------------------------------

\subsection{}

{\nologo
\begin{frame}\frametitle{Conjuntos linealmente dependientes}

\begin{block}{\textbf{Definición 1 (Independencia lineal)}}
	\justifying
	Sea $S=\{\mathbf{v}_1,\mathbf{v}_2,\hdots,\mathbf{v}_k\}$ un conjunto de vectores
	en un espacio vectorial $V$.
	
	\begin{enumerate}[$a$]%\justifying
		\item Se dice que $S$ es \textbf{\textit{linealmente independiente}} (LI) si
		la ecuación vectorial
		\begin{equation}\tag{1}
		c_1\mathbf{v}_1+c_2\mathbf{v}_2+\hdots+c_k\mathbf{v}_k = \mathbf{0}
		\end{equation}
		tiene solamente la \textbf{solución trivial}
		\[
		c_1 =0,\ c_2 =0,\ \hdots,\ c_k=0.
		\]
		
		\item Si existen soluciones no triviales de \eqref{LI}, se dice que $S$ es \textbf{\textit{linealmente dependiente}} (LD).
	\end{enumerate}		
\end{block}

\begin{prop}{\textbf{Propiedad 2}}\justifying
	Un conjunto de vectores $S=\{\mathbf{v}_1,\mathbf{v}_2,\hdots,\mathbf{v}_k\}$ es linealmente dependiente (LD)
	si y sólo si al menos uno de los vectores $\mathbf{v}_j$ puede escribirse como combinación lineal de los otros
	vectores en $S$.
\end{prop}

\end{frame}
}

%------------------------------------------------------------------------------------------------------

\subsection{}

{\nologo
\begin{frame}\frametitle{Conjuntos linealmente dependientes}
	
	\vspace{-2mm}
	\begin{block}{\textbf{Definición 1 (Independencia lineal)}}
		\justifying
		Sea $S=\{\mathbf{v}_1,\mathbf{v}_2,\hdots,\mathbf{v}_k\}$ un conjunto de vectores
		en un espacio vectorial $V$.
		
		\begin{enumerate}[$a$]%\justifying
			\item Se dice que $S$ es \textbf{\textit{linealmente independiente}} (LI) si
			la ecuación vectorial
			\begin{equation}\tag{1}
			c_1\mathbf{v}_1+c_2\mathbf{v}_2+\hdots+c_k\mathbf{v}_k = \mathbf{0}
			\end{equation}
			tiene solamente la \textbf{solución trivial}
			\[
			c_1 =0,\ c_2 =0,\ \hdots,\ c_k=0.
			\]
			
			\item Si existen soluciones no triviales de \eqref{LI}, se dice que $S$ es \textbf{\textit{linealmente dependiente}} (LD).
		\end{enumerate}		
	\end{block}
	
	\begin{ej}{\textbf{Ejemplo 6}} \justifying
		Suponga que $\mathbf{v}$ es un vector de un espacio vectorial $V$.
		Determine si el conjunto $\{\mathbf{v}\}$ es LI o LD.		
	\end{ej}	
	\textit{Solución.}
	
\end{frame}
}

%------------------------------------------------------------------------------------------------------

\subsection{}

{\nologo
\begin{frame}\frametitle{Conjuntos linealmente dependientes}
	
	\vspace{-2mm}
	\begin{block}{\textbf{Definición 1 (Independencia lineal)}}
		\justifying
		Sea $S=\{\mathbf{v}_1,\mathbf{v}_2,\hdots,\mathbf{v}_k\}$ un conjunto de vectores
		en un espacio vectorial $V$.
		
		\begin{enumerate}[$a$]%\justifying
			\item Se dice que $S$ es \textbf{\textit{linealmente independiente}} (LI) si
			la ecuación vectorial
			\begin{equation}\tag{1}
			c_1\mathbf{v}_1+c_2\mathbf{v}_2+\hdots+c_k\mathbf{v}_k = \mathbf{0}
			\end{equation}
			tiene solamente la \textbf{solución trivial}
			\[
			c_1 =0,\ c_2 =0,\ \hdots,\ c_k=0.
			\]
			
			\item Si existen soluciones no triviales de \eqref{LI}, se dice que $S$ es \textbf{\textit{linealmente dependiente}} (LD).
		\end{enumerate}		
	\end{block}
	
	\begin{ej}{\textbf{Ejemplo 7}} \justifying
		Suponga que $\mathbf{v}_1,\mathbf{v}_2,\hdots,\mathbf{v}_m$ son vectores en un espacio 
		vectorial $V$ y que $\mathbf{v}_1$ es el vector cero. Determine si el conjunto 
		$\{\mathbf{v}_1,\mathbf{v}_2,\hdots,\mathbf{v}_m\}$ es LI o LD.		
	\end{ej}	
	\textit{Solución.}
	
\end{frame}
}

%------------------------------------------------------------------------------------------------------

\subsection{}

{\nologo
\begin{frame}\frametitle{Conjuntos linealmente dependientes}
	
	\vspace{-2mm}
	\begin{block}{\textbf{Definición 1 (Independencia lineal)}}
		\justifying
		Sea $S=\{\mathbf{v}_1,\mathbf{v}_2,\hdots,\mathbf{v}_k\}$ un conjunto de vectores
		en un espacio vectorial $V$.
		
		\begin{enumerate}[$a$]%\justifying
			\item Se dice que $S$ es \textbf{\textit{linealmente independiente}} (LI) si
			la ecuación vectorial
			\begin{equation}\tag{1}
			c_1\mathbf{v}_1+c_2\mathbf{v}_2+\hdots+c_k\mathbf{v}_k = \mathbf{0}
			\end{equation}
			tiene solamente la \textbf{solución trivial}
			\[
			c_1 =0,\ c_2 =0,\ \hdots,\ c_k=0.
			\]
			
			\item Si existen soluciones no triviales de \eqref{LI}, se dice que $S$ es \textbf{\textit{linealmente dependiente}} (LD).
		\end{enumerate}		
	\end{block}
	
	\begin{ej}{\textbf{Ejemplo 8}} \justifying
		Suponga que en el listado de vectores $\mathbf{v}_1,\mathbf{v}_2,\hdots,\mathbf{v}_m$ uno de ellos
		es múltiplo escalar de otro. Determine si el conjunto 
		$\{\mathbf{v}_1,\mathbf{v}_2,\hdots,\mathbf{v}_m\}$ es LI o LD.		
	\end{ej}	
	\textit{Solución.}
	
\end{frame}
}