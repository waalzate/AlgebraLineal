\section{Espacios vectoriales}

\subsection{}

{\nologo
\begin{frame}\frametitle{Espacios vectoriales (reales)}

\vspace{-3mm}
%\begin{exampleblock}{\textbf{Definición 2}}
\begin{block}{\textbf{Definición 5 (Espacio vectorial)}}	
	\justifying
	Sea $V$ un conjunto (no vacío) en el que están definidas dos operaciones (\textbf{suma de vectores} y 
	\textbf{multiplicación por escalar}). Se dice que $V$ es un {\color{red} espacio vectorial (real)} si para todo
	$\mathbf{u}, \mathbf{v}$ y $\mathbf{w}$ en $V$ y todo escalar (número real) $c$ y $d$ en $\r$, se 
	cumplen las siguientes propiedades: 
	\begin{multicols}{2}		
		\begin{enumerate}			
			\justifying
			\item $\mathbf{u}+\mathbf{v}$ está en $V$. \\[4mm]			
			\item $\mathbf{u}+\mathbf{v} = \mathbf{v}+\mathbf{u}$. \\[3mm]			
			\item $(\mathbf{u}+\mathbf{v})+\mathbf{w} = \mathbf{u}+(\mathbf{v}+\mathbf{w})$. \\[4mm]			
			\item Existe en $V$ un vector cero $\mathbf{0}$ tal que
			\[
			\mathbf{u}+\mathbf{0} = \mathbf{u}.
			\]
			
			\vspace{2mm}	
			\item Para cada $\mathbf{u}$, existe en $V$ un vector denotado por $-\mathbf{u}$ tal que
			\[
			\mathbf{u}+(-\mathbf{u}) = \mathbf{0}.
			\]	
			\columnbreak
			\item $c\mathbf{u}$ está en $V$. \\[4mm]			
			\item $c(\mathbf{u}+\mathbf{v}) = c\mathbf{u} + c\mathbf{v}$. \\[3mm]
			\item $(c+d)\mathbf{u} = c\mathbf{u} + d\mathbf{u}$. \\[4mm]
			\item $c(d\mathbf{u}) = (cd)\mathbf{u}$. \\[1.1cm]
			\item $1\mathbf{u} = \mathbf{u}$.
		\end{enumerate}		
	\end{multicols}
	
	\vspace{-8mm}
\end{block}

\end{frame}
}

%%------------------------------------------------------------------------------------------------------

\subsection{}

{\nologo
\begin{frame}\frametitle{$\rn$ con las operaciones estándar es un espacio vectorial}

%\begin{exampleblock}{\textbf{Definición 2}}
\begin{block}{\textbf{Axiomas de un espacio vectorial (real) $V$}}	
	\justifying
	Para todo $\mathbf{u}, \mathbf{v}$ y $\mathbf{w}$ en $V$ y todo escalar $c$ y $d$ en $\r$, se 
	cumplen las siguientes propiedades: 
	
	\vspace{-3mm}
	\begin{multicols}{2}		
		\begin{enumerate}			
			\justifying
			\item $\mathbf{u}+\mathbf{v}$ está en $V$. %\\[4mm]			
			\item $\mathbf{u}+\mathbf{v} = \mathbf{v}+\mathbf{u}$. %\\[3mm]			
			\item $(\mathbf{u}+\mathbf{v})+\mathbf{w} = \mathbf{u}+(\mathbf{v}+\mathbf{w})$. %\\[4mm]			
			\item Existe en $V$ un vector cero $\mathbf{0}$ tal que $\mathbf{u}+\mathbf{0} = \mathbf{u}$.		
			\item Para cada $\mathbf{u}$, existe un vector $-\mathbf{u}$ tal que
			$ \mathbf{u}+(-\mathbf{u}) = \mathbf{0}$.
			\columnbreak
			\item $c\mathbf{u}$ está en $V$. 
			\item $c(\mathbf{u}+\mathbf{v}) = c\mathbf{u} + c\mathbf{v}$.
			\item $(c+d)\mathbf{u} = c\mathbf{u} + d\mathbf{u}$.
			\item $c(d\mathbf{u}) = (cd)\mathbf{u}$.
			\item $1\mathbf{u} = \mathbf{u}$.
		\end{enumerate}		
	\end{multicols}
	
	\vspace{-2mm}
\end{block}

\begin{ej}{\textbf{Ejemplo 1}}\justifying
	$\rn$ con las operaciones de \textit{suma} y \textit{multiplicación por escalar} estándar es un espacio vectorial.
\end{ej}

\end{frame}
}

%%------------------------------------------------------------------------------------------------------

\subsection{}

{\nologo
\begin{frame}\frametitle{El espacio vectorial de todas las matrices $2\times 3$}

%\begin{exampleblock}{\textbf{Definición 2}}
\begin{block}{\textbf{Axiomas de un espacio vectorial (real) $V$}}	
	\justifying
	Para todo $\mathbf{u}, \mathbf{v}$ y $\mathbf{w}$ en $V$ y todo escalar $c$ y $d$ en $\r$, se 
	cumplen las siguientes propiedades: 
	
	\vspace{-3mm}
	\begin{multicols}{2}		
		\begin{enumerate}			
			\justifying
			\item $\mathbf{u}+\mathbf{v}$ está en $V$. %\\[4mm]			
			\item $\mathbf{u}+\mathbf{v} = \mathbf{v}+\mathbf{u}$. %\\[3mm]			
			\item $(\mathbf{u}+\mathbf{v})+\mathbf{w} = \mathbf{u}+(\mathbf{v}+\mathbf{w})$. %\\[4mm]			
			\item Existe en $V$ un vector cero $\mathbf{0}$ tal que $\mathbf{u}+\mathbf{0} = \mathbf{u}$.		
			\item Para cada $\mathbf{u}$, existe un vector $-\mathbf{u}$ tal que
			$ \mathbf{u}+(-\mathbf{u}) = \mathbf{0}$.
			\columnbreak
			\item $c\mathbf{u}$ está en $V$. 
			\item $c(\mathbf{u}+\mathbf{v}) = c\mathbf{u} + c\mathbf{v}$.
			\item $(c+d)\mathbf{u} = c\mathbf{u} + d\mathbf{u}$.
			\item $c(d\mathbf{u}) = (cd)\mathbf{u}$.
			\item $1\mathbf{u} = \mathbf{u}$.
		\end{enumerate}		
	\end{multicols}
	
	\vspace{-2mm}
\end{block}

\begin{ej}{\textbf{Ejemplo 2}}\justifying
	El conjunto $M_{23}$ de todas las matrices $2\times 3$, con las operaciones de suma de matrices
	y multiplicación por escalares es un espacio vectorial.
\end{ej}

\end{frame}
}

%%------------------------------------------------------------------------------------------------------

\subsection{}

{\nologo
\begin{frame}%\frametitle{El espacio vectorial de todos los poliomios de grado menor o igual que 2}

\vspace{-2.5mm}
\begin{ej}{\textbf{Ejemplo 3}}\justifying
	Considere el conjunto $P_2$ de todos los poliomios de la forma 
	\[
		p(x) = a_2x^2 + a_1x + a_0,
	\]
	donde $a_0,a_1,a_2$ son números reales. La \textit{suma} de dos polinomios 
	\[
		p(x) = a_2x^2 + a_1x + a_0 \qquad \text{y} \qquad q(x) = b_2x^2 + b_1x + b_0
	\]
	se define como
	\[
		(p + q)(x) = (a_2+b_2)x^2 + (a_1+b_1)x + (a_0+b_0)
	\]
	y la \textit{multiplicación por escalar} del polinomio $p(x) = a_2x^2 + a_1x^1 + a_0$ por el escalar $c$ se define como
	\[
		(cp)(x) = ca_2x^2 + ca_1x + ca_0.
	\]
	Demuestre que $P_2$ es un espacio vectorial.
\end{ej}

\vspace{-2mm}
\begin{alertblock}{\textbf{Observación 1}}
	$P_n$ se define como el conjunto de todos los polinomios de grado menor o igual que $n$, junto con el 
	polinomio cero.
\end{alertblock}

\end{frame}
}

%%------------------------------------------------------------------------------------------------------

\subsection{}

\begin{frame}%\frametitle{El espacio vectorial de todos los poliomios de grado menor o igual que 2}


\vspace{-2mm}
\begin{ej}{\textbf{Ejemplo 4}}\justifying
	Considere el conjunto $\mathcal{F}$ de todas las funciones de valor real definidas en la recta numérica.
	La \textit{suma} de dos funciones $f$ y $g$ en $\mathcal{F}$ se define como
	\[
		(f+g)(x) = f(x) + g(x)
	\]
	y la \textit{multiplicación por escalar} de una función $f$ en $\mathcal{F}$ por el escalar $c$ se define como
	\[
		(cf)(x) = cf(x).
	\]
	Demuestre que $\mathcal{F}$ es un espacio vectorial.
\end{ej}

\vspace{-1mm}

\begin{figure}	
	\begin{subfigure}[b]{0.45\textwidth}
		\centering
		\begin{tikzpicture}[thick,scale=0.4, every node/.style={scale=0.6}]%[scale=.8,font=\scriptsize]
		% axis
		\draw[help lines,black,dotted] (-5,-3) grid (5,5);
		\draw[thick,-latex] (-5,0) -- (5,0) node[below] {\large $x$};
		\draw[thick,-latex] (0,-3) -- (0,5) node[above] {\large $y$};
		% Ticks
		\foreach \x in {1,...,4}
		\draw (\x,1pt) -- (\x,-3pt) node[anchor=north] {\x};
		\foreach \x in {-4,...,-1}		
		\draw (\x,1pt) -- (\x,-3pt) node[anchor=north] {\x};	
		\foreach \y in {1,...,4}
		\draw (1pt,\y) -- (-3pt,\y) node[anchor=east] {\y}; 		
		\foreach \y in {-2,...,-1}
		\draw (1pt,\y) -- (-3pt,\y) node[anchor=east] {\y}; 
		% f
		\draw[line width=0.2mm,samples=200,draw=blue,domain=-5:5] plot(\x,{cos(2\x r)-sin(2*\x r)-cos(3*\x r)+0.5});
		\fill[color=blue,draw] (-1,3) node[above] { $f$};	
		% 0.5f
		\draw[line width=0.2mm,samples=200,draw=verde,domain=-5:5] plot(\x,{0.5*(cos(2\x r)-sin(2*\x r)-cos(3*\x r)+0.5)});	
		\fill[color=verde,draw] (-3,2.5) node[above] { $0.5f$};	
		\draw [line width=0.1mm,color=verde,->] (-2.5,2.7) -- (-1,1.5);
		% 1.5f
		\draw[line width=0.2mm,samples=200,draw=red,domain=-5:5] plot(\x,{1.5*(cos(2\x r)-sin(2*\x r)-cos(3*\x r)+0.5))});		
		\fill[color=red,draw] (-1,4.5) node[left] { $1.5f$};	
		\end{tikzpicture}
		\caption{${\color{blue}f}, {\color{verde}0.5f}, {\color{red}1.5f}$}
	\end{subfigure}
	\hfill
	\begin{subfigure}[b]{0.45\textwidth}
		\centering
		\begin{tikzpicture}[thick,scale=0.4, every node/.style={scale=0.6}]%[scale=.8,font=\scriptsize]
		% axis
		\draw[help lines,black,dotted] (-5,-3) grid (5,5);
		\draw[thick,-latex] (-5,0) -- (5,0) node[below] {\large $x$};
		\draw[thick,-latex] (0,-3) -- (0,5) node[above] {\large $y$};
		% Ticks
		\foreach \x in {1,...,4}
		\draw (\x,1pt) -- (\x,-3pt) node[anchor=north] {\x};
		\foreach \x in {-4,...,-1}		
		\draw (\x,1pt) -- (\x,-3pt) node[anchor=north] {\x};	
		\foreach \y in {1,...,4}
		\draw (1pt,\y) -- (-3pt,\y) node[anchor=east] {\y}; 		
		\foreach \y in {-2,...,-1}
		\draw (1pt,\y) -- (-3pt,\y) node[anchor=east] {\y}; 	
		% f
		\draw[line width=0.2mm,samples=200,draw=blue,domain=-5:5] plot(\x,{cos(\x r)-sin(2*\x r)-cos(3*\x r)});		
		\fill[color=blue,draw] (1,0.6) node[above] { $f$};	
		% g
		\draw[line width=0.2mm,samples=200,draw=verde,domain=-5:5] plot(\x,{sin(\x r)+cos(\x r)+1});
		\fill[color=verde,draw] (2.2,1.5) node[above] { $g$};	
		% f+g
		\draw[line width=0.2mm,samples=200,draw=red,domain=-5:5] plot(\x,{cos(\x r)-sin(2*\x r)-cos(3*\x r)+sin(\x r)+cos(\x r)+1});		
		\fill[color=red,draw] (1,3) node[above] { $f+g$};	
		\end{tikzpicture}
		\caption{${\color{blue} f}, {\color{verde} g}, {\color{red} f+g}$}
	\end{subfigure}
	%	\caption{Pictures of animals}\label{fig:animals}
\end{figure}

\end{frame}

%%------------------------------------------------------------------------------------------------------

\subsection{}

{\nologo
\begin{frame}\frametitle{Un conjunto que no es espacio vectorial}

%\begin{exampleblock}{\textbf{Definición 2}}
\begin{block}{\textbf{Axiomas de un espacio vectorial (real) $V$}}	
	\justifying
	Para todo $\mathbf{u}, \mathbf{v}$ y $\mathbf{w}$ en $V$ y todo escalar $c$ y $d$ en $\r$, se 
	cumplen las siguientes propiedades: 
	
	\vspace{-3mm}
	\begin{multicols}{2}		
		\begin{enumerate}			
			\justifying
			\item $\mathbf{u}+\mathbf{v}$ está en $V$. %\\[4mm]			
			\item $\mathbf{u}+\mathbf{v} = \mathbf{v}+\mathbf{u}$. %\\[3mm]			
			\item $(\mathbf{u}+\mathbf{v})+\mathbf{w} = \mathbf{u}+(\mathbf{v}+\mathbf{w})$. %\\[4mm]			
			\item Existe en $V$ un vector cero $\mathbf{0}$ tal que $\mathbf{u}+\mathbf{0} = \mathbf{u}$.		
			\item Para cada $\mathbf{u}$, existe un vector $-\mathbf{u}$ tal que
			$ \mathbf{u}+(-\mathbf{u}) = \mathbf{0}$.
			\columnbreak
			\item $c\mathbf{u}$ está en $V$. 
			\item $c(\mathbf{u}+\mathbf{v}) = c\mathbf{u} + c\mathbf{v}$.
			\item $(c+d)\mathbf{u} = c\mathbf{u} + d\mathbf{u}$.
			\item $c(d\mathbf{u}) = (cd)\mathbf{u}$.
			\item $1\mathbf{u} = \mathbf{u}$.
		\end{enumerate}		
	\end{multicols}
	
	\vspace{-2mm}
\end{block}

\begin{ej}{\textbf{Ejemplo 5}}\justifying
	El conjunto $\z$ de todos los números enteros con las operaciones usuales de suma
	y multiplicación por escalar (producto de enteros) \textbf{no} es un espacio vectorial.
\end{ej}

\end{frame}
}

%%------------------------------------------------------------------------------------------------------

\subsection{}

{\nologo
\begin{frame}\frametitle{Un conjunto que no es espacio vectorial}

%\begin{exampleblock}{\textbf{Definición 2}}
\begin{block}{\textbf{Axiomas de un espacio vectorial (real) $V$}}	
	\justifying
	Para todo $\mathbf{u}, \mathbf{v}$ y $\mathbf{w}$ en $V$ y todo escalar $c$ y $d$ en $\r$, se 
	cumplen las siguientes propiedades: 
	
	\vspace{-3mm}
	\begin{multicols}{2}		
		\begin{enumerate}			
			\justifying
			\item $\mathbf{u}+\mathbf{v}$ está en $V$. %\\[4mm]			
			\item $\mathbf{u}+\mathbf{v} = \mathbf{v}+\mathbf{u}$. %\\[3mm]			
			\item $(\mathbf{u}+\mathbf{v})+\mathbf{w} = \mathbf{u}+(\mathbf{v}+\mathbf{w})$. %\\[4mm]			
			\item Existe en $V$ un vector cero $\mathbf{0}$ tal que $\mathbf{u}+\mathbf{0} = \mathbf{u}$.		
			\item Para cada $\mathbf{u}$, existe un vector $-\mathbf{u}$ tal que
			$ \mathbf{u}+(-\mathbf{u}) = \mathbf{0}$.
			\columnbreak
			\item $c\mathbf{u}$ está en $V$. 
			\item $c(\mathbf{u}+\mathbf{v}) = c\mathbf{u} + c\mathbf{v}$.
			\item $(c+d)\mathbf{u} = c\mathbf{u} + d\mathbf{u}$.
			\item $c(d\mathbf{u}) = (cd)\mathbf{u}$.
			\item $1\mathbf{u} = \mathbf{u}$.
		\end{enumerate}		
	\end{multicols}
	
	\vspace{-2mm}
\end{block}

\begin{ej}{\textbf{Ejemplo 6}}\justifying
	Sea $V=\r^2$ con la definición usual de suma, pero la multiplicación por escalar es la siguiente:
	\[
		c(x_1,x_2) = (cx_1,0).
	\]
	Demuestre que $V$ \textbf{no} es espacio vectorial.
\end{ej}

\end{frame}
}

%------------------------------------------------------------------------------------------------------

\subsection{}

\begin{frame}\frametitle{Propiedades de la multiplicación escalar}

\begin{prop}{\textbf{Propiedad 3}}
	\justifying
	Sea $\mathbf{v}$ un vector de un espacio vectorial $V$ y $c$ un escalar. Entonces:
	\begin{enumerate}
		\item[\labelname{$a$}] $0\mathbf{v} = \mathbf{0}$.
		\item[\labelname{$b$}] $c \mathbf{0} = \mathbf{0}$.
		\item[\labelname{$c$}] Si $c \mathbf{v} = \mathbf{0}$, entonces $c=0$ ó $\mathbf{v} = \mathbf{0}$.
		\item[\labelname{$d$}] $(-1)\mathbf{v} = -\mathbf{v}$.
	\end{enumerate}
\end{prop}

\end{frame}
