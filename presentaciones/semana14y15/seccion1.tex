\section{Diagonalización ortogonal}

\subsection{}

\begin{frame}\frametitle{Introducción}
	
	\begin{ej}{\textbf{Ejemplo 1}}
		Si es posible, diagonalice la matriz 
		\[
		A =
		\left(
		\begin{array}{rr}
		1 & 2 \\[1mm]
		2 & -2
		\end{array}
		\right).
		\]	
	\end{ej}
	
\end{frame}

% ---------------------------------------------------------------------------------------------------

\subsection{}

\begin{frame}\frametitle{Matrices diagonalizables ortogonalmente}
	
	\begin{defi}{\textbf{Definición 1}}\justifying
		\justifying
		Una matriz cuadrada $A$ es \textbf{\textit{diagonalizable ortogonalmente}} si existe una matriz 
		ortogonal $Q$ y una matriz diagonal $D$ tales que
		\[
		Q^TAQ = D.
		\]
	\end{defi}	
	
	\begin{prop}{\textbf{Propiedad 1}}
		\justifying
		Si $A$ es diagonalizable ortogonalmente, entonces $A$ es simétrica.
	\end{prop}	
	
\end{frame}

% ---------------------------------------------------------------------------------------------------

\subsection{}

\begin{frame}\frametitle{Propiedad de las matrices simétricas}
	
	\begin{prop}{\textbf{Propiedad 2}}
		\justifying
		Si $A$ una matriz simétrica con entradas reales, entonces los valores propios de $A$ son reales.		
	\end{prop}	
	
\end{frame}

% ---------------------------------------------------------------------------------------------------

\subsection{}

\begin{frame}\frametitle{Propiedad de las matrices simétricas}
	
	\begin{prop}{\textbf{Propiedad 3}}
		\justifying
		Sea $A$ una matriz simétrica. Si $\lambda_1$ y $\lambda_2$ son valores propios \textit{distintos} de $A$,
		entonces sus correspondientes vectores propios $\mathbf{v}_1$ y $\mathbf{v}_2$ son \textit{ortogonales}.		
	\end{prop}	
	
\end{frame}

% ---------------------------------------------------------------------------------------------------

\subsection{}

\begin{frame}\frametitle{Vectores propios de una matriz simétrica}
	
	\begin{ej}{\textbf{Ejemplo 2}}
		Muestre que cualquier par de vectores propios de 
		\[
		A =
		\left(
		\begin{array}{rr}
		3 & 1 \\[1mm]
		1 & 3
		\end{array}
		\right),
		\]	
		correspondientes a valores propios \textit{distintos}, son ortogonales.
	\end{ej}
	
\end{frame}

% ---------------------------------------------------------------------------------------------------

\subsection{}

\begin{frame}\frametitle{Teorema espectral}
	
	\begin{prop}{\textbf{Propiedad 4 (Teorema espectral)}}
		\justifying
		Sea $A$ una matriz cuadrada con entradas reales. Entonces $A$ es simétrica si y sólo si $A$ es 
		\textit{diagonalizable ortogonalmente}.			
	\end{prop}	
	
\end{frame}

%%------------------------------------------------------------------------------------------------------

\subsection{}

\begin{frame}\frametitle{Matrices diagonalizables ortogonalmente}
		
	\begin{ej}{\textbf{Ejemplo 3}} \justifying 
		Determine cuáles de las siguientes matrices son \textit{diagonalizables ortogonalmente}.
		\[
		A_1 =
		\left(
		\begin{array}{rrr}
		 1  & 2 & 3 \\[1mm]
		 2  & 0 & 1 \\[1mm]
		 3  & 1 & 1
		\end{array}
		\right),\quad
		A_2 =
		\left(
		\begin{array}{rrr}
		 3 & 2 & 1 \\[1mm]
		 2 & 1 & 7 \\[1mm]
		-1 & 7 & 0
		\end{array}
		\right),\quad
		% https://math.stackexchange.com/a/1658399
		A_3 =
		\left(
		\begin{array}{cc}
		1 & i \\[1mm]
		i & -1
		\end{array}
		\right).
		\]	
	\end{ej}
	
\end{frame}

%%------------------------------------------------------------------------------------------------------

\subsection{}

{\nologo 
\begin{frame}\frametitle{Procedimiento de diagonalización ortogonal}
	
	\vspace{0mm}
	\begin{defi}{\textbf{Definición 1}}\justifying
		\justifying
		Una matriz cuadrada $A$ es \textbf{\textit{diagonalizable ortogonalmente}} si existe una matriz 
		ortogonal $Q$ y una matriz diagonal $D$ tales que
		\[
		Q^TAQ = D.
		\]
	\end{defi}	
	
	\vspace{-1mm}
	\begin{ejem}{\textbf{Procedimiento 1}}\justifying
		\justifying
		Sea $A$ una matriz simétrica $n\times n$. 
		\begin{enumerate}\justifying
			\item Halle los vectores propios de $A$ y la multiplicidad de cada uno.
			\item Para cada vector propio de multiplicidad 1, elija un vector propio unitario (normalice el vector propio).
			\item Para cada vector propio de multiplicidad $k\geq 2$, encuentre un conjunto 
			de $k$ vectores propios linealmente independientes. Si este conjunto no es ortonormal,
			aplique el proceso de ortonormalización de Gram-Schmidt.			
			\item La aplicación de los pasos (2) y (3) generan un conjunto ortonormal de $n$ vectores propios.
			Utilice estos vectores para formar las columnas de $Q$. La matriz $Q^TAQ=D$ será diagonal.
			%(Los elementos en la diagonal principal son los valores propios de $A$).
		\end{enumerate}
	\end{ejem}		
	
\end{frame}
}

%%------------------------------------------------------------------------------------------------------

\subsection{}

\begin{frame}\frametitle{Ejemplo de una matriz diagonalizable}
	
	\begin{ej}{\textbf{Ejemplo 4}}
		Diagonalice ortogonalmente la matriz
		\[
		A =
		\left(
		\begin{array}{rrr}
			2 & 2 & -2 \\[1mm]
			2 & -1 & 4 \\[1mm]
			-2 & 4 & -1
		\end{array}
		\right).
		\]	
	\end{ej}
	
\end{frame}

% ---------------------------------------------------------------------------------------------------

\subsection{}

{\nologo 
\begin{frame}\frametitle{Consecuencias del teorema espectral}
	
	\begin{itemize}
		\item Si $A$ es una matriz simétrica, entonces			
		
		\[
		A = QDQ^T 
		=
		\left(
		\begin{array}{@{\hspace{0.3\tabcolsep}}c@{\hspace{0.5\tabcolsep}}c@{\hspace{0.5\tabcolsep}}c@{\hspace{0.3\tabcolsep}}}
		\mathbf{q}_1 & \cdots & \mathbf{q}_n
		\end{array}
		\right)
		\left(
		\begin{array}{@{\hspace{0.3\tabcolsep}}c@{\hspace{\tabcolsep}}c@{\hspace{\tabcolsep}}c@{\hspace{0.3\tabcolsep}}}
		\lambda_1 & \cdots & 0 \\[0mm]
		        0 & \ddots & 0 \\[1.5mm]
				0 & 0 & \lambda_n
		\end{array}
		\right)
		\left(
		\begin{array}{@{\hspace{0.3\tabcolsep}}c@{\hspace{0.3\tabcolsep}}}
		\mathbf{q}_1^T \\[1mm]
		\vdots  \\[1mm]
		\mathbf{q}_n^T
		\end{array}
		\right)
		\]	
		
		\vspace{2mm}
		\[
		\phantom{A = QDQ^T} 
		=		
		\left(
		\begin{array}{@{\hspace{0.3\tabcolsep}}c@{\hspace{0.5\tabcolsep}}c@{\hspace{0.5\tabcolsep}}c@{\hspace{0.3\tabcolsep}}}
		\lambda_1\mathbf{q}_1 & \cdots & \lambda_n\mathbf{q}_n
		\end{array}
		\right)		
		\left(
		\begin{array}{@{\hspace{0.3\tabcolsep}}c@{\hspace{0.3\tabcolsep}}}
		\mathbf{q}_1^T \\[1mm]
		\vdots  \\[1mm]
		\mathbf{q}_n^T
		\end{array}
		\right) \hspace{1.5cm}
		\]	
		
		\vspace{2mm}
		\[
		\phantom{A = QDQ^T} 
		=		
		\lambda_1\mathbf{q}_1\mathbf{q}_1^T + \lambda_2\mathbf{q}_2\mathbf{q}_2^T + \cdots + \lambda_n\mathbf{q}_n\mathbf{q}_n^T  		
		\hspace{3mm}
		\]	

	\end{itemize}
	
	\begin{defi}{\textbf{Definición 2}}\justifying
		\justifying
		La \textbf{\textit{descomposición espectral}} de una matriz simétrica $A$ se define como
		\[
			A=\lambda_1\mathbf{q}_1\mathbf{q}_1^T + \lambda_2\mathbf{q}_2\mathbf{q}_2^T + \cdots + \lambda_n\mathbf{q}_n\mathbf{q}_n^T
		\]
		%se le conoce como la \textbf{\textit{descomposición}} espectral de $A$.
	\end{defi}	

\end{frame}
}

%%------------------------------------------------------------------------------------------------------

\subsection{}

\begin{frame}\frametitle{Matrices diagonalizables ortogonalmente}
	
	\begin{ej}{\textbf{Ejemplo 5}} \justifying 
		Encuentre una matriz $2\times 2$ con valores propios $\lambda_1=3$ y $\lambda_2=-2$, y correspondientes vectores propios
		\[
		\mathbf{v}_1 =
		\left(
		\begin{array}{@{\hspace{0.5\tabcolsep}}c@{\hspace{0.5\tabcolsep}}}
		3 \\[1mm]
		4 
		\end{array}
		\right)
		\qquad \text{y} \qquad 
		\mathbf{v}_2  =
		\left(
		\begin{array}{@{\hspace{0.3\tabcolsep}}r@{\hspace{0.3\tabcolsep}}}
		-4 \\[1mm]		
		 3
		\end{array}
		\right).
		\]	
	\end{ej}
	
\end{frame}